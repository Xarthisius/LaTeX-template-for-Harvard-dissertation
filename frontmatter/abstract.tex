% the abstract

Przedmiotem rozważań zaprezentowanych w niniejszej pracy jest niestabilność
strumieniowa wynikająca z oddziaływania pyłu i gazu w dysku protoplanetarnym z
udziałem i bez udziału samograwitacji. Praca opiera się na serii symulacji
numerycznych fragmentu dysku protoplanetarnego w dwu i trzech wymiarach oraz
liniowej analizy stabilności, które pozwoliły potwierdzić hipotezę, iż w
globalnym dysku gazowo-pyłowym, w którym oba składniki są traktowana jako płyn,
znaczącą rolę w powstawaniu gęstych obiektów pyłowych odgrywa niestabilność
strumieniowa. Istotnym osiągnięciem jest również stwierdzenie, że niestabilność
strumieniowa w globalnym dysku gazowo-pyłowym, w obecności samograwitacji
materii, może prowadzić do wytworzenia się grawitacyjnie związanych obiektów.
Przedstawiony materiał pozwala również na oszacowanie całkowitej masy,
liczby i spektrum masowego powstających, związanych grawitacyjnie zagęszczeń
pyłowych.


% vim: tw=80 ts=3: 
