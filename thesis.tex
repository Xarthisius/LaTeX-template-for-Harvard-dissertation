%!TEX TS-program = xelatex
%!TEX encoding = UTF-8 Unicode

\documentclass{harvard-thesis}

%\usepackage{astron}
%\usepackage{polski}
%\usepackage[utf8]{inputenc}
\usepackage[polish]{babel}
%\usepackage[T1]{fontenc}

%%%%%%%%%%%%%%%%%%%%%%%%%%%%%%%%%%%%%%%%%%%%%%%%%%%%%%%%%%%%%%%%%%%%%%%%
%%% LaTeX Package Summary (http://trackchanges.sourceforge.net/)
%%% The trackchanges.sty style file adds five new LaTeX comands commands:
%%% \note[editor]{The note}
%%% \annote[editor]{Text to annotate}{The note}
%%% \add[editor]{Text to add}
%%% \remove[editor]{Text to remove}
%%% \change[editor]{Text to remove}{Text to add}
%%%
%%% In all cases [editor] can be ommitted.
%%%
%%% All of the TrackChanges commands allow for the specification of an editor. Specifing an editor will prefix the edits with the editor name and color code their changes in the final pdf or dvi file.
%%%
%%% To specify an editor name, the editor must first be declared in the preamble:
%%% \addeditor{editor one}
%%% \addeditor{editor two}
%%%
%%% Display Options
%%% Track changes has a number of different ways that it can display the edits in the final dvi or pdf file.
%%% finalold  - Reject all edits.
%%% finalnew  - Accept all edits.
%%% footnotes - Display edits as footnotes.
%%% margins   - Display edits as margin notes.
%%% inline    - Display edits inline.
%%%
%%% See the Documentation for additional options.
%%% Examples with the differnt display options can be found on the Preview page.
%%%%%%%%%%%%%%%%%%%%%%%%%%%%%%%%%%%%%%%%%%%%%%%%%%%%%%%%%%%%%%%%%%%%%%%%
\usepackage[inline]{trackchanges}           %%% !!! USE THIS TO DISPLAY ALL CHANGES AND NOTES  !!!
\addeditor{MH}

\begin{document}

% the front matter
% some details about the thesis
\title{Połączone działanie niestabilności płynowych w dyskach protoplanetarnych}
\author{Kacper Kowalik}
\advisor{prof. dr hab. Michał Hanasz}

% about the degree
\degree{Rozprawa doktorska}
\field{Astronomia}
\degreeyear{2014}
\degreemonth{wrzesień}

% about the university
\department{Katedrze Astronomii i Astrofizyki}
\university{Uniwersytet Mikołaja Kopernika}
\universitycity{Toruń}
\universitystate{}

\maketitle
\copyrightpage
\abstractpage
\tableofcontents
%\authorlist
\listoffigures
\dedicationpage
\acknowledgments

\onehalfspacing

% include each chapter...
\begin{savequote}[75mm]
vulputate egestas, eros pede varius leo.  \qauthor{Quoteauthor Lastname}
\end{savequote}

\chapter{Wprowadzenie}
Teoretyczne rozważania dotyczące pochodzenia planet mają długą historię
sięgającą przynajmniej XVIII wieku, kiedy to Immanuel Kant wysunął \dq{}Hipotezę
mgławicową\dq{}. Już wtedy unikalność Układu Słonecznego stanowiła przedmiot
debaty. Dopiero na początku XX wieku pogląd, iż układy planetarne są czymś
powszechnym we Wszechświecie, został zaakceptowany przez środowisko naukowe, a w
roku 1992 została odkryta pierwsza, pozasłoneczna planeta orbitująca pulsar PSR
1257+12 b~\cite{1992Natur.355..145W}.

W połowie XVIII wieku, filozof francuski Immanuel Kant zasugerował, iż rozmyte
o\-biek\-ty obserwowane przez niego przez teleskop mogą być wyspowymi
Wszechświatami takimi jak nasz, bądź obłokami materii w których formują się
gwiazdy i planety~\cite{ImmanuelKant.etal:2008}.

\begin{figure}[!ht]
\centering
%\includegraphics[width=0.8\textwidth]{figures/laplace.png}
\caption{Model mgławicy Laplace'a: (a) rotująca mgławica; (b) kolapsująca
mgławica ulega spłaszczeniu wzdłuż osi rotacji; (c) soczewkowaty kształt
mgławicy; (d) pierścienie materii pozostawione przez zapadający się obiekt
centralny; (e) zagęszczenia na poszczególnych pierścieniach kolapsują tworząc
planety. (Obrazek z pracy Woolfson, 1993)} 
\label{fig:laplace}
\end{figure}

\section{Paradygmat formowania się planet}
Formowanie się planet jest nierozerwalnie związane z narodzinami gwiazd, które
biorą swój początek w gęstych pyłowo--gazowych obłokach materii. Zanurzone w
gorącym ośrodku międzygwiazdowym, początkowo w stanie równowagi termodynamicznej
z otaczającym je gazem, obłoki takie często występują w
ogromnych kompleksach i obserwowane są jako ciemne mgławice molekularne. {\bf
(see Tielens 2005 for a review on a phases of the ISM)} Procesy zachodządze w
samych obłokach, tj. turbulencja, samograwitacja, lub w ośrodku zewnętrznym tj.
wybuchy supernowych mogą powodować wzrost gęstości poszczególnych zagęszczęń w
obłoku. W momencie w którym obszar gęstej materii przekroczy  masę krytyczną,
nazywaną masą Jeansa $M_J$, grawitacja przeważa i chmura zaczyna się zapadać
(rysunek 1a). Masa Jeansa zależy od temperatury kinetycznej ośrodka $T$ oraz
jego gęstości $\rho$ \begin{equation} M_J \sim \left( \frac{k_B T}{G} \right)
^\frac{3}{2} {\rho}^{-\frac{1}{2}}, \end{equation} gdzie $k_B$ jest stałą
Boltzmanna a $G$ jest stałą grawitacji. Dla typowych warunków panujących
wewnątrz obłoków materii międzygwiazdowej, masa Jeansa przyjmuje wartość
\begin{equation}
 M_J \approx 2.9 M_{\odot} \left(\frac{T}{10\K}\right)^{1.5} 
 \left(\frac{n}{10^4\cm^{-3}}\right)^{-0.5},
\end{equation}
gdzie $n = \rho_g / \mu \mH$ jest koncentracją cząstek materii.
Kolaps obłoku następuje w tzw. skali czasowej spadku swobodnego
\begin{equation}
   t_{\textrm{ff}} \sim \frac{1}{\sqrt{G\rho}} \sim 10^5\yr
   \left(\frac{n}{10^4\thinspace \cm^{-3}}\right)^{-0.5},
\end{equation}
co z punktu widzenia całkowitego czasu potrzebnego do uformowania się planet
jest czasem relatywnie krótkim. Podczas kolapsu energia
grawitacyjna zapadającego się obłoku jest przekształcana w energię termiczną.  W
wypadku braku efektywnego mechanizmu chłodzenia się gazu, temperatura wewnątrz
obłoku wzrasta (a wraz z nią masa Jeansa) zatrzymując dalszy kolaps
grawitacyjny. Obserwacyjna klasyfikacja formujacego się układu protoplanetarnego
określa obiekty na tym etapie mianem obiektów \emph{klasy 0}~\cite{andre} (rysunek 1b).
Ich obserwowana temperatura jest stosunkowo niska $(T \lesssim
30\K)$. Obiekty tej klasy charakteryzują się maksimum w
rozkładzie energii promieniowania wypadającym w dalekiej podczerwieni, bez
wykrywalnej nadwyżki w bliskiej podczerwieni.
Kolejnym krokiem ewolucji zapdającego się obłoku jest rozpalenie reakcji
termojądrowych w centrum. Pole promieniowania pochodzące od protogwiazdy
podgrzewa opadająca materię i formujący się dysk do temperatury $T \sim
100\K$ tworząc obiekt \emph{klasy I} (rysunek 1c).  Część
akreująca z dysku materii w pewnym momencie zostaje uniesiona z gwiazdy w
postaci silnego wypływu, który wybija dziurę w otoczce progwiazdowej. Wydarzenie
to może być bardzo gwałtowne i znacznie wpłynąć na tempo akrecji i co za tym
idzie jasność obiektu. Gwałtowny wybuch zwyczajowo nazywa się \emph{rozbłyskiem
FU--Orionis}.
W obiektach \emph{klasy I} maksimum w rozkładzie energii promieniowania przesuwa
się w kierunku bliskiej podczerwieni, praktycznie wypłaszczając widmo.  \par Po
około $10^5$ lat, większość otoczki zostaje odrzucona przez wiatr, a dysk
protogwiazdowy zostaje zredukowany do kilku procent masy gwiazdy macierzystej.
Osiągana jest bardziej stabilna konfiguracja, tzw. faza \emph{T-Tauri} dla
gwiazd o masach mniejszych niż $2\thinspace\Msun$, badź faza \emph{Herbig Ae/Be}
dla gwiazd masywniejszych. Dla obiektów tej klasy (\emph{klasa II} rysunek 1d)
znacznie spada nadwyżka w podczerwieni. Okres ten trwa od $10^5$ do $10^6$ lat.

% Pozniej
%Na tym etapie musi nastąpić formacja planetezymali i gazowych olbrzymów,
%ponieważ po jej zakończeniu dysk protoplanetarny zanika.  Pozostaje gwiazda
%zbliżająca się do ciągu głównego, nie wykazująca już żadnej nadwyżki w
%podczerwieni --- obiekt \emph{klasy III} (rysunek 1e).

\begin{figure}[p]
\centering 
\includegraphics[width=0.9\textwidth]{figures/planetformation.png}
\caption{Ilustracja przedstawia kolejne fazy formowania się małomasywnej gwiazdy
   wraz z systemem planetarnym: a) kolaps grawitacyjny gęstego obłoku; b)
   oddziaływanie centralnego pola grawitacyjnego oraz siły odśrodkowej powoduje
   opadanie materii i formowanie się dysku; c) faza FU Orionis: silna akrecja w
   dysku oraz wypływ materii w okolicach osi obrotu; d) faza T Tauri: zmniejsza
   się tempo akrecji $\sim 10^{-8}\Msun\yt^{-1}$ oraz wypływu materii,
   rozpoczyna się proces formowania planet; e) zanika składowa gazowa, planety
otwierają przerwy w dysku, następuje również ich migracja; f) cały gaz oraz
mniejsze ciała zostają pochłonięte przez planety lub usunięte z dysku, układ
planetarny przyjmuje ostateczny kształt.}

\label{fig:planet}
\end{figure}

%\section{Faza T Tauri}

Podczas tej fazy większość energii wyświecanej przez gwiazdę pochodzi ciągle z
energii grawitacyjnej traconej podczas kontrakcji. Ewolucja zachodzi w tzw.
skali czasowej Kelvina--Helmholtza 

\begin{equation} 
   t_{KH} \sim  \frac{G{M_{\star}}^2}{R_{\star}L_{\star}} = 3\cdot 10^7 \yr
   \left(\frac{M_\star}{M_\odot}\right)^{2}
   \left(\frac{L_\star}{L_\odot}\right)^{-1}
   \left(\frac{R_\star}{R_\odot}\right)^{-1}
\end{equation}

gdzie $M_{\star}$, $R_{\star}$, $L_{\star}$ są odpowiednio masą, promieniem i
dzielnością promieniowania gwiazdy. Formująca się gwiazda powoli przesuwa się na
diagramie Hertzsprunga--Russela w kierunku ciągu głównego wieku
zerowego~\cite{palla}.  Dla gwiazd o masie Słońca ewolucja ta będzie przebiegała
po tzw. ścieżce Hayashiego, na której obiekty pozostają całkowicie konwektywne,
podczas gdy gwiazdy bardziej masywne utworzą promieniste jądra i przesuną się
w stronę wyższych temperatur. W tym czasie otaczający gwiazdę dysk traci
gaz na skutek fotoewaporacji i wiatru gwiazdowego. Zarówno obserwacje gwiazd
T-Tauri jak i rozważania teoretyczne~\cite{alexanderpassuci2013} {\bf and ref
therein} szacują że proces utraty gazu z dysku trwa od jednego do kilku Myr.
Stanowi to niewątpliwie limit na wytworzenie się gazowych olbrzymów, tj. w dysku
protoplanetarnym muszą wcześniej wytworzyć się obiekty zdolne akreaować wodór.

\par Po dyspersji dysku gazowego gwiazda powoli traci również nadwyżkę w
podczerwieni. Sugeruje to fakt, iż skala czasowa koagulacji pyłu w większe
ziarna jest porównywalna z czasem życia frakcji gazowej dysku. Ewolucję pyłu
można podzielić na zgrubsza na 4 etapy:
\begin{description}
   \item[i) koagulacja ziaren pyłu $\left(\mum \rightarrow \km\right):$] 
      Z doświadczeń laboratoryjnych wynika~\cite{me}, że drob\-ne cząsteczki pyłu
      mogą na skutek wzajemnych zderzeń zwiększać swoje rozmiary. ,,Spoiwem'' stają
      się siły van der Waalsa bądź oddziaływanie elektrostatyczne. Opierając się
      na analizie drogi swobodnej jednorodnej frakcji cząstek pyłu o promienu
      $a$, można określić charakterystyczną skalę czasową koagulacji jako 

   \begin{equation}\label{coag} 
      t_{\textrm{coag}} % = \frac{1}{n_d \sigma \Delta v}
      \sim \frac{a}{\Delta v}\frac{\rho_p}{\rho_d} \approx 
      10^{-12} \rho_d^{-1}\yr\thinspace
      \left(\frac{a}{1\mum}\right)
      \left(\frac{\Delta v}{0.1\m\s^{-1}}\right)^{-1}
      \left(\frac{\rho_p}{3\g\cm^{-3}}\right),
   \end{equation}

   gdzie $\Delta v$ jest średnią prędkością względna cząstek, $\rho_p$ jest
   gęstością materiału budującego cząstki natomiast $\rho_d$ jest gęstością
   ośrodka pyłowego w dysku.  Biorąc pod uwagę typowe gęstości pyłu w obłokach
   gwiazdowych $(\rho_d \sim 10^{-20}\g\cm^{-3})$ proces ten
   zachodzi w skali czasowej milionów lat. Jednakże dla dysków protoplanetarnych
   o typowych gęstościach rzędu $10^{-10}\g\cm^{-3}$\footnote{jest to całkowita
   gęstość z uwzględnieniem obu składników: gazu i pyłu. Przyjmuje się że
kanoniczna wartość stosunku gęstości pyłu do gęstości gazu $\epsilon$ wynosi
0.01. Zatem $\rho_d \sim 10^{-12}\g\cm^{-3}$} proces
koagulacji zachodzi w skalach lat czy tez dziesiątek lat i może bardzo szybko
prowadzić do wytworzenia się planetezymali. W rzeczywistości dla ziaren pyłu o
rozmiarach decymetrów czy metrów pojawia się szereg procesów przeciwdziałających
dalszemu wzrostowi, a także ulega zmianie średnia prędkość względna cząstek pyłu
zmieniając prawdopodobnięstwo wyniku kolizji na korzyść fragmentacji raczej niż
koagulacji.  

\item[ii) oligarchiczny wzrost $\left(\km \rightarrow
   10^3\km\right)$:]
   Faza druga formowania się planet rozpoczyna się w momencie w którym przeważa
   wzajemne oddziaływanie grawitacyjne pomiędzy planetezymalami i to grawitacja
   staje się nowym spoiwem łączącym zderzające się obiekty. Tarcie
   aerodynamiczne jest nadal niezaniedbywalne i zapewnia kołowość orbit
   planetezymali, co zwiększa szanse na zderzenia.
\item[iii) akrecja gazu]
   Po osiągnięciu rozmiarów rzędu $10^3\km$ jądra planetarne
   są w stanie wiązać grawitacyjnie gaz na swoich powierzchniach. Globalne
   oddziaływanie dysk $\iff$ planety staje się istotne i może prowadzić z jednej
   strony do migracji planet, a z drugiej strony do otworzenia się przerw w
   dysku.
\item[iv) długoskalowa ewolucja dynamiczna:]
   Faza zdominowana tylko i wyłącznie przez wzajemne oddziaływanie grawitacyjne
   pomiędzy utworzonymi planetami, a także z gwiazdą macierzystą. Układ
   planetarny może na tym etapie utracić znaczną część masy pyłowej, poprzez
   pochłonięcię planety przez gwiazdę, bądź wprowadzenie jej na orbitę
   hiperboliczną.
\end{description}
Powyższy scenariusz formowania się planet nosi nazwę modelu ,,akrecji na
jądra''~\footnote{ang. \emph{core-accretion}}. Alternatywną teorią, szczególnie
wdzięcznie wyjaśniająca powstawanie gazowych olbrzymów w masywnych dyskach
protoplanetarnych, jest model zakładający kolaps i fragmentację grawitacyjną
dysku~\cite{Boss}. Zostanie ona omówiona w dalszej części tego rozdziału.

\section{Ważne pojęcia}
Gazowo--pyłowy dysk, który powstaje podczas kolapsu obłoku protogwiazdowego jest
miejscem w którym powstają planety. Pewne szczególne mechanizmy oraz globalna
dynamika mogą temu procesowi pomagać, bądź mu przeciwdziałać. Poniższe akapity
pokrótce opisują strukturę dysku protoplanetarnego oraz najważniejsze efekty
dynamiczne związane z samym gazem, następnie przechodząc do ich wpływu na
dynamikę i ewolucję pyłu. Pozwoli to wskazać problemy z jakimi boryka się model
,,akrecji na jądra'' i naturalnie przejść do celu tej rozprawy.

\subsection{Struktura dysku protoplanetarnego}
Proces kolapsu grawitacyjnego, sferycznego obłoku materii międzygwiazdowej nie
wpływa na jego całkowity moment pędu. Przy założeniu, że obłok rotuje, nawet
bardzo wolno, to materia nie opada bezpośrednio na obiekt
centralny, lecz formuje dysk w płaszczyźnie prostopadłej do wektora całkowitego
momentu pędu. Aby określić przybliżone warunki fizyczne w formującym się dysku
możemy posłużyć się równaniami hydrodynamiki:

\begin{gather}
   \partial_t \rho_g + \nabla\cdot\left(\rho_g\mathbf{u}\right) = 0,
   \label{eq:hd1}\\
\partial_t \mathbf{u} + \left(\mathbf{u}\cdot\nabla\right)\mathbf{u} = 
-\nabla\Phi + -\frac{1}{\rho_g} \nabla P \label{eq:hd2}
\end{gather}

gdzie $\rho_g$ jest gęstością gazu, $P$ ciśnieniem, 
związanych z tarciem, a $\Phi$ potecjałem grawitacyjnym. Jeżeli ponadto
założymy, że dysk:

\begin{enumerate}
   \item jest izotermiczny to $P = c_s^2 \rho_g \implies -\frac{1}{\rho_g}\nabla
      P = -c_s^2\nabla\ln\rho_g$, gdzie $c_s = \sqrt{\frac{k_{\textrm{B}} T}{\mu
    \mH}}$ jest izotermiczną prędkością dźwięku.
   \item jest stacjonarny i znajduję się w równowadzę hydrostatycznej w kierunku
      $z$ to wertykalne przyspieszenie grawitacyjne $\partial_z \Phi = g_z =
      (GM_\star/r^2) z/r = \Omega^2 z$, gdzie $M_\star$ to masa gwiazdy
      macierzystej, $\Omega$ orbitalna częstość keplerowska, $G$ stała
      grawitacji Newtona, jest równoważone przez siłę wynikająca z gradientu
      ciśnienia gazu $\partial_z P / \rho_g$
\end{enumerate}

Łącząc powyższe założenia otrzymujemy rozkład gęstości gazu

\begin{equation}
   \rho_g(z) = \frac{\Sigma_G}{H\sqrt{2\pi}} \exp \left[
   \frac{1}{2}\left(\frac{z}{H}\right)^2 \right],
\end{equation}

gdzie $\Sigma_g = \int \rho_g(z) dz$ jest gęstością powierzchniową, a
$H=\frac{c_s}{\Omega}$ to charakterystyczna skala grubości dysku.

%\par Zakładając w pierwszym przybliżeniu że dysk jest optycznie gruby, t.j.
%absorbuję całkowicie promieniowanie pochodzące od gwiazdy i następnie reemituje
%je jako ciało doskonale czarne, można pokazać~\cite{armitage07} że $T \propto
%r^{-3/4}$ i co za tym idzie $c_s \propto r^{-3/8}$. Dokładniejsze szacunki,
%które lepiej oddają obserwowane dystrucje spektralne energii, można znaleźć w
%pracach~\cite{KenyonHART87, ChaingGold97} {\bf patrz armitage}.
%{\bf Tu raczej trzeba przedstawić tę wersję z której wynika $T \propto
%r^{-1/2}$}
%
\par W modelu stacjonarnym w kierunku radialnym grawitacja i siła wynikająca z
gradientu ciśnienia jest równoważona poprzez siłę odśrodkową
\begin{equation}\label{eq:radial_balance}
\frac{u_\phi^2}{r} = \frac{GM_\star}{r^2} +
  \frac{1}{\rho_g}\frac{\textrm{d}P}{\textrm{dr}},
\end{equation}
gdzie $u_\phi$ jest prędkością orbitalną gazu. Wpływ gradientu ciśnienia na
globalną dynamikę jest znikomy (rzędu $O(H/r)^2$), dlatego dla cienkich dysków
$(H/r \ll 1)$ z dobrym przybliżeniem można przyjąć, że właściwy moment pędu dla
gazu jest równy momentowi pędu wynikającemu z ruchu keplerowskiego. Z równania
\mref{eq:radial_balance} wynika zatem, że moment pędu jest rosnącą funkcją
promienia:
\begin{equation}\label{eq:angmom}
l = r^2\Omega = \sqrt{GM_\star r}.
\end{equation}
Z równania \mref{eq:angmom} wypływa niezwykle ważny fakt: aby materiał z dysku
mógł być akreaowany przez gwiazdę macierzystą, w układzie musi działać mechanizm
powodujący utratę bądź chociaż redystrybucję momentu pędu.
\subsection{Niestabilność magnetorotacyjna}
Z obserwacji jasno wynika~{\bf cytacja i rysunek 0402599?}, że dyski
protoplanetarne nie są obiektami stacjonarnymi. Co więcej, obiekty klasy I
posiadają wysokie tempa akrecji $\sim 10^{-5}\Msun\yr^{-1}$. Aby zapewnić
radialny przepływ materii w kierunku gwiazdy macierzystej, potrzeba wprowadzić
efektywny mechanizm transportu momentu pędu. W tym celu musimy odejść od
przybliżenia hydrodynamiki dla idealnego gazu i zastować równania
Naviera-Stokesa, które rozszerzają równanie ruchu~\ref{eq:hd2} o tensor naprężeń
płynu

\begin{gather}
   \partial_t \rho_g + \nabla\cdot\left(\rho_g\mathbf{u}\right) = 0,
   \label{eq:ns1}\\
\partial_t \mathbf{u} + \left(\mathbf{u}\cdot\nabla\right)\mathbf{u} = 
-\nabla\Phi -\frac{1}{\rho_g} \nabla P + \frac{1}{\rho_g} \nabla \cdot \Pi.
\label{eq:ns2}
\end{gather}
Jeżeli dodatkowo założymy, że mamy doczynienia z płynem doskonale lepkim, tensor
naprężeń można zredukować do postaci $\Pi = (\rho_g \nu)\nabla\cdot\mathbf{u}$,
gdzie $\nu$ jest lepkością kinematyczną. Całkując układ
równań~\mref{eq:ns1}-\mref{eq:ns2} w kierunku wertykalnym i dokonująć prostych
przekształceń można otrzymać
\begin{equation}\label{eq:sigma}
   \partial_t \Sigma_g =
   \frac{3}{R}\partial_R\left(\frac{1}{R\Omega}\partial_R\left(R^2\Sigma_g \nu
         \Omega\right)\right).
\end{equation}
Rozwiązaniem stacjonarnym równania~\mref{eq:sigma} jest warunek $\Sigma_g\nu =
\textrm{const}$, co przekłada się na tempo akrecji $\dot{M} = 3\pi\Sigma_g\nu$.
Z powyższego warunku wynika, że lepkość jest kluczowym elementem napędzającym
akrecję materii. Problemem pozostaje jakie proces fizyczny jest za nią
odpowiedzialny?
\par Podstawowa lepkość, tj. lepkość molekularna $\nu_{\textrm{m}} \sim c_s
\lambda$, gdzie $\lambda = 1 / n\sigma$ to średnia droga swobodna molekuł gazu,
zaś $n$ to koncetracja molekuł gazu, a $\sigma$ ich przekrój czynny, dla
typowych wartości gęstości i temperatury dysków protoplanetarnych wynosi
$\nu_{\textrm{m}}\sim10^5\cm^2\s^{-1}$~\cite{armitage}. Przekłada się to na
tempo akrecji na poziomie $\dot{M}\sim 10^{-17}\Msun\yr^{-1}$. Co więcej
charakterystyczna skala czasowa takiego procesu $\tau \simeq R^2 /
\nu_{\textrm{m}}$ wynosi $10^{13}\yr$. Z tego względu lepkość molekularną można
całkowicie zaniedbać w dalszych rozważaniach.
\par W słynnej pracy Shakura \& Sunyaev~\citep{SS73} zauważyli, że turbulencja
w dysku może być dobrym źródłem lepkości, znacznie wydajnieszym niż lepkość
molekularna. Dla izotropowej turbulencji, maksymalna skala wirów w dysku jest
proporcjonalna do charakterystycznej skali grubości dysku $H$, zaś maksymalna
prędkość ruchu turbulentnych jest rzędu prędkości dźwięku, ponieważ fale
uderzeniowe bardzo szybko dyssypują energię kinetyczną. Shakura \& Sunyaev
zaproponowali parametryzację lepkości turbulentnej

\begin{equation}\label{eq:alpha}
\nu = \alpha c_s H
\end{equation}

, gdzie $\alpha$ jest bezwymiarowym parametrem określającym wydajność
turbulencji w transporcie momentu pędu. Aby wyjaśnić obserwowane tempo akrecji
dla gwiazd \emph{T Tauri}, parameter $\alpha$ powinnien byc rzędu $10^{-2}$.
Problem dalej pozostaje źródło turbulencji w dysku. Z kryterium
Rayleigha

\begin{equation}
   \frac{\mathrm{d}}{\mathrm{d}r}\left(r^2\Omega\right) > 0,
\end{equation}

wynika że hydrodynamiczny dysk keplerowski jest liniowo stabilny. Sytuacja
diametralnie się zmienia w jeżeli uwzględnimy obecność w układzie dowolnie
słabego pola magnetycznego. Balbus \& Hawley~\citep{BH91} pokazali, że przepływ
magnetohydrodynamiczny jest stabilny liniowo wtedy i tylko wtedy, gdy

\begin{equation}\label{eq:mri}
   \frac{\mathrm{d}}{\mathrm{d}r}\left(\Omega^2\right) > 0.
\end{equation}
Warunek \mref{eq:mri} \emph{nie jest} spełniony dla dysków keplerowskich. W
rezultacie nawet szczątkowe pole magnetyczne, jest w stanie wzmocnić wykładniczo
zaburzenia gęstości w gazie w dynamicznej skali czasowej, powodując silną
turbulencję. Proces ten określany jest mianem niestabilności magnetorotacyjnej
(MRI). Co więcej, zarówno symulacje lokalne~\cite{Davis_Stone_Pessah_2010} jak
i globalne~\cite{Flock} szacują współczynnik $\alpha$ wynikający z turbulencji
wzbudzonej przez MRI na $\alpha\sim 10^{-2}$.

\begin{figure}
   \includegraphics[width=0.9\textwidth]{figures/chap1_sed.png}
   \caption{Rozkład widma energii dla gwiazdy HD 34282. Symbole oznaczają
      obserwacje różnymi metodami dla odpowiednich długości fali. Do obserwacji
      dopasowano następujące modele: linia przerywana model widma gwiazdowego
      dla obiektu typu A3V $(T\sim 8600\K)$, cienka ciągła
      linia model ciała doskonale czarnego dla $T=1400\K$,
      linia kropkowana reprezentuje model dysku o nachyleniu $i=56^o$, tempie
      akrecji $\dot{M} = 8.2\times10^{-9}\thinspace\Msun\yr^{-1}$
      rozciągającym się od $0.31\AU$ do $705\AU$.
   Obrazek pochodzi z pracy~\cite{0402599}}
\end{figure}

\subsection{Niestabilność grawitacyjna}
\subsection{Oddziaływanie pomiędzy gazem, a pyłem}
\subsection{Radialny dryf pyłu}
\subsection{Sedymentacja i pułapkowanie pyłu (KHI)}
\subsection{Koagulacja i wzrost rozmiarów}

W fazie T Tauri następuje koagulacja ziaren pyłu. Z perspektywy obserwatorów
jest ona równoważna usuwaniu drobnego pyłu z dysku a, co za tym idzie, spadkowi
nadwyżki promieniowania w zakresie podczerwonym. Dzięki pracom obserwacyjnym
wiemy, że zanik tej nadwyżki dla długości fal charakterystycznych dla
wewnętrznych obszarów dysków (wyższa temperatura) zajmuje ok. $10^6$ lat
\cite{hillenbrand}, natomiast zanik chłodnej składowej pyłowej następuje w
skalach czasowych ok. $2$ rzędy wielkości większych \cite{carpenter}.


%\section{Metrowa bariera wzrostu}
{\it
Cząstki pyłu sklejają się dzięki oddziaływaniom międzycząsteczkowym, które są
szczególnie wydajne w przypadku małych cząstek i dużego ich zagęszczenia. Takie
warunki występują w dysku protoplanetarnym. Wraz ze wzrostem cząstek proces
zlepiania zachodzi coraz wolniej. Koagulacja faworyzuje małe cząstki, które mają
większy stosunek powierzchni do masy. Prędkości względne cząstek w ogólności
rosną, gdy cząstki osiągną masę pozwalającą na odsprzęgnięcie się od gazu, co
również utrudnia dalszy wzrost. Nawet bez silnych ruchów turbulentnych gazu, w
odległości $1$ j.a. od gwiazdy metrowe cząstki osiągają prędkości względne rzędu
$10$ m/s. Zderzenia przy tak wysokich prędkościach względnych prowadzą do
fragmentacji cząstek. Zjawisko to jest w literaturze nazywane metrową barierą
wzrostu (ang. {\it meter-size barrier}) \cite{ormel}. Co więcej, oddziaływanie
cząstek z gazem powoduje utratę momentu pędu i szybki dryf radialny cząstek w
kierunku gwiazdy. Stanowi to~kolejne ograniczenie czasowe ($\sim 10^2$ lat) dla
procesu formowania się planet.

Aby planety mogły powstać, metrowa bariera wzrostu musi być pokonana w
stosunkowo krótkiej skali czasowej. Nie jest jasne, czy zwykłe mechanizmy
sklejania są wystarczające aby otrzymać cząstki o rozmiarach rzędu kilometrów,
dla których dopiero zaczynają odgrywać rolę oddziaływania grawitacyjne. Istnieje
alternatywna teoria, w której metrowa bariera wzrostu jest pokonywana dzięki
wystąpieniu w pyle niestabilności grawitacyjnej. Niestabilność ta wymaga jednak
wysokich koncentracji cząstek o stosunkowo niskich prędkościach względnych.
Możliwość występowania niestabilności grawitacyjnej w pyłowej składowej dysków
protoplanetarnych jest wciąż przedmiotem naukowej dyskusji \cite{gi1,gi2}.
}
\subsection{Niestabilność strumieniowa}
\section{Cel pracy}
Głównym celem pracy jest zbadanie niestabilności strumieniowej w bardziej
realistycznym przyblizeniu radialnie rozciągłego dysku.

\newthought{Skąd wzięły się planety?} Pytanie które nurtuję ludzkość od
zamierzchłych czasów 

\section{Luźne myśli}

Formowanie się planet jest złożonym procesem, który wymaga wzrostu rozmiaru
mikrometrowych ziaren pyłu o wiele rzędów wielkości. Pomimo tego, że najmniejsze
drobiny pyłu są silnie sprzężone z gazem, mogą one dryfować zarówno w kierunku
radialnym jak oraz wertykalnym i podlegać wzajemnym zderzeniom. Przy odpowiednio
niskiej prędkości względnej, taka kolizja może prowadzić do tworzenia coraz to
większych aglomeratów cząstek~\citep{BW08}. Z drugiej strony, ciała o rozmiarach
setek czy tysięcy metrów są na tyle duże, iż opór aerodynamiczny stawiany na nie
przez gaz jest całkowicie zaniedbywalne, zaś dominująca dynamicznie siłą są
wzajemne oddziaływania grawitacyjne~\citep{KKI06}.

\par Nierozwiązaną dotąd zagadką współczesnej astrofizyki jest pośredni etap
wzrostu centymetrowych ziaren pyłu do kilometrowych głazów stanowiących budulec
planet. Istnieje szereg procesów, które przeciwdziałają możliwemu wzrostowi
rozmiaru ziaren pyłu lub nakładają silne więzy czasowe na formację systemów
planetarnych. Najsilniejszym ograniczeniem jest szybki radialny dryf dla ziaren
pyłu luźno związanych z gazem, tj. takich dla których charakterystyczna skala
czasowa dla tarcia aerodynamicznego jest porównywalna z ich okresem
orbitalnym~\citep{W77}. Ponadto typowe prędkości drobiny pyłu o rozmiarach od
$1\textrm{ cm}$ do $1\textrm{ m}$ zawierają się w przedziale $1\div10\textrm{ m
s}^{-1}$, co sprawia że najbardziej prawdopodobnym rezultatem zderzenia jest
fragmentacja bądź odbicie~\citep{Z10}.

\par Jednym z możliwych scenariuszy formowania się planet jest szybki wzrost
gęstości pyłu na skutek sedymentacji ziaren, którą wymusza pionowa składowa
grawitacji pochodzącej od gwiazdy macierzystej. Po przekroczeniu wartości
krytycznej gęsta warstwa pyłu rozpada się pod własnym ciężarem~\citep{GW73}.
Należy jednak mieć na uwadzę, że sam proces sedymentacji może prowadzić do
wzbudzenia się niestabilności Kelvina-Helmholza~\cite{JHK06}, a to przeciwdziała
tworzeniu się cienkiej i ciężkiej warstwy pyłu w płaszczyźnie dysku. Ostatnie
badania pokazują że odpowiednio masywne i metaliczne dyski są nie wrażliwe na
ten proces~\citep{L10}.

{\bf więcej o niestabilnościach powodujących turbulencje}

\par Zdecydowaną wadę powyższej hipotezy jest całkowite zaniedbanie globalnej
turbulencji występującej w dyskach okołogwiazdowych, będącej jedynym
mechanizmem zdolnym do wyjaśnienia obserwowanych temp akrecji materii na
formujące się gwiazdy w ramach tzw. teorii $\alpha$-dysków~\citep{SS73}.
Obecnie za dominujący proces odpowiedzialny za turbulencję uważa się niestabilność
magnetorotacyjna~\citep{BH98}. Dopuszcza ona obecność w dysku obszarów pozbawionych
turbulencji np. w miejscach o niewystarczającym stopniu jonizacji gazu, jednakże
istnieje szereg innych zjawisk które mieszają płyn~\citep{LP10}.  {\bf Tu by się
pewnie przydało opisać}

\par Pomimo tych niesprzyjających warunków istnieje proces który dominuje
ewolucję pyłu w momencie w którym stosunek koncetracji ziaren pyłu do gęstości
gazu zbliża się do jedności. Tym mechanizmem jest {\it niestabilność
strumieniowa} po raz pierwszy przedstawiona w pracy~\cite{YG05}. Okazuje się, że
połączenie komulowanie się pyłu w lokalnych maksimach w rozkładzie ciśnienia
gazu i wzjamne oddziaływanie pomiędzy tymi dwoma składnikami dysku, prowadzi do
znacznego wzrostu koncentracji ziaren pyłu~\citep{J11}. Nawet zaniedbując efekt
samograwitacji w trakcie ewolucji niestabilności strumieniowej lokalna gęstość
pyłu może zwiększyć się tysiąckrotnie~\cite{JY07}, co może prowadzić do
wytworzenia się grawitacyjnie związanych obiektów~\cite{J07}. Ostatnie badania
niestabilności strumieniowej skupiały się na różnych aspektach fizycznych które
mają wpływ na jej rozwój t.j.: uwzględnienie szerokiego spektrum rozmiaru
cząstek pyłu~\cite{BS10a}, wpływ globalnego gradientu ciśnienia w dysku
okołogwiazdowym~\cite{BS10b}, stratyfikacja dysku~\cite{T12}. Niemniej jednak
wszystkie publikacje naukowe były ograniczone do lokalnego przybliżenia dysku.


%%%%%%%%%%%%%%%%%%%%%%%%%%%%%%%%%%%%%%%%%%%%%%%%%%%%%%%%%%%%%%%%%%%%%%%%%%%%%%%%
% vim: tw=80 ts=3: 

%\begin{savequote}[75mm]
%   $\ldots$\qauthor{$\ldots$}
%\end{savequote}

\chapter{Liniowa analiza stabilności}
\label{sec:lsa}
W~tym rozdziale przedstawiono poszczególne kroki liniowej analizy stabilności
dla niestabilności strumieniowej w oparciu o pracę~\citep{YG05}.
Analiza stabilności opiera się na trzech elementach:
\begin{enumerate}
   \item znalezieniu stanu równowagi dla zadanego układu równań,
   \item zadaniu zaburzeń o małej amplitudzie,
   \item znalezieniu modów własnych problemu liniowego, ich częstości oraz
      liniowego tempa wzrostu. 
\end{enumerate}
Niestety, dla układu równań hydrodynamiki dwóch sprzężonych płynów w~rozciągłym
dysku keplerowskim nie istnieje warunek równowagowy. Spowodowane jest to
migracją pyłu na centrum grawitacji na skutek utraty momentu pędu przez tarcie
aerodynamiczne. Naturalną konsekwencją migracji są znaczne zmiany w~radialnym
profilu rozkładu gęstości pyłu. Należy jednak zauważyć, że charakterystyczna
skala czasowa migracji jest rzędu setek bądź więcej lat, przy założeniu, iż mamy
do czynienia z~aglomeratami pyłu o rozmiarach metrowych. Z~tego względu możemy
założyć, że zmiany w~profilu gęstości są procesem powolnym w~stosunku do typowego
tempa wzrostu niestabilności strumieniowej, co zostanie wykazane w~dalszej
części pracy.

Kolejnym utrudnieniem jest radialna rozciągłość rozważanego dysku
okołogwiazdowego, która implikuje zależność stanu niezaburzonego od
promienia dysku. W~tym wypadku najogólniejszym podejściem byłaby
globalna analiza stabilności poprzez rozwiązanie dwupunktowego problem
brzegowego (np.~\cite{PHM04, KH06}), jednakże takie podejście jest dużo bardziej
skomplikowane. Przedstawianą tu lokalną analizę należy zatem traktować jako
pierwsze przybliżenie pełnej liniowej analizy niestabilności strumieniowej.
Niestabilne mody, uzyskane w~ramach liniowej analizy, posłużą jako punkt
odniesienia dla modów uzyskiwanych w~globalnym eksperymencie numerycznym.

Najbardziej wygodnym układem do lokalnego opisu niestabilności strumieniowej
jest tzw. kostka ścinana~\footnote{ang. shearing box}~\citep{HGB95}, czyli
kartezjański układ współrzędnych, którego początek współporusza się z~płynem na
wybranej orbicie $R_0$ z~częstością keplerowską $\Omega_0 \equiv
\Omega\left(R_0\right)$. Zwyczajowo przyjmuje się, że oś $x$ jest skierowana
radialnie na zewnątrz, oś $y$ jest w~kierunku azymutalnym, zaś $z$ jest osią
wertykalną. Zgodnie z~pracą~\cite*{YJ07}, układ równań ciągłości oraz ruchu dla
obu składników można wyrazić poprzez:
%
\begin{align}
\partial_t \rho_g &+ \mathbf{u}\cdot\nabla\rho_g - \frac{3}{2}\Omega x\partial_y\rho_g 
 = -\rho_g\nabla\cdot\mathbf{u},\label{eqc1}\\
\partial_t \rho_d &+ \mathbf{w}\cdot\nabla\rho_d - \frac{3}{2}\Omega x\partial_y\rho_d 
 = -\rho_d\nabla\cdot\mathbf{w},\label{eqc2}\\
\partial_t \mathbf{u} &+ \left(\mathbf{u}\cdot\nabla\right)\mathbf{u} 
 - \frac{3}{2}\Omega x\partial_y\mathbf{u} 
 = 2\Omega u_y \hat{\mathbf{x}} -\frac{1}{2}\Omega u_x \hat{\mathbf{y}} \notag\\
 &- \frac{\epsilon}{\tau_f}(\mathbf{u}-\mathbf{w}) -c_s^2\nabla\ln\rho_g 
 +2\eta\Omega^2 R \hat{\mathbf{x}},\label{eqm1}\\
\partial_t \mathbf{w} &+ \left(\mathbf{w}\cdot\nabla\right)\mathbf{w} 
 - \frac{3}{2}\Omega x\partial_y\mathbf{w}
 = 2\Omega w_y \hat{\mathbf{x}} -\frac{1}{2}\Omega w_x \hat{\mathbf{y}} \notag\\
 &- \frac{1}{\tau_f}(\mathbf{w}-\mathbf{u}), \label{eqm2}
\end{align}
%
gdzie wyrazy, takie jak $(3/2)\Omega x$ po lewej stronie równań, pojawiają się na
skutek transformacji wszystkich prędkości względem liniowego, ścinanego przepływu
$\mathbf{v}_0 = -(3/2)\Omega x \hat{\mathbf{y}}$ w~rotującym układzie
współrzędnych. Warto nadmienić, że wyraz $-(1/2)\Omega \{u,w\}_x
\hat{\mathbf{y}}$ po prawej stronie równań ruchu \mref{eqm1}-\mref{eqm2}, jest
sumą dwóch składników: $(-2\Omega \{u,w\}_x + (3/2)\Omega \{u,w\}_x)
\hat{\mathbf{y}}$ z~których pierwszy jest składową siły Coriolisa, a drugi
wynika z~odjęcia wspomnianego wcześniej przepływu średniego. Główną różnicą
pomiędzy równaniami \mref{eqm1} oraz \mref{eqm2} jest brak wyrazu ciśnieniowego
dla składnika pyłowego. YG05 zauważyli, że można w~spójny sposób uwzględnić
globalny, radialny gradient ciśnienia gazu w~ramach kostki ścinanej poprzez
dodanie liniowego wyrazu, który jest sparametryzowany wielkością określającą
bezwymiarową miarę rotacji podkeplerowskiej:
%
\begin{equation}
\eta \equiv - \frac{\partial_R P}{2\rho_g\Omega^2 R} \sim \frac{c_s^2}{v_K^2}.
\end{equation}
%
Układ równań \mref{eqc1}-\mref{eqm2} posiada znane rozwiązanie
równowagowe~\citep{N86}:
%
\begin{align}
\bar{\mathbf{w}} &= \left[ 
 -2\tau_s\xi, \frac{\tau_s^2\xi - 1}{1+\epsilon},
 0
\right]\eta v_K, \label{eq:w0}\\
\bar{\mathbf{u}} &= \left[ 
 2\epsilon\tau_s\xi, -\frac{1 + \epsilon\tau_s^2\xi}{1+\epsilon},
 0
\right]\eta v_K, \label{eq:u0}
\end{align}
%
gdzie $\tau_s = \Omega \tau_f$ to bezwymiarowy \emph{czas
zatrzymania}\footnote{ang. \emph{dimensionless stopping time}} i~$\xi =
((1+\epsilon)^2 + \tau_s^2)^{-1}$.  Linearyzacja równań \mref{eqc1}-\mref{eqm2}
polega na rozbiciu zmiennych na część stałą $\bar{\mathbf{q}}$ oraz zaburzenie
$\mathbf{q}^\prime$. W~rezultacie $\mathbf{q} =
\bar{\mathbf{q}} + \mathbf{q}^\prime$, gdzie $\mathbf{q}=[\rho_d, w_x, w_y, w_z,
\rho_g, u_x, u_y, u_z]$. Zakładamy równocześnie, że zaburzenie przyjmuje postać
osiowo-symetrycznej fali płaskiej:
%
\begin{equation}
   \label{eq:planar}
   \mathbf{q}^\prime(x,z,t) = \tilde{\mathbf{q}}
 \exp\left[i(k_x x + k_z z~-\omega t)\right].
\end{equation}
%
Po~podstawieniu liniowego zaburzenia, układ równań przyjmuje następującą postać:
%
\begin{align}
-i(\omega- k_x\bar{w}_x)\tilde{\rho}_d &= 
 - i~\bar{\rho}_d(k_x\tilde{w}_x + k_z\tilde{w}_z), \label{lin1}\\
-i(\omega- k_x\bar{u}_x)\tilde{\rho}_g &= 
 - i~\bar{\rho}_g(k_x\tilde{u}_x + k_z\tilde{u}_z), \label{lin2}\\
-i(\omega- k_x\bar{u}_x)\tilde{\mathbf{u}} &= 
 2\Omega \tilde{u}_y\hat{\mathbf{x}} - \frac{1}{2}\Omega \tilde{u}_x
 \hat{\mathbf{y}}
 -\frac{\epsilon}{\tau_f}(\tilde{\mathbf{u}} - \tilde{\mathbf{w}}) \notag\\
  -\frac{\tilde{\rho}_d}{\bar{\rho}_g\tau_f}
  (\bar{\mathbf{u}} &- \bar{\mathbf{w}})
  - \frac{c_s^2}{\bar{\rho}_g}ik_x\tilde{\rho}_g\hat{\mathbf{x}} -
  - \frac{c_s^2}{\bar{\rho}_g}ik_z\tilde{\rho}_g\hat{\mathbf{z}}, \label{lin3}\\
-i(\omega- k_x\bar{w}_x)\tilde{\mathbf{w}} &= 
 2\Omega \tilde{w}_y\hat{\mathbf{x}} - \frac{1}{2}\Omega \tilde{w}_x
 \hat{\mathbf{y}} 
 - \frac{1}{\tau_f} (\tilde{\mathbf{w}} - \tilde{\mathbf{u}}), \label{lin4}
\end{align}
%
gdzie $\epsilon = \bar{\rho}_d/\bar{\rho}_g$. Układ równań
\mref{lin1}-\mref{lin4} można zapisać jako:
%
\begin{equation}
 \eurom{A}(k_x,k_z,\omega)\tilde{\mathbf{q}} = 0,
 \label{eq:linset}
\end{equation}
natomiast: 
\begin{equation}
 A =
 \setlength\arraycolsep{2pt}
 \begin{bmatrix}
    -i\tilde{\omega}_d & i~k_x \bar{\rho}_d & 0 & i~k_z \bar{\rho}_d & 0 & 0 & 0 & 0 \\
    0 & \frac{1}{\tau_f} -i \tilde{\omega}_d & -2\Omega & 0 & 0 & \frac{1}{\tau_f} & 0 & 0 \\
    0 & \frac{1}{2}\Omega & \frac{1}{\tau_f} -i \tilde{\omega}_d & 0 & 0 & 0 & \frac{1}{\tau_f} & 0 \\
    0 & 0 & 0 & \frac{1}{\tau_f} -i \tilde{\omega}_d & 0 & 0 & 0 & \frac{1}{\tau_f} \\
    0 & 0 & 0 & 0 & -i\tilde{\omega}_g & i~k_x \bar{\rho}_g & 0 & i~k_z \bar{\rho}_g \\
    \frac{\bar{u}_x - \bar{w}_x}{\tau_f \bar{\rho}_g} & -\frac{\epsilon}{\tau_f} & 0 & 0 &
    \frac{c_s^2}{\bar{\rho}_g i~k_x} & \frac{\epsilon}{\tau_f}-i\tilde{\omega}_g &
    -2\Omega & 0\\
    \frac{\bar{u}_y - \bar{w}_y}{\tau_f \bar{\rho}_g} & 0 & -\frac{\epsilon}{\tau_f} & 0 & 0 &
    \frac{1}{2}\Omega & \frac{\epsilon}{\tau_f} -i \tilde{\omega}_g & 0 \\
    0 & 0 & 0 & -\frac{\epsilon}{\tau_f} & \frac{c_s^2}{\bar{\rho}_g i~k_z} & 0 & 0 &
    \frac{\epsilon}{\tau_f} -i \tilde{\omega}_g
 \end{bmatrix},
%
\end{equation}
zaś $\tilde{\omega}_d = \omega - k_x \bar{w}_x$ i~$\tilde{\omega}_g = \omega -
k_x \bar{u}_x$.
%
Nietrywialne rozwiązania układu liniowego \mref{eq:linset} istnieją wtedy i~tylko
wtedy, gdy równanie dyspersyjne:
\begin{equation}
 \det|\eurom{A}(k_x,k_z,\omega)|=0.
 \label{eq:disprel}
\end{equation}
%
Dla zadanych wartości $(k_x, k_x)$, relację \mref{eq:disprel} można rozwiązać ze
względu na $\omega$ i~dzięki temu otrzymać związki pomiędzy amplitudami
składowych wektora $\tilde{\mathbf{q}}$.
Tempo wzrostu niestabilności definiujemy jako urojoną część zespolonej
częstotliwości $s=\Im(\omega)$.
%
Analiza  liniowej fazy wzrostu niestabilności strumieniowej, wzbudzanej w
symulacjach, została przedstawiona w podrozdziale~\ref{sec:simulation_analysis},
gdzie porównano tempo wzrostu najszybciej rosnących modów w eksperymentach
numerycznych do tempa wzrostu obliczonego, jako rozwiązanie związku
dyspersyjnego~\mref{eq:disprel}. Podobnie, liczby falowe  najszybciej rosnących
modów niestabilności w eksperymentach numerycznych zostaną porównane z
przewidywaniami analitycznymi.

%%%%%%%%%%%%%%%%%%%%%%%%%%%%%%%%%%%%%%%%%%%%%%%%%%%%%%%%%%%%%%%%%%%%%%%%%%%%%%%%
% vim: tw=80 ts=3: 

\begin{savequote}[75mm]
Strike hot iron and call forth sparks; strike a man and call forth fury; to shape man or metal to thy will, thou must
strike with force.
\qauthor{Collected Sermons of Carras, Thief by Looking Glass Studios}
\end{savequote}

\chapter{Eksperyment numeryczny}
\newthought{Lorem ipsum dolor sit amet}, consectetuer adipiscing elit. Morbi commodo, ipsum sed pharetra gravida, orci
\section{Metodyka}
\label{sec:metodyka}
W~ramach poniższej pracy doktorskiej przeprowadzono szereg eksperymentów
numerycznych przy użyciu autorskiego kodu siatkowego PIERNIK, który opiera się
na:
\begin{itemize}
   \item zachowawczym schemacie numerycznym Relaxing TVD~\cite{jin-xin-95} w
      połączeniu z~dzielonym, kierunkowym całkowaniem przestrzennym i~czasowym
      przy użyciu algorytmu\linebreak Runge-Kutta drugiego
      rzędu~\cite{2003PASP..115..303T,2003ApJS..149..447P},
   \item implementacji wielopłynowości, tj. możliwości symulowania wielofazowego
      ośrodka np. płynu neutralnego i~płynu bezciśnieniowego (pył) z
      uwzględnieniem oddziaływań międzypłynowych~\cite{piernik1,piernik2},
   \item solwerze multigridowym pozwalającym efektywnie rozwiązywać paraboliczne
      i~eli\-pty\-czne równania różniczkowe, a w~szczególności równanie Poissona
      opisujące samograwitację pyłu~\citep{HG00},
   \item solwerze multipolowym pozwalającym na szybkie określenie warunków
      brzegowych dla potencjału grawitacyjnego~\citep{J77}.
   \item implementacji cylindrycznego układu współrzędnych w~postaci
      zachowującej moment pędu~\cite{M07,SO10},
   \item implementacji algorytmu szybkiego transportu
      eulerowskiego~\footnote{ang. \emph{Fast Advection in Rotating Gaseous Objects --
      FARGO}}~\citep{M00} w~ujęciu wielopłynowym.

\end{itemize}
Ponadto PIERNIK korzysta z~algorytmu typu \emph{Constraint
Transport}~\cite{EH88} zapewniającego bez źródłowość ewolucji pola
magnetycznego, a także możliwość wykonywania symulacji przy użyciu siatek
adaptywnych~\footnote{ang.  \emph{Adaptive Mesh Refinement -- AMR}}.  
%
\par Obliczenia opisywane w~tej pracy zostały przeprowadzone z~wykorzystaniem:

\begin{itemize}
   \item infrastruktury PL-Grid, w~szczególności klastrów
      komputerowych Galera+ (TASK), Hydra (ICM), Zeus (ACK CYFRONET), Inula
      (PCSS) w~ramach grantu \emph{plggpiernik},
   \item klastrów PRACE, w~szczególności klastrów komputerowych
      Cartesius (SurfSara), Fionn (ICHEC) w~ramach grantu \emph{PIERNIK-SI} w
      projekcie DECI-11
\end{itemize}
Sumaryczne zużycie wyniosło kilka milionów CPUh.


%We conduct numerical simulations with the aid of a
%parallel MHD code PIERNIK using the cylindrical coordinate system. 


\subsection{Podstawowe równania}
Globalna dynamika dysku okołogwiazdowego można opisać poprzez dwa, wzajemnie ze
sobą oddziałujące płyny: neutralny gaz podlegający izotermicznemu równaniu
stanu oraz pył jako bezciśnieniowy płyn. Równania hydrodynamiki przyjmują dla
takiego modelu następującą postać:

% CONSERVATIVE FORM
\begin{align}
   \partial_t \rho_g &+ \nabla\cdot\left(\rho_g\mathbf{u}\right) = 0,\label{eq1}\\
   \partial_t \rho_d &+ \nabla\cdot\left(\rho_d\mathbf{w}\right) = 0,\label{eq2}\\
\partial_t \left(\rho_g\mathbf{u}\right) &+
   \nabla\cdot(\mathbf{u}\otimes(\rho_g\mathbf{u})+P) \notag\\
 &= -\rho_g\left(\nabla\Phi +
\frac{\rho_d}{\tau_f\rho_g}(\mathbf{u}-\mathbf{w})\right),\label{eq3}\\
\partial_t \left(\rho_d\mathbf{w}\right) &+
\nabla\cdot(\mathbf{w}\otimes(\rho_d\mathbf{w})) \notag\\
 &= -\rho_d\left(\nabla\Phi + \frac{1}{\tau_f}(\mathbf{w}-\mathbf{u})\right)
\label{eq4}.
\end{align}
% NON-CONSERVATIVE FORM
%\begin{align}
%\partial_t \rho_g &+ \nabla\cdot\left(\rho_g\mathbf{u}\right) = 0,\\
%\partial_t \rho_d &+ \nabla\cdot\left(\rho_d\mathbf{w}\right) = 0,\\
%\partial_t \mathbf{u} &+ \left(\mathbf{u}\cdot\nabla\right)\mathbf{u} = 
% -\nabla\Phi + \frac{\rho_d}{\tau_f\rho_g}(\mathbf{w}-\mathbf{u})
% -c_s^2\nabla\ln\rho_g,\label{eq3} \\
%\partial_t \mathbf{w} &+ \left(\mathbf{w}\cdot\nabla\right)\mathbf{w} = 
% -\nabla\Phi - \frac{1}{\tau_f}(\mathbf{w}-\mathbf{u}),\label{eq4}
%\end{align}

\noindent gdzie $\rho_g$, $\rho_d$ to odpowiednio gęstości gazu i~pyłu,
$\mathbf{u}$, $\mathbf{w}$ ich prędkości, $P$ to ciśnienie gazu, $\tau_f$ jest
skalą czasową tarcia~\footnote{ang. friction time}~\mref{eq:tauf}, a $\Phi$ to
potencjał grawitacyjny.

\par Ze względu na specyfikę zagadnienia badanego w~niniejszej pracy
najwygodniejszym układem współrzędnych jest układ cylindryczny. W~kodzie
PIERNIK geometria cylindryczna jest zaimplementowana w~formie zachowującej
moment pędu~\cite{M07,SO10}, która wprowadza tylko jeden, dodatkowy wyraz
źródłowy do równań ruchu~\mref{eq3} - \mref{eq4}: odpowiednio
$\left((\rho_g u_\phi + P) / R\right)\mathbf{\hat{R}}$ oraz $(\rho_d w_\phi / R)
\mathbf{\hat{R}}$.
\par Sprzężenie pomiędzy gazem a pyłem, które przyjmuje postać
$\rho_d/\tau_f\rho_g(\mathbf{u}-\mathbf{w})$ i~$1/\tau_f(\mathbf{w}-\mathbf{u})$
odpowiednio dla równań \mref{eq3} i~\mref{eq4} w~zależności od tego jak zostanie
potraktowane, może prowadzić do znacznego skrócenia kroku czasowego. Z tego
względu zdecydowano o implementacji pół niejawnego schematu modyfikującego
bezpośrednio prędkości gazu i~pyłu, zaproponowanego przez~\cite{TB09}.

%W~ramach pracy skupiono się na dyskach rozciągających się relatywnie dużych
%promienii tj. 2~AU. Zakładając, za pracą~\cite{CD93}, że przejście do
%reżimu Stokesa zachodzi dla ziaren pyłu o promieniu większym niż
%$a = 9/4\lambda_g$ gdzie $\lambda_g = 4.2\times 10^4\textrm{
%cm} (10^{-14}\textrm{ g cm}^{-3}/\rho_g) \approx (R/1 \textrm{AU})^{2.75}$~cm 
%jest średnią drogą swobodną molekuł gazu~\citep{W77,BT09}, zaś $R$ jest
%odległością radialną od centrum dysku. Przy tych założeniach, reżim Epstein ma
%%zastosowanie dla dominującej części domeny obliczeniowej nawet dla największych
%symulowanych przez nas ziaren pyłu. Skala czasowa tarcia przyjmuję zatem
%następującą postać:
%

\subsection{PIERNIK}
Do rozwiązania układu równań \mref{eq1}--\mref{eq4} zastosowano kod numeryczny
\textbf{PIERNIK} stworzony i~rozwijany w~Centrum Astronomii UMK.  Implementuje
on tzw. \textit{metodę relaksacji TVD}~\cite{jin-xin-95}, która zapewnia wysoką
dokładność rozwiązań równań hydrodynamicznych oraz numeryczną stabilność, przy
stosunkowo niskich nakładach mocy obliczeniowych. Poniżej przedstawiono krótki
zarys użytej \textit{metody relaksacji TVD} na przykładzie równania

\begin{equation}\label{diffeuler}
   \partial_t \mathbf{u} + \partial_{x} \mathbf{F}(\mathbf{u}) = 0.
\end{equation}
wzorując się na pracach Trac \& Pen~\cite*{2003PASP..115..303T}
oraz Pen, et al.~\cite*{2003ApJS..149..447P}, które zostały użyte jako punkt
wyjścia przy tworzeniu \textbf{PIERNIKa}. 

\paragraph{Podział ,,zaburzenia'' na fale biegnące w~prawą i~lewą stronę.} ~\\

Układ \mref{diffeuler} można zastąpić układem relaksacyjnym postaci:

\begin{gather}
   \partial_t \mathbf{u} + \partial_x (c\mathbf{w}) = 0, \label{rel1}\\
   \partial_t \mathbf{w} + \partial_x (c\mathbf{u}) = 0, \label{rel2}
\end{gather}

gdzie $c(x,t)$ jest dowolną, dodatnią funkcja nazywaną \emph{prędkością
mrożącą}, a $\mathbf{w} = \mathbf{F}(\mathbf{u})/c$. Układ relaksacyjny zawiera
dwa sprzężone, liniowe równania adwekcji wielkości zachowawczych $\mathbf{u}$
oraz $\mathbf{w}$. Aby rozwiązać równania \mref{rel1} i~\mref{rel2} należy
dokonać zamiany zmiennych:

\begin{gather}
   \mathbf{u}^R = \frac{\mathbf{u} + \mathbf{w}}{2}, \\
   \mathbf{u}^L = \frac{\mathbf{u} - \mathbf{w}}{2}, \\
   \intertext{co pozwala zapisać}
   \partial_t \mathbf{u}^R + \partial_x (c\mathbf{u}^R) = 0,\label{rel3}\\
   \partial_t \mathbf{u}^L + \partial_x (c\mathbf{u}^L) = 0.\label{rel4}
\end{gather}

Równania \mref{rel3} i~\mref{rel4} można interpretować jako przepływ wielkości
zachowawczych w~pra\-wą i~lewą stronę z~prędkością $c$. Sumując można je wyrazić
jako

\begin{equation}
   \partial_t \mathbf{u} + \partial_x \mathbf{F}^R - \partial_x \mathbf{F}^L = 0,\\
\end{equation}

gdzie $\mathbf{F}^R=c \mathbf{u}^R$ i~$\mathbf{F}^L=c \mathbf{u}^L$. 

\paragraph{Obliczenie strumieni wielkości zachowawczych metodą \emph{,,pod wiatr''}.}~\\

W~celu rozwiązania \mref{rel3} i~\mref{rel4} należy policzyć strumienie
wielkości zachowawczych. W~podstawowej wersji \textbf{Piernika--MHD} strumienie
na brzegach komórek są liczone z~dokładnością do drugiego rzędu w~przestrzeni
$\left(\epsilon \sim O\left[(\Delta x)^2\right]\right).$ Duża dokładność jest
uzyskiwana poprzez obliczenie strumieni pierwszego rzędu
$\mathbf{F}^{(1),t}_{n+1/2}$ \emph{metodą ,,pod wiatr''}\footnote{ang.
,,upwind'' method}~\cite{cir}, która zapewnia stabilność rozwiązania w~tym
sensie, że błędy numeryczne nie narastają nieograniczenie w~czasie.  Korzystając
z definicji strumienia $\mathbf{F}_n^{(1),t}=c \mathbf{u}_n^t$, jako wielkości
określonej na środku komórki, w~zależności od kierunku propagacji fali strumień
na brzegu jest obliczany w~następujący sposób:

\begin{equation}
   \mathbf{F}^{(1),t}_{n+1/2} = 
   \begin{cases}
      \mathbf{F}^{(1),t}_{n}  \quad \textrm{jeżeli }c>0,\\
      \mathbf{F}^{(1),t}_{n+1}\quad \textrm{jeżeli }c<0.
   \end{cases}
\end{equation}

Następnie liczona jest poprawka strumieni drugiego rzędu, ponownie z
uwzględnieniem kierunku propagacji fali:

\begin{align} \label{lab1}
   \begin{cases} 
      \Delta \mathbf{F}^{L,t}_{n+1/2} = \frac{\mathbf{F}^t_{n} - \mathbf{F}^t_{n-1}}{2} \\
      \Delta \mathbf{F}^{R,t}_{n+1/2} = \frac{\mathbf{F}^t_{n+1} - \mathbf{F}^t_{n}}{2}
   \end{cases} \textrm{ dla prędkości }c>0,\\
   \label{lab2}\begin{cases} 
   \Delta \mathbf{F}^{L,t}_{n+1/2} = -\frac{\mathbf{F}^t_{n+1} - \mathbf{F}^t_{n}}{2} \\
   \Delta \mathbf{F}^{R,t}_{n+1/2} = -\frac{\mathbf{F}^t_{n+2} - \mathbf{F}^t_{n+1}}{2}
   \end{cases} \textrm{ dla prędkości }c<0.
\end{align}

\paragraph{Użycie ,,ogranicznika strumienia''.}~\\
Do wyznaczenia ostatecznej poprawki drugiego rzędu jest użyta specjalna funkcja,
tzw. \emph{ogranicznik strumienia}. Ograniczniki strumienia są używane, aby
uniknąć niefizycznych oscylacji, które pojawiłyby się po dodaniu poprawek
drugiego rzędu w~obszarach zawierających nieciągłości.  Ostateczna poprawka jest
wyznaczana z~wyrażenia

\begin{equation}
   \Delta \mathbf{F}^{t}_{n+1/2} = \phi\left(\Delta
   \mathbf{F}^{L,t}_{n+1/2},\Delta \mathbf{F}^{R,t}_{n+1/2} \right),
\end{equation}

gdzie przykładowy ogranicznik strumienia~\cite{leer} to

\begin{equation}
   \phi(a,b) = 
   \begin{cases}
      \frac{2ab}{a+b}, & \textrm{ gdy }ab>0 \\
      0, & \textrm{ gdy }ab<0
   \end{cases}.
\end{equation}

Użycie funkcji $\phi$ jest jednym z~warunków metody \emph{relaksacji TVD} i
przekłada się na stabilność schematu numerycznego.

\paragraph{Całkowanie w~czasie}~\\

Całkowanie w~czasie jest przeprowadzone przy użyciu standardowego schematu
Rungego -- Kutty. Najpierw wykonywany jest krok połówkowy

\begin{equation}
   \mathbf{u}^{t+\Delta t/2}_{n} = \mathbf{u}^{t}_{n} -
   \left(\frac{\mathbf{F}^t_{n+1/2} - \mathbf{F}^t_{n-1/2}}{\Delta x}
   \right)\frac{\Delta t}{2} + \mathbf{S}_n^t \frac{\Delta t}{2},
\end{equation}

gdzie

\begin{equation}
   \mathbf{F}^{t}_{n+1/2} = \mathbf{F}^{R,t}_{n+1/2} -
   \mathbf{F}^{L,t}_{n+1/2};\quad \mathbf{S}_n^t\textrm{ --- wyrazy źródłowe}.
\end{equation}

Następnie, przy użyciu wartości połówkowych $\mathbf{u}^{t+\Delta t/2}_{n}$,
obliczane są poprawki do strumieni i~wykonywany jest całkowity krok czasowy

\begin{equation}
   \mathbf{u}^{t+\Delta t}_{n} = \mathbf{u}^{t}_{n} -
   \left(\frac{\mathbf{F}^{t+\Delta t/2}_{n+1/2} - \mathbf{F}^{t+\Delta
   t/2}_{n-1/2}}{\Delta x} \right)\Delta t + S_n^{t+\Delta t/2}\Delta t.
\end{equation}

Dodatkowym warunkiem, który trzeba nałożyć na powyższy schemat dla zapewnienia
stabilności numerycznej, jest założenie, aby prędkość mrożąca $c$ była większa
niż dowolna prędkość charakterystyczna układu. Dla układu równań Eulera spełnia
to warunek

\begin{equation}\label{fs}
   c = |v| + c_s
\end{equation}

Równanie \mref{fs}, zgodnie z~warunkiem CFL zapewniającym stabilność schematu

\begin{equation}\label{cfl}
   \frac{c_{\textrm{max}}\Delta t}{\Delta x} \le 1,
\end{equation}

nakłada nam silne ograniczenie na krok czasowy. 
% Zgodnie z~\mref{cfl} im większa prędkość gazu, tym krótszy krok czasowy, co
% okaże się dużą niedogodnością przy zastosowaniu \emph{klasycznej metody kostki
% ścinanej} opisanej dalszej części pracy.

\subsection{FARGO}
Ze względu na obecność rotacji, dyski keplerowskie stanowią mało wdzięczny
obiekt badań numerycznych, szczególnie w~wypadku kiedy charakterystyczne
prędkości płynu osiągane w~interesujących procesach są drobnym ułamkiem
prędkości rotacji. Zgodnie z~warunkiem Couranta-Friedrichsa-Lewy'ego~\cite{cir},
stabilność schematu numerycznego jest zapewniona wtedy i~tylko wtedy, kiedy w
jednym kroku całkowania numerycznego sygnał nie propaguję się dalej niż o jedną
komórkę obliczeniową. W~związku z~tym, że prędkość rotacji dysku keplerowskiego
jest malejąca, a azymutalny rozmiar komórki na siatce cylindrycznej jest rosnącą
funkcją promienia, to najsilniejsze ograniczenie na rozmiar kroku czasowego
wprowadza dynamika gazu na najkrótszej symulowanej orbicie. Jedną z~technik
pozwalających uniknąć powyższych ograniczeń jest algorytm FARGO~\citep{M00}.
Oryginalnie został on zaprojektowany dla dwuwymiarowych dysków, lecz został
rozszerzony przez innych autorów~\cite{KBK09} do przypadków trójwymiarowych.
Poniższa praca rozwija go w~kierunku zastosowań do układu wielu płynów.

\par FARGO opiera się na kierunkowym podziale części adwekcyjnej równań
hydrodynamiki. W~kierunku radialnym i~wertykalnym stosuje się klasyczny solwer
(w przypadku PIERNIKa RTVD), natomiast w~kierunku azymutalnym rozbija się
adwekcję na trzy etapy:
\begin{enumerate}
   \item obliczenie średniej prędkości kątowej $\bar{\omega}_i$ dla każdego
      płynu i~każdego promienia o indeksie $i$
      \begin{equation}
         \bar{\omega}_i = \frac{1}{N_\varphi~N_z} ~ \sum_{j,k} \omega_{i,j,k},
      \end{equation}
      gdzie $N_\varphi,\,N_z$ to odpowiednio liczba komórek w~kierunku
      azymutalnym i~wertykalnym, zaś $\omega_{i,j,k}$ to prędkość kątowa
      poszczególnych komórek obliczeniowych.
   \item  obliczenie całkowitej liczby komórek dla przesunięcia w~kierunku
      azymutalnym
      \begin{equation}
         n_i = {\tt Nint} \left( \bar{\omega}_i \Delta t/\Delta \varphi \right),
      \end{equation}
      gdzie {\tt Nint} oznacza funkcję określającą \emph{najbliższą liczbę
      całkowitą}, $\Delta\varphi$ rozmiar komórki w~kierunku azymutalnym,
      $\Delta t$ bieżący krok czasowy. Przesunięcie o odległość $n_i \Delta
      \varphi$ w~czasie $\Delta t$ można wykorzystać do
      zdefiniowania tzw. ,,prędkości kątowej przesunięcia'' (ang. \emph{Shift
      velocity})
      \begin{equation}
         \omega_i^{\rm sh} = n_i \frac{\Delta \varphi}{\Delta t}.
      \end{equation}
   \item obliczenie ,,stałej, rezydualnej'' (ang. \emph{Constant Residual})
      prędkości kątowej  dla każdego promienia, będącej odchyleniem prędkości
      przesunięcią od wartości średniej
      \begin{equation}
         \omega_i^{\rm cr}= \bar{\omega}_i - \omega_i^{\rm sh}
      \end{equation}
   \item obliczenie właściwej prędkości ,,rezydualnej'' (ang. \emph{residual})
      dla każdej komórkim, będacej odchyleniem lokalnej prędkości kątowej w
      każdej komórce od średniej prędkości kątowej na promieniu na którym
      znajduje się dana komórka obliczeniowa 
      \begin{equation}
         \omega_{i,j,k}^{\rm res} = \omega_{i,j,k} - \bar{\omega}_i
      \end{equation}
\end{enumerate}

Powyższe cząstkowe prędkości kątowe $\omega_i^{\rm sh}, \omega_i^{\rm cr},
\omega_{i,j,k}^{\rm res}$ sumują się do wyjściowej prędkości kątowej
$\omega_{i,j,k}$ dla poszczególnych komórek:

\begin{equation}
   \omega_{i,j,k} = \omega_{i,j,k}^{\rm res} + \omega_i^{\rm cr} + \omega_i^{\rm
sh}.
\end{equation}

Transport zachowawczych wielości płynowych zgodnie z~rówaniami \mref{eq1} --
\mref{eq4} w~kierunku azymutalnym jest następnie wykonywana w~trzech krokach:

\begin{enumerate}
   \item przesunięcie wartości wielkości zachowawczych o $n_i$ komórek. Jak już
      było wcześniej wspomniane, ten krok jest równoznaczny z~transportem płynu
      z~prędkością $\omega_i^{\rm sh}$. Ze względu na fakt, iż ta operacja
      sprowadza się do przeindeksowania komórek obliczeniowych, nie nakłada ona
      żadnych ograniczeń na numeryczny krok czasowy.
   \item adwekcja~\footnote{adwekcja jest rozumiana jako rozwiązanie układu
         równań \mref{eq1} -- \mref{eq4}, tak jakby prawe strony tych równań
      były równe 0} wielkości zachowawczych z~prędkością $\omega_i^{\rm cr}$
      przy użyciu metody RTVD
   \item wykonanie pełnego całkowania równań \mref{eq1} -- \mref{eq4} przy
      założenie, że płyn poruszą się teraz z~prędkością kątową
      $\omega_{i,j,k}^{\rm res}$.
\end{enumerate}
Korzystając z~faktu, że $\omega_i^{\rm sh} \gg \max\left(\omega_i^{\rm cr},
\omega_{i,j,k}^{\rm res}\right)$, a tylko prędkości po prawej stronie
nierówności mają wpływ na warunek CFL, w~znaczący sposób zwiększamy krok
czasowy. Co prawda, aby zachować stabilność algorytmu należy zapewnić że
przesunięcie w~kierunku azymutalnym nie odseparuje dwóch sąsiednich (w kierunku
radialnym i~wertykalnym) komórek, co przekłada się na warunek

\begin{equation}\label{eq:tshear}
   \Delta t_{\rm shear} = 0.5 ~ \min_{i,j,k} \left( \frac{\Delta\varphi}
   {|\omega_{i,j,k} - \omega_{i-1,j,k}|} \right)
\end{equation}

Niemniej jednak pomimo ograniczenia~\mref{eq:tshear}, dla typowych eksperymentów
numerycznych przeprowadzonych w~tej pracy zastosowanie FARGO pozwoliło uzyskać
od 10 do 100 krotnego wydłużenia kroku czasowego.


\subsection{Potencjał grawitacyjny}
Potencjał grawitacyjny obecny w~równianach ruchu~\mref{eq3} - \mref{eq4} można
rozbić na dwa składniki
\begin{equation}
   \Phi = \Phi_{\textrm{ext}} + \Phi_{\textrm{self}},
\end{equation}
gdzie $\Phi_{\textrm{ext}}$ jest stałym w~czasie potencjałem pochodzącym od
gwiazdy macierzystej, zaś $\Phi_{\textrm{self}}$ potencjałem samograwitującego
płynu. Potencjał zewnętrzny $\Phi_{\textrm{ext}}$ został przyjęty jako potencjał
od masy punktowej
\begin{equation}
   \Phi_{\textrm{ext}} = -\frac{GM}{r} \mathbf{e}_r,
\end{equation}
gdzie $G$ to stała grawitacji, $M = 1\Msun$ masa obiektu centralnego, $r$
promień sferyczny, $\mathbf{e}_r$ radialny wersor kierunkowy.
Jako, że celem pracy jest wyizolowanie niestabilności strumieniowej z~pośród
szeregu innych procesów, które mogą w~dysku protoplanetarnym. Z tego względu
zaniedbano pionową składową przyspieszenia grawitacyjnego pochodzącą od
centralnego obiektu w~dysku, która prowadziła by do naturalnej sedymentacji pyłu
w płaszczyźnie dysku i~wzbudzenia się niestabilności
Kelvina-Helmholtza~\cite{JHK06}
\begin{equation}\label{eq:phiext}
   \Phi_{\textrm{ext}} = -\frac{GM}{R} \mathbf{e}_R.
\end{equation}

\par Potencjał $\Phi_{\textrm{self}}$ jest określony przez równanie Poissona
\begin{equation}\label{eq:poisson}
   \nabla^2 \Phi_{\textrm{self}} = 4\pi G \rho.
\end{equation}
Do rozwiązania równania \mref{eq:poisson} został użyty iteracyjny, solwer
multigridowy~\citep{HG00} połączony z~solwerem multipolowym~\citep{J77} w~celu
odpowiedniego obliczenia potencjału na nieperiodycznych brzegach domeny
obliczeniowej. Oba algorytmy zostały zaimplementowane w~PIERNIKu przez dra
Artura Gawryszczaka (CAMK W-wa).

\section{Warunki początkowe}
Domena obliczeniowa we wszystkich eksperymentach rozciąga się pomiędzy $2\AU$, a
$7\AU$ w~kierunku radialnym i~ma $0.375\AU$ wysokości. W~przypadkach
trójwymiarowych rozciągłość w~kącie azymutalnym wynosi $\pi / 6$.

\par Początkowy rozkład gęstości materii w~dysku określony jest poprzez formułę
wynikająca z~przepisu Minimalnej Masy Mgławicy Słonecznej~\footnote{ang.
\emph{Minimal Mass Solar Nebula}}~\cite{H81}
\begin{equation}\label{eq:mmsn}
   \Sigma(R) = 1700 \left(\frac{R}{1\textrm{ AU}}\right)^{-3/2} 
   \textrm{ g cm}^{-2}.
\end{equation}
W~przeciwieństwie do MMSN dla której profil temperatury jest wykładniczą funkcją
promienia, zakładamy że izotermiczny gaz posiada stałą temperaturę $T_0 = 170\K$
w całej objętości. Należy mieć na uwadze, że w~izotermicznym gazie najłatwiej
doprowadzić do formowania się zagęszczeń na skutek jego samograwitacji, w
porównaniu do przypadku adiabatycznego, który wymaga uważnego potraktowania
dodatkowych procesów fizycznych t.j. grzanie i~chłodzenie się gazu~\cite{Nel00}.
Ze względu na szereg komplikacji i~technicznych przeszkód na drodze do bardziej
realistycznego opisu gazu, zdecydowano się na zastosowania warunku
izotermicznego jak najprostszego z~możliwych opisów.
\par Przyjmujemy iż zewnętrzny potencjał~\mref{eq:phiext} jest określony przez
masę punktową $M=1\,\textrm{M}_\odot$ Pomimo zaniedbywania pionowej składowej
grawitacji, określamy charakterystyczną pionową skalę wysokości $H$, aby
oszacować gęstość przestrzenna na wybranym promieniu,
wykorzystując~\mref{eq:mmsn}, tak jakby gaz znajdował się w~pionowej równowadze
hydrostatycznej
%
\begin{equation}\label{eq:rhoR}
   \rho(R,z) =  \rho(R,0) \exp\left(-\frac{z^2}{2H(R)^2}\right),
\end{equation}
gdzie $\rho(R,0)$ jest gęstością gazu w~płaszczyźnie dysku, a $H^2 = 2 c_s^2 R^3/
GM$.
%
Powyższe równanie całkujemy ze względu na współrzędną \emph{z} korzystająć z
definicji gęstości powierzchniowej
\begin{equation} \label{eq:sigmaR}
   \Sigma(R) = \int_{-\infty}^\infty \rho(R,z) dz,
\end{equation}
%
otrzymujemy zależność
\begin{equation}
   \label{eq:rho}
    \rho(R,0) = \frac{\Sigma(R) }{\int_{-\infty}^\infty
   \exp\left(-\frac{z^2}{2H(R)^2}\right) dz}.
\end{equation}
Dla wybranej temperatury $T_0$ dysku, całka w~mianowniku po prawej stronie
równanie~\mref{eq:rho} przyjmuje wartości z~przedziału $[0.4,2]\AU$ dla $R \in
[2,7]\AU$. Dla ułatwienia obliczeń przyjmujemy, że wartość tej całki wynosi $1\AU$.
Warunek początkowy opiera się o radialną równowagę sił obliczoną niezależnie dla
składnika gazowego i~pyłowego. Gaz utrzymywany jest w~równowadze hydrostatycznej
pomiędzy grawitacją, siła odśrodkową i~gradientem ciśnienia, pył natomiast
porusza się z~prędkością keplerowską tak, aby równoważyć radialną składową
grawitacji.

\subsection{Warunki brzegowe}
Zarówno w~kierunku $z$ jak i~$\phi$ zastosowano periodyczne warunki brzegowe.
Aby zapobiec ucieczce masy z~domeny obliczeniowej w~kierunku radialnym użyto
warunków odbiciowych, tzn. w~warstwie brzegowej wielkości płynowe z~najbliższej
brzegu warstwy fizycznej domeny obliczeniowej są kopiowane ze zmianą znaku
składowej pędu prostopadłej do ściany domeny.

\par Aby zminimalizować niefizyczne odbicia fal od radialnych brzegów domeny,
wprowadzono obszary tłumiące na wewnętrznych i~zewnętrznych obszarach dysku o
szerokości $\sim0.5\AU$. W~obszarach tych wszystkie wielkości płynowe są
poddawane ewolucji z~dodatkowym wyrazem tłumiącym
\begin{equation}
  \frac{\textrm{d}X}{\textrm{d}t} = - \frac{X-X_0}{T_d}f(R).
\end{equation}
Ze względów stabilności numerycznej funkcja $f(R)$ ma skomplikowany przebieg
\begin{equation}\label{eq:overlap}
   \begin{split} 
      f(R) &= 1 - \tanh\left(\left(R - R_\textrm{in} + 1
      \right)^{f_\textrm{in}}\right)\\ &+ \max\left\{ \tanh\left(\left(R -
      R_\textrm{out} + 1\right)^{f_\textrm{out}}\right), 0\right\}, 
   \end{split}
\end{equation}
gdzie $X_0$ jest początkową wartością $X$, a $T_d$ jest skalą czasową tłumienia,
rzędu okresu orbitalnego na najniższej orbicie.
Wykładniki $f_\textrm{in}=f_\textrm{out}\equiv10$ określają szerokość przedziału
przejściowego pomiędzy obszarem ewoluującym bez tłumienia, a obszarem tłumionym
i zostały dobrane tak, aby tłumienie nie wpływało w~znaczącym stopniu na
stabilność całego układu. Przebieg funkcji $f(R)$ opisywanej rówananiem
\mref{eq:overlap} został przedstawiony na rysunku~\ref{fig:overlap}.
%
\begin{figure}
   \centering
   \includegraphics[width=0.5\textwidth]{figures/overlap}
   \caption{Przebieg funkcji $f(R)$ opisywanej równaniem~\mref{eq:overlap} dla
   zakresu promieni użytego we wszystkich symulacjach przedstawionych w~tej
pracy.}
   \label{fig:overlap}
\end{figure}
%
\subsection{Parametry symulacji}\label{ch2:simpar}
%
W~ramach pracy doktorskiej przeprowadzono szereg symulacji 2D modyfikując
początkowy $\epsilon$ oraz promień cząstek $a$, w~celu weryfikacji użytych metod
i wskazania optymalnych parametrów dla pełnych symulacji trójwymiarowych. Aby
móc porównać wyniki z~pracą innych autorów~\citep{JY07} parametr $\epsilon$
wybrano z~przedziału $[0.2, 2.0]$. Taki dobór parametru $\epsilon$ pozwala także
wzbudzić morfologicznie odmienny wyniki niestabilności strumieniowej, które
zostaną opisane w~dalszej części pracy. Promień cząstek pyłu został dobrany, tak
aby otrzymane wyniki można było porównać z~pracą JY, a także tak, aby były na
tyle małe aby ich obecność w~dysku protoplanetarnym można było wyjaśnić w~ramach
modelu zderzeniowego wzrostu przedstawionego w~rozdziale 1. Pełne zestawienie
użytych parametrów zostało przedstawione w~tabeli~\ref{tab1}.

\begin{table}
   \centering
   \begin{tabular}{cccccc}
      \hline
      Nazwa & $N_r \times N_\varphi \times N_z$ &
      $a$~[cm] & $\epsilon$ & $T_\textrm{end}$~[yr] \\
      \hline
      BD3d  &  $2560  \times 512 \times 192$  & 50  & 3.0 & 500  \\
      BD3dS &  $2560  \times 512 \times 192$  & 50  & 3.0 & 250  \\
      AA    &  $5120  \times 1   \times 300$  & 10  & 0.2 & 3000 \\
      AB    &  $5120  \times 1   \times 300$  & 10  & 1.0 & 3000 \\
      AC    &  $5120  \times 1   \times 300$  & 10  & 2.0 & 3000 \\
      BA    &  $5120  \times 1   \times 300$  & 50  & 0.2 & 3000 \\
      BB    &  $5120  \times 1   \times 300$  & 50  & 1.0 & 3000 \\
      BB3d  &  $2560  \times 512 \times 192$  & 50  & 1.0 & 500  \\
      BC    &  $5120  \times 1   \times 300$  & 50  & 2.0 & 3000 \\
      AAh   &  $10240 \times 1   \times 600$  & 10  & 0.2 & 1700 \\
      AAu   &  $20480 \times 1   \times 1200$ & 10  & 0.2 & 1800 \\
      ABh   &  $10240 \times 1   \times 600$  & 10  & 1.0 & 1400 \\
      BAh   &  $10240 \times 1   \times 600$  & 50  & 0.2 & 1730 \\
      BBh   &  $10240 \times 1   \times 600$  & 50  & 1.0 & 3000 \\
      \hline
   \end{tabular}
\caption{Parametry użyte w~symulacjach. Kolumny opisują w~kolejności: oznaczenie
   kodowe symulacji, ilość komórek obliczeniowych w~kierunkach $r$, $\varphi$ i
   $z$, promień cząstek, początkowy stosunek gęstości pyłu do gęstości gazu,
całkowity czas trwania symulacji w~latach.}
\label{tab1}
\end{table}

%%%%%%%%%%%%%%%%%%%%%%%%%%%%%%%%%%%%%%%%%%%%%%%%%%%%%%%%%%%%%%%%%%%%%%%%%%%%%%%%
% vim: tw=80 ts=3: 

\begin{savequote}[75mm]
Nulla facilisi. In vel sem. Morbi id urna in diam dignissim feugiat. Proin molestie tortor eu velit. Aliquam erat volutpat. Nullam ultrices, diam tempus vulputate egestas, eros pede varius leo.
\qauthor{Quoteauthor Lastname}
\end{savequote}

\chapter{Consectetuer adipiscing elit}

\newthought{Lorem ipsum dolor sit amet}, consectetuer adipiscing elit. Morbi commodo, ipsum sed pharetra gravida, orci
magna rhoncus neque, id pulvinar odio lorem non turpis. Nullam sit amet enim. Suspendisse id velit vitae ligula volutpat

\begin{savequote}[75mm]
   If I model a phenomenon accurately, does that mean I understand it? Or might
   it be simple coincidence, or an artifact of the technique? Of course, as an
   ardent simulationist, I myself put much faith in Engine-modeling.
\qauthor{Edward Mallory, The Difference Engine by William Gibson and Bruce
   Sterling}
\end{savequote}

\chapter{Dyskusja i plany na przyszłość}
W ramach tej pracy przeprowadzono serię dwu- i trójwymiarowych symulacji
wzajemnie oddziałujących płynów: gazu i pyłu, w dysku keplerowskim. Przy czym
zaniedbano pionową składową grawitacji od masy punktowej umieszczonej w centrum
układu współrzędnych. Dla ziaren pyłu o rozmiarach od 10 do 50$\cm$
zaobserwowano gwałtowny wzrost zaburzeń w gęstości i prędkości pyłu na
charakterystyczych długościach fali zgodnych z przewidywaniami lokalnej,
liniowej analizy stabilności takiego układu. Wcześniejsze badania innych
autorów~\cite{YG05, JY07, TB09, BS10a, BS10b} opierały się w pełni na lokalnym
przybliżeniu kostki ścinanej, co prowadziło do szeregu uproszczeń m.in.
traktowania globalnego, radialnego ciśnienia gazu jako stałego i niezmiennego
współczynnika~\cite{N86}, stosowania bezwymiarowego czasu zatrzymania przy
obliczaniu wzajemnego oddziaływania obu płynów~\cite{YG05}. Bardzo ważnym
osiągnięciem poniższej pracy jest rozszerzenie modelu stosowanego przez innych
autorów poprzez uwzględnienie pełnej dynamiki okołogwiazdowego dysku w kierunku
radialnym i wykazanie, że w takim przypadku można wzbudzić niestabilność
strumieniową. Ponadto do opisu oddziaływania wiążącego oba płyny użyto
właściwego dla ich charakterystyk prawa tarcia aerodynamicznego~\mref{eq:tauf}
zamiast przybliżonego rachunku opartego na stałym, bezwymiarowym czasie
zatrzymania. Przyjęty model, choć stanowi wyraźny postęp względem innych prac, jest
ciągle mało realistycznym przybliżeniem pełnego dysku protoplanetarnego. Do jego
największych wad można zaliczyć brak stratyfikacji oraz nieuwzględnienie wpływu
globalnej turbulencji wywołanej niestabilnością magneto-rotacyjną~\cite{DKJ14}. 
%
\par Pierwszy etap niniejszej pracy opierał się na wykonaniu testów
zbieżnościowych dla różnych parametrów fizycznych dysku protoplanetarnego co
pozwolilo na określenie minimalnej liczby komórek obliczeniowych $(n\approx32)$
dla kodu \textsc{PIERNIK}, umożliwiającej na dokładne odwzorowanie liniowej fazy
wzrostu.  Otrzymane, we wszystkich przeprowadzonych eksperymentach numerycznych,
liczby falowe najszybciej rosnących modów niestabilności, a także ich pozycja na
mapie stabilności ($s(k_x, k_z)$), są zgodne z przewidywaniami liniowej analizy
niestabilności strumieniowej i potwierdzają właściwy wybór użytych metod
numerycznych. Kolejnym krokiem było wykonanie symulacji trójwymiarowych bez
samograwitacji i porównanie ich wyniku z wcześniejszymi symulacjami
dwuwymiarowymi. Uzyskane wyniki pokazują iż trójwymiarowe symulacje odtwarzają w
pełni przebieg ewolucji niestabilności strumieniowej, zgodnie z przewidywaniami
liniowej analizy stabilności przeprowadzonej dla układu zredukowanego do dwóch
wymiarów. Finalnym etapem pracy było przeprowadzenie w pełni trójwymiarowej
symulacji dwuskładnikowego dysku protoplanetarnego z uwzględnieniem wpływu
samograwitacji. Należy podkreślić, że pomimo założenia dużo większej niż
kanoniczna wartości stosunku gęstości gazu do gęstości pyłu, symulowany dysk
jest początkowo stabilny zgodnie z kryterium Toomre'a~\mref{eq:toomre}.
Parameter $Q$ jest stały dla całej domeny obliczeniowej i wynosi około $40$
(rów.~\ref{eq:Qemp}). Wzbudzenie się niestabilności strumieniowej prowadzi do
uformowania lokalnych zagęszczeń pyłu, których gęstość jest nawet stukrotnie
większa niż gęstość początkowa pyłu. Ilość zgromadzonej lokalnie masy pyłu
przekracza granicę stabilności grawitacyjnej i w rezultacie w układzie formuje
się znaczna liczby związanych grawitacyjnie obiektów, które należy interpretować
jako zalążki planetezymali. Pomimo tego, że zgromadzona w tych obiektach materia
pyłowa reprezentuje $50\cm$ ziarna pyłu, to ich całkowita masa jest
wystarczająca aby po ,,sprasowaniu'' na skutek grawitacji utworzyć ciała o
rozmiarach rzędu setek kilometrów. Dzięki temu połączone działanie
niestabilności strumieniowej i niestabilności grawitacyjnej jest w stanie obejść
\emph{metrowej bariery wzrostu} o którym była mowa w
rozdziale~\ref{sec:paradigm}.
%
\par Wyciągając wnioski odnośnie skuteczności opisywanego modelu w przełamaniu
\emph{metrowej bariery wzrostu} należy pamiętać o pewnych uproszczeniach i
przyjętych założeniach. Głównym niedostatkiem modelu jest przeszacowany
początkowy stosunek gęstości pyłu do gęstości gazu. Dla przypomnienia jest on
ponad stukrotnie wyższy niż wartość kanoniczna~\cite{FS03}. Jak pokazują
przedstawione w rozdziale~\ref{sec:sim_2d} wyniki symulacji dwuwymiarowych,
niższy stosunek gęstości pyłu do gęstości gazu wydłuża tempo wzrostu
niestabilności strumieniowej i jest dużo trudniejszy do śledzenia przy użyciu
zastosowanych w tej pracy metod numerycznych opisanych w
rozdziale~\ref{sec:metodyka}. Jednakże całkowita masa pyłu zawarta w domenie
obliczeniowej wynosi około $35\Mearth$ i co do rzędu wielkości odpowiada masie
skalistych planet i jąder planet gazowych Układu Słonecznego. Biorąc pod uwagę,
że użyty w symulacjach radialny profil gęstości MMSN jest tylko \emph{dolnym
ograniczeniem} rzeczywistego rozkładu gęstości materii w dysku protoplanetarnym,
to przedstawiony w niniejszej pracy model ciągle pozostaje w dużej zgodności z
przewidywaniami odnośnie Układu Słonecznego~\cite{D07}.

\par Teoria akrecji na jądra silnie zależy od procesu odpowiedzialnego za
formowanie się planetezymali~\cite{HBP13}. Poczatkowa funkcja masy planetezymali
nie jest dobrze określona. Przyjmuje się, że ma postać funkcji potęgowej, bądź
złożenia dwóch funkcji potęgowych~\cite{R03}. Wyniki przedstawione w
rozdziale~\ref{sec:sim_3d} pozwalają określić rozkład masy zgromadzonej w
grawitacyjnie związanych obiektach, które można traktować jako
\emph{protoplanetezymale}, co stanowi niezwykle cenną przesłankę odnośnie
początkowego rozkładu masy samych planetezymali. Zakładając, że rozkład masy
jest dany funkcją:
%
\begin{equation}
   f(m) \propto m^{-a},
\end{equation}
%
w rozdziale~\ref{sec:sim_3d} dopasowano do otrzymanego rozkładu masy funkcję o
wykładniku $a = 1.25$. Wykładnik ten mieści się w przewidywaniach innych
autorów~\cite{R03}, którzy szacują jego wartość między $1$ a $3$. Ponadto prawy
skraj funkcji masy (Rysunek~\ref{fig:massfun} można opisać funkcją o wykładniku
$a$ w przedziale $2 < a < 3$ który otrzymywany jest w N-ciałowych symulacjach
formowania się planet~\cite{MFFK98}.  Należy jednak zauważyć, iż przedstawiona
na rysunku~\ref{fig:massfun} funkcja masy utworzonych w symulacji grawitacyjnie
związanych obiektów jest obarczona sporymi błędami. W szczególności lewy skraj
jest ograniczony poprzez skończoną rozdzielczość siatki obliczeniowej:
typowe rozmiary obiektow związanych grawitacyjnie wynoszą od kilkudziesięciu do
kilkuset komórek, co przekłada się na rozmiar liniowy rzędu kilku komórek
obliczeniowych. Jest to wielkość wysoce niewystarczająca do poprawnego
odwzorowania niestabilności grawitacyjnej w takim obiekcie.
Prawy skraj natomiast, zawiera mało reprezentatywną statystycznie liczbę obiektów.
Dodatkową niepewność wprowadza zależność od parametru $\alpha$ w kryterium
związania grawitacyjnego\mref{eq:ekin}. 
: dla $0.1$ tylko $8\%$.
Nie ma dobrego kryterium na zwiazanie obłoku, potrzeba duzo wiekszej rozdzielczosci
%
\par Typowe rozmiary otrzymanych w symulacji BD3dS grawitacyjnie związanych
zagęszczęń mieszczą się w przedziale $10^{11} \div 10^{12}\cm$, zaś ich srednie
gęstości wynoszą $10^{-9} \div 10^{-8}\g/\cm^3$ i są one spójne z
charakterystykami obiektów, które są mogą tworzyć zwarte planetezymale w toku
dalszej ewolucji~\cite{HS08}.
sink particles

Dopiero uwzględnienie ewolucji obłoku (cytacja), dluzsza skala czasowa
pozwalajaca uwzglednic dynamike wiekowa i efekty migracyjne (cytacja).

\par nie jest określone tempo akrecji masy przez planetezymale

Rozszerzenie moze isc w dwie strony: bardziej realistyczny model dysku t.j.
stratyfikacja~(Johansen ostatnie prace) i pole magnetyczne, rozwoj
algorytmniczny t.j. adaptywna siatka niezbedna do sledzenia zapadania sie
obloku, podskalowy model ewolucji pylu np.  metodami MC~(Aska)


%%%%%%%%%%%%%%%%%%%%%%%%%%%%%%%%%%%%%%%%%%%%%%%%%%%%%%%%%%%%%%%%%%%%%%%%%%%%%%%%
% vim: tw=80 ts=3: 


\singlespacing

% the back matter
\clearpage
\bibliography{references}
\addcontentsline{toc}{chapter}{References}
\bibliographystyle{plainnat}
\include{endmatter/colophon}

\end{document}
