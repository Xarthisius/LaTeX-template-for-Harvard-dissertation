\begin{savequote}[75mm]
   If I model a phenomenon accurately, does that mean I understand it? Or might it be simple coincidence, or an artifact
   of the technique? Of course, as an ardent simulationist, I myself put much faith in Engine-modeling.
\qauthor{Edward Mallory, The Difference Engine by William Gibson and Bruce Sterling}
\end{savequote}

\chapter{Wyniki}

\subsection{Liniowa analiza stabilności}
\label{sec:lsa}
Jednym z celów naukowych poniższej pracy jest walidacja wyników eksperymentu
numerycznego poprzez porównanie z tempem wzrostu najsilniejszych modów
otrzymanych z liniowej analizy stabilności. W tym rozdziale przedstawiono
poszczególne kroki liniowej analizy stabilności dla niestabilności strumieniowej
w oparciu o pracę~\citep{YG05}.

Analiza stabilności opiera się na trzech elementach:
\begin{enumerate}
   \item znalezieniu stanu równowagi dla zadanego układu równań,
   \item zaburzeniu układu znaną funkcją o małej amplitudzie 
   \item wyprowadzeniu modów własnych oraz ich tempa wzrostu.
\end{enumerate}
Niestety dla układu równań hydrodynamiki dwóch sprzężonych płynów w rozciągłym
dysku keplerowskim nie istnieje warunek równowagowy. Spowodowane jest to
migracja pyłu na centrum grawitacji na skutek utraty momemtu pędu przez tarcie
aerodynamiczne. Naturalną konsekwencją migracji są znaczne zmiany w radialnym
profilu rozkładu gęstości pyłu. Należy jednak zauważyć, że charakterystyczna
skala czasowa migracji jest rzędu setek bądź wiecej lat, przy założeniu iż mamy
do czynienia z aglomeratami pyłu o rozmiarach metrowych. Z tego względu możemy
założyć że zmiany w profilu gęstości są procesem powolnym w stosunku do typowego
tempa wzrostu niestabilności strumieniowej, co zostanie wykazane w dalszej
części wywodu.

Kolejną przeszkodą jest sama rozciągłość symulowanego dysku okołogwiazdowego,
która implikuje zmienność potencjalnego stanu równowagowego wraz z promieniem
dysku. W tym wypadku powinno przeprowadzić się globalną analizę stabilności po
przez rozwiązanie dwupunktowego problem brzegowego (np.~\cite{PHM04, KH06}),
jednakże takie podejście jest dużo bardziej skomplikowane. Przedstawianą tu
lokalną analizę należy zatem traktować jako pierwsze przybliżenie pełnej analizy
stabilności niestabiności strumieniowej. Niestabilne mody uzyskane w ramach
liniowej analizy posłużą nam jako punkt odniesienia dla modów uzyskiwanych w
globalnym eksperymencie numerycznym.

Najbardziej wygodnym układem do lokalnego opisu niestabilności strumieniowej
jest tzw. kostka ścinana~\footnote{ang. shearing box}~\citep{HGB95}, czyli
kartezjański układ współrzędnych, którego początek współporusza się z płynem na
wybranej orbicie $R_0$ z częstością keplerowską $\Omega_0 \equiv
\Omega\left(R_0\right)$. Zwyczajowo przyjmuje się, że oś $x$ jest skierowana
radialnie na zewnątrz, oś $y$ jest w kierunku azymutalnym, zaś $z$ jest osią
wertykalną. Zgodnie z pracą~\cite*{YJ07} układ równań ciągłości oraz ruchu dla
obu składników można wyrazić poprzez:
%
\begin{align}
\partial_t \rho_g &+ \mathbf{u}\cdot\nabla\rho_g - \frac{3}{2}\Omega x\partial_y\rho_g 
 = -\rho_g\nabla\cdot\mathbf{u},\label{eqc1}\\
\partial_t \rho_d &+ \mathbf{w}\cdot\nabla\rho_d - \frac{3}{2}\Omega x\partial_y\rho_d 
 = -\rho_d\nabla\cdot\mathbf{w},\label{eqc2}\\
\partial_t \mathbf{u} &+ \left(\mathbf{u}\cdot\nabla\right)\mathbf{u} 
 - \frac{3}{2}\Omega x\partial_y\mathbf{u} 
 = 2\Omega u_y \hat{\mathbf{x}} -\frac{1}{2}\Omega u_x \hat{\mathbf{y}} \notag\\
 &- \frac{\epsilon}{\tau_f}(\mathbf{u}-\mathbf{w}) -c_s^2\nabla\ln\rho_g 
 +2\eta\Omega^2 R \hat{\mathbf{x}},\label{eqm1}\\
\partial_t \mathbf{w} &+ \left(\mathbf{w}\cdot\nabla\right)\mathbf{w} 
 - \frac{3}{2}\Omega x\partial_y\mathbf{w}
 = 2\Omega w_y \hat{\mathbf{x}} -\frac{1}{2}\Omega w_x \hat{\mathbf{y}} \notag\\
 &- \frac{1}{\tau_f}(\mathbf{w}-\mathbf{u}), \label{eqm2}
\end{align}
%
gdzie wyrazy takie jak $(3/2)\Omega x$ po lewej stronie równań, pojawiają się na
skutek relatywizacji wszytkich predkości względem liniowego, ścinanego przepływu
$\mathbf{v}_0 = -(3/2)\Omega x \hat{\mathbf{y}}$ w rotującym układzie
współrzędnych. Warto nadmienić, że wyraz $-(1/2)\Omega \{u,w\}_x
\hat{\mathbf{y}}$ po prawej stronie równań ruchu \mref{eqm1}-\mref{eqm2} jest
sumą dwóch składników: $(-2\Omega \{u,w\}_x + (3/2)\Omega \{u,w\}_x)
\hat{\mathbf{y}}$ z których pierwszy jest składową siły Coriolisa, a drugi
wynika z odjęcia wspomnianego wcześniej przepływu średniego. Główną różnicą
pomiędzy równaniami \mref{eqm1} oraz \mref{eqm2} jest brak wyrazu ciśnieniowego
dla składnika pyłowego. Jak zauważyli YG05, można w spójny sposób uwzględnić
globalny, radialny gradient ciśnienia gazu ograniczając się do liniowego wyrazu,
który jest zparametryzowany tzw. bezwymiarową miarą rotacji podkeplerowskiej:

\begin{equation}
\eta \equiv - \frac{\partial_R P}{2\rho_g\Omega^2 R} \sim \frac{c_s^2}{v_K^2}.
\end{equation}

Układ równań \mref{eqc1}-\mref{eqm2} posiada znane rozwiązanie równowagowe~\citep{N86}

\begin{align}
\bar{\mathbf{w}} &= \left[ 
 -2\tau_s\xi, \frac{\tau_s^2\xi - 1}{1+\epsilon},
 0
\right]\eta v_K, \label{eq:w0}\\
\bar{\mathbf{u}} &= \left[ 
 2\epsilon\tau_s\xi, -\frac{1 + \epsilon\tau_s^2\xi}{1+\epsilon},
 0
\right]\eta v_K, \label{eq:u0}
\end{align}
%
gdzie $\tau_s = \Omega \tau_f$ to dimensionless stopping time i 
$\xi = ((1+\epsilon)^2 + \tau_s^2)^{-1}$. 
Linearyzacja równań \mref{eqc1}-\mref{eqm2}, polega na rozbiciu zmiennych na
część stałą oraz zaburzenie $\mathbf{q} = \bar{\mathbf{q}} + \mathbf{q}^\prime$,
gdzie $\mathbf{q}=[\rho_d, w_x, w_y, w_z, \rho_g, u_x, u_y, u_z]$. Zakładamy
równocześnie, że zaburzenie jest osiowosymetryczne (niezależne od współrzędnej $y$)
i przyjmuje postać fali płaskiej:

\begin{equation}
   \label{eq:planar}
   \mathbf{q}^\prime(x,z,t) = \tilde{\mathbf{q}}
 \exp\left[i(k_x x + k_z z -\omega t)\right]
\end{equation}

Po podstawieniu liniowego zaburzenia układ równań przyjmuje następującą postać

\begin{align}
-i(\omega- k_x\bar{w}_x)\tilde{\rho}_d &= 
 - i \bar{\rho}_d(k_x\tilde{w}_x + k_z\tilde{w}_z), \label{lin1}\\
-i(\omega- k_x\bar{u}_x)\tilde{\rho}_g &= 
 - i \bar{\rho}_g(k_x\tilde{u}_x + k_z\tilde{u}_z), \label{lin2}\\
-i(\omega- k_x\bar{u}_x)\tilde{\mathbf{u}} &= 
 2\Omega \tilde{u}_y\hat{\mathbf{x}} - \frac{1}{2}\Omega \tilde{u}_x
 \hat{\mathbf{y}}
 -\frac{\epsilon}{\tau_f}(\tilde{\mathbf{u}} - \tilde{\mathbf{w}}) \notag\\
  -\frac{\tilde{\rho}_d}{\bar{\rho}_g\tau_f}
  (\bar{\mathbf{u}} &- \bar{\mathbf{w}})
  - \frac{c_s^2}{\bar{\rho}_g}ik_x\tilde{\rho}_g\hat{\mathbf{x}} -
  - \frac{c_s^2}{\bar{\rho}_g}ik_z\tilde{\rho}_g\hat{\mathbf{z}}, \label{lin3}\\
-i(\omega- k_x\bar{w}_x)\tilde{\mathbf{w}} &= 
 2\Omega \tilde{w}_y\hat{\mathbf{x}} - \frac{1}{2}\Omega \tilde{w}_x
 \hat{\mathbf{y}} 
 - \frac{1}{\tau_f} (\tilde{\mathbf{w}} - \tilde{\mathbf{u}}), \label{lin4}
\end{align}
%
gdzie $\epsilon = \bar{\rho}_d/\bar{\rho}_g$. Układ równań
\mref{lin1}-\mref{lin4} można zapisać jako

\begin{equation}
 \eurom{A}(k_x,k_z,\omega)\tilde{\mathbf{q}} = 0
 \label{eq:linset}
\end{equation}
%
Nietrywialne rozwiązania układu linoweg \mref{eq:linset} istnieją wtedy i tylko
wtedy gdy równanie dyspersyjne 
\begin{equation}
 \det|\eurom{A}(k_x,k_z,\omega)|=0.
 \label{eq:disprel}
\end{equation}
%
Dla zadanych wartości $(k_x, k_x)$ relację \mref{eq:disprel} można rozwiąć ze
względu na $\omega$ i dzięki temu otrzymać związki pomiędzy amplitudami
składowych wektora $\tilde{\mathbf{q}}$.
Tempo wzrostu niestabilności definiujemy jako urojoną część zespolonej
częstotliwości $s=\Im(\omega)$.
%

Początkowe stadia ewolucji niestabilności strumieniowej są podobne dla
wszystkich wartości parametrów $\epsilon$ i $a$ t.j. dominują mody liniowe.
Dokładna analiza fazy liniowej została przedstawiona w
podrozdziale~\ref{simulation_analysis}.

%The following two sections describe
%different non-linear evolution of streaming instability in quasi-global setup
%with reference to similar case shown by JY07.

\section{Symulacje dwuwymiarowe}
Poniżej przedstawiono opis wyników uzyskanych dla symulacji w przypadku
dwuwymiarowym. Ich celem, w połączeniu z liniową analizą stabilności, była
weryfikacja zaimplementowanych metod oraz poprawności przyjętego modelu. Każdy z
przedstawionych wyników można porównać do analogicznych eksperymentów
numerycznych z pracy JY05~\cite{JY05}. Pomimo istotnych różnic w użytych
metodach (JY stosowali kod wysokiego rzędu -- PENCIL, a także traktowali pył
jako cząstki materialne, a nie płyn), wyniki we wszytkich przypadkach wykazują
zgodność ilościową i jakościową.

\subsection{Luźno związane ,,kamienie''}% ($a=50\,\textrm{cm}$, $\tau_s\approx
%1.2$)\label{marg_boulders}}

\begin{figure*}
   \centering
   \includegraphics[width=0.99\linewidth]{figures/fig1a}
   \includegraphics[width=0.99\linewidth]{figures/fig1b}
   \includegraphics[width=0.99\linewidth]{figures/fig1c}
   \includegraphics[width=0.99\linewidth]{figures/fig1d}
   \caption{Migawki gęstości pyłu dla symulacji z $50$~cm ziarnami pyłu
      dla czasu $100$, $200$, $450$ oraz $700$~lat odpowiednio dla lewego
      górnego, prawego górnego, dolnego lewego i dolnego prawego panelu.
      Każdy z paneli jest podzielony na trzy części różniące się początkowym 
      $\epsilon = 0.2, 1, 2.0$ odpowiednio dla górnego (BAh), środkowego (BB) i
      dolnego (BC) podpanelu.}
   \label{fig1}
\end{figure*}
%

Niestabilność strumieniowa wzbudza się najbardziej spektakularnie dla cząstek
pyłu o $\tau_s \sim 1$, kiedy sprzężenie z gazem nie jest już tak silne. Przy
wybranych parametrach symulacji opisanych w rozdziale~\ref{ch2:simpar},
bezwymiarowy parameter ,,stopping time'' jest bliski jedności dla cząstek pyłu o
rozmiarach $50\cm$ (BA, BB, BC). Początkowa liniowa faza ewolucji jest
zdominowana przez wzrost gęstości dla najbardziej niestabilnych modów i prowadzi
do znacznego wzrostu lokalnej gęstości pyłu i uformowania się
charakterystycznych, wydłużonych diagonalnie włókien. Po osiągnięciu nasycenia i
przejścia do nieliniowej ewolucji, włókna, na skutek dużej prędkości w kierunku
wertykalnym ulegają fragmentacji i wzajemnie się przenikają. Jednakże pył dalej
poruszą się po charaterystycznych ,,v''--kształtnych trajektoriach, które
wytworzyły się podczas liniowej fazy ewolucji. Migawki z czasowej ewolucji
gęstości pyłu zostały przedstawione na obrazku~\ref{fig1}. Wyraźny jest wpływ
początkowego stosunku gęstości pyłu do gęstości gazu $\epsilon$ na morfologię
formujących się obłoków: zarówno ich typowy rozmiar, jak i kąt nachylenia
względem płaszczyzny dysku. Dla $\epsilon = 1$ pył poruszą się i formuje
struktury nachylone pod kątem $45^o$, dla pozostałych przypadków $\epsilon=0.2,
2.0$ kąt nachylenia jest dużo większy, przez co formujące się struktury są
rozciągnięte w kierunku \emph{z}. We wszystkich wypadkach niestabilność
strumieniowa prowadzi do lokalnego wzrostu gęstości o prawie 2 rzędy wielkości.
Ze względu na swoją gęstość

\subsection{Mocno sprzężony ,,żwir''}
%($a=10\,\textrm{cm}$, $\tau_s\approx 0.24$)
%\label{tight_boulders}}
\begin{figure*}
   \centering
   \includegraphics[width=0.99\linewidth]{figures/fig2a}
   \includegraphics[width=0.99\linewidth]{figures/fig2b}
   \includegraphics[width=0.99\linewidth]{figures/fig2c}
   \includegraphics[width=0.99\linewidth]{figures/fig2d}
   \caption{Migawki gęstości pyłu dla symulacji z $10$~cm ziarnami pyłu
      dla czasu $100$, $200$, $800$ oraz $1200$~lat odpowiednio dla lewego
      górnego, prawego górnego, dolnego lewego i dolnego prawego panelu.
      Każdy z paneli jest podzielony na trzy części różniące się początkowym 
      $\epsilon = 0.2, 1, 2.0$ odpowiednio dla górnego (AA), środkowego (AB) i
      dolnego (AC) podpanelu.}
   \label{fig2}
\end{figure*}
Czasowy przebieg symulacji dla silnie sprzężonego pyłu (AA - AC) został
przedstawiony na Rysynku~\ref{fig2}. Ewolucja pyłu dla przypadku $\epsilon =
2.0$ i $10\cm$ ziaren pyłu (AC) nie odbiega znacząco od ewolucji analogicznego
przypadku dla $50\cm$ ziaren pyłu. Liniowa faza niestabilności podczas której
pył zostaje zagęszczony dla najbardziej niestabilnych modów, prowadzi do
wytworzenia się struktur wydłużonych w kierunku wertykalnym. Ze względu na
mniejsze tempo migracji dla sprzężonych cząstek pyłu, zagęszczenia są pochylone
w kierunku radialnym tylko w niewielkim stopniu. Maksymalna gęstość pyłu nie
przekracza dwudziestokrotności wartości początkowej.
\par Przypadek $\epsilon = 0.2$ charakteryzuje się niskim tempem wzrostu. W
początkowej liniowej fazie wzrost lokalnej gęstości pyły prowadzi do uformowania
się charakterystycznej diagonalnej siatki zagęszczeń. Lokalne maksima gęstości
sięgają nawet stukrotnej wielokrotności stanu początkowego. Jednakże, przejscie
do fazy nieliniowej prowadzi do gwałtownego rozmycia skupisk pyłu i osiągnięcia
quasi-stacjonarnego stanu, w którym zagęszczenia podlegają silnej turbulencji,
zaś ich maksymalna amplituda nie przekracza jednego rzędu wielkości względem
stanu początkowego (por. Rysunek~\ref{fig4}).

\par Najciekawszym przypadkiem, zasadniczo odbiegającym od pozostałych, jest
symulacja AB ($\epsilon=1.0,\, a=10\cm$). Początkowo ewolucja przebiega w
identyczny sposób jak dla pozostalych warunków, tzn. pojawia się regularny wzór
w gęstości pyłu odpowiadający najbardziej niestabilnym długościom fal. Jednakże 
po $800\yr$ w symulacji AB (mniej więcej dwa razy szybciej w symulacji ABh)
pojawiąją się gwałtownie rozszerzające się bąble wypełnione nie wielką ilością
pyłu, które silnie koncentrują pył na swoich brzegach (por. Rysunek~\ref{fig3}). 
Znacząco wzrasta prędkość z którą porusza się pył, po okresie kilkuset lat
silnie turbulentne ruchy propagują się na całą domenę obliczeniową. Ta
specyficzna ewolucja niestabilności strumieniowej jest spowodowana odmiennym
przebiegiem liniowego tempa wzrostu niestabilności (patrz Rysunek~\ref{fig2b)}).
Dla ustalonych parametrów fizycznych AB tempo wzrostu niestabilności $s$ wypada w
punkcie przegięcia funkcji $s(\epsilon)$. Nawet drobna lokalna odchyłka od
początkowego stanu, przy rozpiętości $s$ o rząd wielkości na bardzo krótkim
przedziale $\epsilon$ w znaczny sposób wpływa na ewolucję pyłu. Jest to
bezpośrenim powodem obserwowanego zjawiska, które YJ określili mianem
,,kawitacji''~\footnote{jako analog gwałtownego pojawiania się bardzo rzadkich
   bąbli pyło do nagłego przejścia z fazy ciekłej do gazowej}.
Turbulencja w pyle prowadzi do wytworzenia się lokalnych zagęszczęn sięgających
rząd wielkości ponad stan początkowy. Jest to zgodne z wynikami przedstawionymi
przez JY07 (patrz Rysunek.~8 w Ich pracy).

\begin{figure} 
\centering
\includegraphics[width=0.98\linewidth]{figures/fig3}
\caption{Migawki gęstości gazu dla symulacji ABh pokazujące dwa ,,zbliżenia''
   obszarów gdzie: występuje pseudo-kawitacja wynikająca z lokalnych zagęszczen
   wytworzonych w liniowej fazie wzrostu (lewy panel) oraz region całkowicie
   opanowany przez wybuchające bąble pustki i silną turbulencję pyłu (prawy
   panel). Animacja przedstawiająca ewolucję symulacji ABh jest dostępna pod
   adresem \href{http://youtu.be/NoA5-TiQabQ}{http://youtu.be/NoA5-TiQabQ}.}
\label{fig3}
\end{figure}

\begin{figure}
   \centering
   \includegraphics[width=0.5\linewidth]{figures/growthAB}
   \caption{Analityczne tempo wzrostu niestabilności strumieniowej jako funkcja
      stosunku gęstości pyłu do gęstości gazu. Pozostałe parametry niezbędne do
      rozwiązania równania~\mref{eq:disprel} zostały wzięte obszaru znajdującego
      się na lewym panelu Rysunku~\ref{fig2}. Przełamanie przebiegu funkcji dla
      $\epsilon\sim 1$ jest bezpośrednim powodem ,,kawitacji'' obserwowanej w
      symulacji AB}
   \label{fig2b}
\end{figure}


\begin{figure}
   \includegraphics[width=0.98\linewidth]{figures/fig4}
   \caption{
      Maksymalne stosunek gęstości pyłu do gęstości gazu dla trzech symulacji z
      identycznymi, fizycznymi warunkami początkowymi, lecz różną
      rozdzielczością. Ewolucja niestabilności strumieniowej przebiaga według
      scenariusza: (1) szybki wzrost podczas liniowej fazy ewolucji, który
      zwiększa lokalnie $\epsilon$ o ponad dwa rzędy wielkości, (2) po
      osiągnięciu krytycznego $\epsilon\approx 10$, zagęszczenia pyłu zostają
      gwałtownie rozmyte, (3) niestabilność osiąga wysycenie, nieliniowa
      ewolucja sprowadza się do oscylacji gęstości pyłu w postaci dużych,
      rozmytych obszarów o maksymalnej gęstości nieprzekraczającej
      dziesięciokrotności wartości początkowej. Zmiana rozdzielczości nie ma
      wpływu na powyższy scenariusz, jedynie wprowadza krótsze i szybciej
   rosnące długości fal skracając etap (1).}

   \label{fig4}
\end{figure}
%

\subsection{Porównanie wyników z liniową analizą stabilności}
Aby móc porównać obserwowane tempa wzrostu niestabilności strumieniowej z
analizą przedstawioną w rozdziale~\ref{sec:lsa} z domeny obliczeniowej
wyodrębniono małe, kwadratowe ,,łatki'' na wybranych orbitach. Rozmiar wybranych
obszarów $(0.15^2\AU)$ pozwala traktować je w ramach lokalnego przyliżenia
kostki ścinanej, a także umożliwia wyrażenie stałych parametrów występujących w
równaniach \mref{lin1}--\mref{lin4}) poprzez wartości średnie w łatce.
Można zatem przyjąc, że $\bar{\rho}_g = \left<\rho_g\right>$, $\bar{\rho}_d =
\left<\rho_d\right>$ to średnie przestrzenne odpowiednio gęstości gazu i
gęstości pyłu, oraz że ich średni wzajemny stosunek to $\bar{\epsilon} =
\left<\rho_d / \rho_g\right>$. Jako śrędnią częstość kątowa $\bar{\Omega}$
przyjęto częstość kątową środka łatki. Bezwymiarowa miara podkeplerowskiej
rotacji jest obliczona zgodnie ze wzorem (patrz YG05 rów.~(16) albo JY07
rów.~(1))
%In the case of 3D run the patches are chosen on the {\it r-z} plane at $\varphi
%= \varphi_\textrm{max} / 2$. 
%
\begin{equation}
   \bar{\eta} = -\frac{c_s^2\left<\partial_R \left<\rho_g\right>_z\right>_R}
      {2\bar{\rho}_g\bar{\Omega}^2 R},
   \label{eq:eta}
\end{equation}
%
W równaniu~\mref{eq:eta} średnia z gęstości gazu jest liczona najpierw w
kierunku wertykalnym, a następnie jest obliczana średnia z radialnej pochodnej 
$\left<\partial_R \rho_g\right>$. Wyrażenie na średni ,,stopping time'' zostało
wyprowadzone z równania~\mref{eq:tauf}
\begin{equation}
   \bar{\tau}_f = \rho_\bullet a / \left(\bar{\rho}_g \sqrt{c_s^2 +
   \left<\left|\mathbf{u} - \mathbf{v}\right|^2\right>} \right).
\end{equation}
%
Dla spójności średnie prędkości gazu i pyłu $\bar{\mathbf{u}},
\bar{\mathbf{w}}$ również są brane jako średnie wartości
$\left<\mathbf{u}\right>, \left<\mathbf{w}\right>$ zamiast obliczania ich przy
pomocy równań~\mref{eq:w0}-\mref{eq:u0}. Nie jest to duże odstępstwo od
równowagowych warunków, jako ze różnice względem analitycznych wartości nie
przekraczają $10\%$.
\par Aby okresłić liniowe tempo wzrostu dla wzbudzanych modów niestabilności
rozkład gęstości i prędkości poszczególnych płynów został przekształcony przy
pomocy transformaty Fouriera, tak aby uzyskać informację o amplitudach. Analiza
przebiegu czasowego zmienności amplitud dla poszczególnych częstości pozwala
wyizolować najbardziej niestabilne mody układy. Ten etap analizy dla jednej z
łatek został przedstawiony na Rysunku~\ref{fig7}.

\begin{figure}
  \includegraphics[width=0.98\linewidth]{figures/fig7}

  \caption{Czasowa ewolucja amplitud zaburzenia gęstości pyłu w przestrzeni
     fourierowskiej dla łatki z symulacji BB obejmującej obszar 
     $[2.85,3.15]\times[-0.15,0.15]~\AU^2$. Każda cienka, szara linia
     reprezentuje amplitudę dla wybranej pary liczb falowych $k_x, k_z$.
     $\Delta T = T_e - T_s$ to odcinek czasu dla którego do modów jest
     dopasowywane równanie~\mref{eq:fit}. $T_{\textrm{ref}}$ to punkt
     odniesienia dla ktorego identyfikowane są dominujące mody na podstawie
     wartości maksymalnej amplitud. Gruba, czarna linia pokazuje uśredniony
     przebieg zmienności dla modów, których amplituda dla $t = T_{\textrm{ref}}$
     jest większa niż $10^{-4}$.} 
   \label{fig7} 
\end{figure}

Dla każdego z badanych obszarów określono czas $T_s$ dla którego nie które mody
zaczynają wyłaniać się z szumu i rozpoczynają fazę liniowego wzrostu. Na
podobnej zasadzie wyznaczono czas $T_e$, dla którego następuje wysycenie
wzrostu. Następnie do wszystkich modów na przedziale $\Delta T = T_e - T_s$
zostaje dopasowana funkcja
%
\begin{equation}
   f(t) = A\exp\left(-s t\right).
   \label{eq:fit}
\end{equation}
%
Powyższa procedura pozawala na określenie tempa wzrostu w funkcji liczb falowych
$s(k_x, k_z)$ dla wszystkich modów podczas ich liniowego wzrostu. Mody rosnące
najszybciej są określane poprzez znalezienie modów o najwyższej amplitudzie w
wybranym momencie czasu $T_{\textrm{ref}} \lesssim T_e$, tuż przed ich
saturacją. Po określeniu ich tempa wzrostu $s(k_x, k_z)$ jest ono porównywane z
tempem wzrostu $s_0(k_x, k_z)$ wynikającym bezpośrednio z liniowej analizy dla
śrenich wielkości płynowych w łatce.

\par Rysunek~\ref{fig8} pokazuje czasową ewolucję amplitud zaburzenia gęstości
pyłu dla 3 najszybciej rosnących modów, wraz z dopasowaną funkcją~\mref{eq:fit}
i tempem wzrostu wynikającym z liniowej analizy dla symulacji BB3d, BB oraz BBh.
Wyraźnie widoczny jest wpływ rozdzielczości na numeryczne tempo wzrostu, a także
zbieżność wyników eksperymentu numerycznego z przewidywaniami teoretycznymi. W
najniższej rozdzielczości tempo wzrostu jest $10\div30\%$ mniejsze niż tempo
analityczne. Ostateczny poziom saturacji dla poszczególnych symulacji znajduje
się na różnym poziomie, ale należy zwrócić uwagę na fakt, że zmiana
rozdzielczości dostarcza coraz to krótszych fal, które mogą rosnąć szybciej i
zmieniać obraz całkowity niestabilności strumieniowej.
 
\begin{figure} 
   \includegraphics[width=0.98\linewidth]{figures/fig8}
   \caption{Czasowa ewolucja amplitud dominujących modów niestabilności
      strumieniowej mierzona dla zaburzenia gęstości (szara linia) wraz z
      dopasowaniem~\mref{eq:fit} (czarna linia) i przewidywanym tempem wzrostu
      wynikającym z liniowej analizy stabilności (linia przerywana).
      Rozwiązanie równania~\mref{eq:linset} jest określone na podstawie
      parametrów średnich kwadratowej łatki ulokowanej na $R=3\AU$ dla symulacji
      BB3d (górny panel), BB (środkowy panel) and BBh (dolny panel). 
   }
   \label{fig8}
\end{figure}

\par Aby rozszerzyć wzmocnić wnioski wypływające z porównania z liniową analizą
na Rysunku~\ref{fig9} przedstawiono wykres konturowy tempa wzrostu wynikający z
rozwiązania równania~\mref{eq:disprel} w zależności od liczby falowej. Wyraźnie
widać na nim, że najszybciej rosnące mody niestabilności strumieniowej układają
się w charakterystyczny grzbiet w kierunku rosnących $k_x$ i $k_z$. Wykres
tworzony jest dla stanu średniego w łatce dla czasu $T_{\textrm{ref}}$.
Następnie 9 dominujących modów wybranych wedle kryterium opisywanego wcześniej
jest zaznaczane przy użyciu punktów. Procedura jest powtarzana dla symulacji o
tych samych parametrach początkowych, ale różnych rozdzielczościach.
Powyższy schemat działania pozwala potwierdzić, że liczby falowe dominujących modów
wzbudzanych w eksperymencie numerycznym, układają wzdłuż wspomnianego wcześniej
grzebietu. Jedynym czynnikiem ograniczającym jest dostępna rozdzielczość domeny
obliczeniowej. Dla serii symulacji BB3d, BB, BB3d efektywna rozdzielczość siatki
wyniosła odpowiednio $150^2, 300^2, 600^2$ komórek obliczeniowych. Dominujące
mody symulacji o najniższej rozdzielczości grupują się poniżej konturu
''$(-1.0)$'', zaś dla najwyższej praktycznie wszystkie mają tempo wzrostu
powyżej $10^{-1}$. Podobne zachowanie jest widoczne dla pozostałych symulacji
dla których przerowadzono testy zbieżności, tj. AB, ABh (prawy panel na
rysunku~\ref{fig9}). Powodem obecności wyraźnego obcięcia dla krótki długości
fali w przeprowadzonych symulacjach jest wenętrzna, numeryczna dyfuzyjność metody
RTVD użytej w PIERNIKu. Z przeprowadzonych analiz wynika, że używane algorytmy
numeryczne potrzebują przynajmniej 32 komórek obliczeniowych na dlugość fali
niestabilnego modu, aby poprawnie oddać jego tempo wzrostu. Należy przy tym
zauważyć, że mody opisywane przez mniejszą ilość komórek nadalaj pozostają
niestabilne, aczkolwiek mogą wykazywać mniejsze tempo wzrostu niż to wynikające
z liniowej analizy stabilności.

\begin{figure*}
  \includegraphics[width=0.48\linewidth]{figures/fig9a}
  \includegraphics[width=0.48\linewidth]{figures/fig9b}
  \caption{Dziewięć najszybciej rosnących modów o liczbach falowych $(k_x, k_z)$
     wyodrębnionych z symulacji o tych samych fizycznych warunkach początkowych,
     lecz różnej rozdzielczości siatki obliczeniowej. (lewy panel: BB3d, BB,
     BBh, prawy: AB, ABh). Kontury wyznaczają tempo wzrostu $\log_{10}( s_0(k_x,
  k_z))$ wynikające z rozwiązania równania~ \mref{eq:disprel} dla średniego
  stanu wybranych łatek w momencie czasu  $T = T_{\textrm{ref}}$ (dla porównania
  por. Rysunek.~2 z pracy~\cite{YG05})}
   \label{fig9}
\end{figure*}
 
% \subsection{Convergence}
We varied the resolution of our simulations to check how our numerical scheme
affects the obtained solutions. As the streaming instability in general
"prefers" shorter wavelengths for the optimal growth, increasing resolution
always leads to more, smaller overdensities emerging during the course of
evolution (see Fig.~\ref{fig10}). However we note that our results
follow the fastest growing, linear modes (see Fig.~\ref{fig9}) and
that most phenomena described in previous sections i.e. cavitation (see
Fig.~\ref{fig3}) or sudden growth dumping of the tightly coupled
boulders in the gas dominated regime (see Fig.~\ref{fig4}) are independent of
the size of the smallest computational cell.

\begin{figure}
   \includegraphics[width=0.98\linewidth]{figures/fig10}
   \caption{Migawki gęstości pyły dla $t = 160\yr$ dla łatek pochodzących z
      oribt $R=3.5$ i $4.5\AU$ wyodrębnione z symulacji o tych samych warunkach
      początkowych lecz różnej rozdzielczości siatki obliczeniowej. Górny panel
      symulacja BB, zaś dolny symulacja BBh. Ze względu na właściwości
      niestabilności numerycznej, wyższa rozdzielczość promuje krótsze długości
      fali, aczkolwiek ostateczny rezultat w nieliniowej fazie ewolucji jest
      jakościowo identyczny.} 
   \label{fig10} 
\end{figure}

\section{3D run}
The course of the evolution of our single 3D (BB3D) closely follows the
corresponding 2D run BB (compare Fig.~\ref{fig5} and Fig.~\ref{fig1}). During
the linear phase of growth elongated clumps or rather sheets of dense dust are
formed, as the deviation from axis axisymmetry is almost negligible.
%
\begin{figure}
   \includegraphics[width=0.98\linewidth]{figures/fig5}
   \caption{Snapshot of dust density distribution from run BB3D taken at
   $t=500$~yr. Upper panel shows a slice through the disc midplane, whereas lower
   panel is a part of slice perpendicular to disc plane zoom out to the region
   that corresponds to physical span of 2D simulations.}
   \label{fig5}
\end{figure}
In order to estimate whether we should expect effects of self-gravity  in the
nonlinear evolution of streaming instability, we have used \yt{}~\citep{yt} clump
finding facility to identify all the largest disconnected contour spanning for
minimum value of 50 computational cells in BB3D run. Then we calculated their
total mass and velocity dispersion $\sigma$ to obtain Jeans' radius:
%
\begin{equation}
   R_J = 2 GM / \sigma^2.
\end{equation}
Subsequently we compared the average size of each individual clump $L =
\sqrt{\sum_{i\in{\{x,y,z\}}} \max (L_i^2)}$ and found that $L < R_J$ for every
identified overdense clump, indicating that the clumps fulfill the gravitational
binding condition (see Fig.~\ref{fig6}). This indicates that in certain
situations, i.e. for dust dominated runs of the semi-global disc configuration,
streaming instability acting in a selfgravitating medium can lead to planetesimals
formation~\citep{J07}.

\begin{figure} 
  \includegraphics[width=0.98\linewidth]{figures/fig6}
  \caption{
     Small patch showing projection of dust density along azimuthal axis from
     run BB3D with annotated contours of gravitationally bound clumps.}
  \label{fig6} 
\end{figure}
% vim: tw=80 ts=3: 
