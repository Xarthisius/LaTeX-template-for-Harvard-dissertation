%\begin{savequote}[75mm]
%   $\ldots$\qauthor{$\ldots$}
%\end{savequote}

\chapter{Liniowa analiza stabilności}
\label{sec:lsa}
Jednym z~celów naukowych poniższej pracy jest walidacja wyników eksperymentu
numerycznego poprzez porównanie z~tempem wzrostu najsilniejszych modów
otrzymanych z~liniowej analizy stabilności. W~tym rozdziale przedstawiono
poszczególne kroki liniowej analizy stabilności dla niestabilności strumieniowej
w oparciu o pracę~\citep{YG05}.

Analiza stabilności opiera się na trzech elementach:
\begin{enumerate}
   \item znalezieniu stanu równowagi dla zadanego układu równań,
   \item zaburzeniu układu znaną funkcją o małej amplitudzie 
   \item wyprowadzeniu modów własnych oraz ich tempa wzrostu.
\end{enumerate}
Niestety dla układu równań hydrodynamiki dwóch sprzężonych płynów w~rozciągłym
dysku keplerowskim nie istnieje warunek równowagowy. Spowodowane jest to
migracja pyłu na centrum grawitacji na skutek utraty momentu pędu przez tarcie
aerodynamiczne. Naturalną konsekwencją migracji są znaczne zmiany w~radialnym
profilu rozkładu gęstości pyłu. Należy jednak zauważyć, że charakterystyczna
skala czasowa migracji jest rzędu setek bądź więcej lat, przy założeniu iż mamy
do czynienia z~aglomeratami pyłu o rozmiarach metrowych. Z~tego względu możemy
założyć że zmiany w~profilu gęstości są procesem powolnym w~stosunku do typowego
tempa wzrostu niestabilności strumieniowej, co zostanie wykazane w~dalszej
części wywodu.

Kolejnym utrudnieniem jest radialna rozciągłość rozważanego dysku
okołogwiazdowego, która implikuje zmienność potencjalnego stanu równowagowego
wraz z~promieniem dysku. W~tym wypadku najogólniejszym podejściem byłaby
globalna analiza stabilności po przez rozwiązanie dwupunktowego problem
brzegowego (np.~\cite{PHM04, KH06}), jednakże takie podejście jest dużo bardziej
skomplikowane. Przedstawianą tu lokalną analizę należy zatem traktować jako
pierwsze przybliżenie pełnej analizy stabilności niestabilności strumieniowej.
Niestabilne mody uzyskane w~ramach liniowej analizy posłużą nam jako punkt
odniesienia dla modów uzyskiwanych w~globalnym eksperymencie numerycznym.

Najbardziej wygodnym układem do lokalnego opisu niestabilności strumieniowej
jest tzw. kostka ścinana~\footnote{ang. shearing box}~\citep{HGB95}, czyli
kartezjański układ współrzędnych, którego początek współporusza się z~płynem na
wybranej orbicie $R_0$ z~częstością keplerowską $\Omega_0 \equiv
\Omega\left(R_0\right)$. Zwyczajowo przyjmuje się, że oś $x$ jest skierowana
radialnie na zewnątrz, oś $y$ jest w~kierunku azymutalnym, zaś $z$ jest osią
wertykalną. Zgodnie z~pracą~\cite*{YJ07} układ równań ciągłości oraz ruchu dla
obu składników można wyrazić poprzez:
%
\begin{align}
\partial_t \rho_g &+ \mathbf{u}\cdot\nabla\rho_g - \frac{3}{2}\Omega x\partial_y\rho_g 
 = -\rho_g\nabla\cdot\mathbf{u},\label{eqc1}\\
\partial_t \rho_d &+ \mathbf{w}\cdot\nabla\rho_d - \frac{3}{2}\Omega x\partial_y\rho_d 
 = -\rho_d\nabla\cdot\mathbf{w},\label{eqc2}\\
\partial_t \mathbf{u} &+ \left(\mathbf{u}\cdot\nabla\right)\mathbf{u} 
 - \frac{3}{2}\Omega x\partial_y\mathbf{u} 
 = 2\Omega u_y \hat{\mathbf{x}} -\frac{1}{2}\Omega u_x \hat{\mathbf{y}} \notag\\
 &- \frac{\epsilon}{\tau_f}(\mathbf{u}-\mathbf{w}) -c_s^2\nabla\ln\rho_g 
 +2\eta\Omega^2 R \hat{\mathbf{x}},\label{eqm1}\\
\partial_t \mathbf{w} &+ \left(\mathbf{w}\cdot\nabla\right)\mathbf{w} 
 - \frac{3}{2}\Omega x\partial_y\mathbf{w}
 = 2\Omega w_y \hat{\mathbf{x}} -\frac{1}{2}\Omega w_x \hat{\mathbf{y}} \notag\\
 &- \frac{1}{\tau_f}(\mathbf{w}-\mathbf{u}), \label{eqm2}
\end{align}
%
gdzie wyrazy takie jak $(3/2)\Omega x$ po lewej stronie równań, pojawiają się na
skutek transformacji wszystkich prędkości względem liniowego, ścinanego przepływu
$\mathbf{v}_0 = -(3/2)\Omega x \hat{\mathbf{y}}$ w~rotującym układzie
współrzędnych. Warto nadmienić, że wyraz $-(1/2)\Omega \{u,w\}_x
\hat{\mathbf{y}}$ po prawej stronie równań ruchu \mref{eqm1}-\mref{eqm2} jest
sumą dwóch składników: $(-2\Omega \{u,w\}_x + (3/2)\Omega \{u,w\}_x)
\hat{\mathbf{y}}$ z~których pierwszy jest składową siły Coriolisa, a drugi
wynika z~odjęcia wspomnianego wcześniej przepływu średniego. Główną różnicą
pomiędzy równaniami \mref{eqm1} oraz \mref{eqm2} jest brak wyrazu ciśnieniowego
dla składnika pyłowego. YG05 zauważyli, że można w~spójny sposób uwzględnić
globalny, radialny gradient ciśnienia gazu w~ramach kostki ścinanej, poprzez
dodanie liniowego wyrazu, który jest sparametryzowany wielkością określającą
bezwymiarową miarę rotacji podkeplerowskiej:

\begin{equation}
\eta \equiv - \frac{\partial_R P}{2\rho_g\Omega^2 R} \sim \frac{c_s^2}{v_K^2}.
\end{equation}

Układ równań \mref{eqc1}-\mref{eqm2} posiada znane rozwiązanie równowagowe~\citep{N86}

\begin{align}
\bar{\mathbf{w}} &= \left[ 
 -2\tau_s\xi, \frac{\tau_s^2\xi - 1}{1+\epsilon},
 0
\right]\eta v_K, \label{eq:w0}\\
\bar{\mathbf{u}} &= \left[ 
 2\epsilon\tau_s\xi, -\frac{1 + \epsilon\tau_s^2\xi}{1+\epsilon},
 0
\right]\eta v_K, \label{eq:u0}
\end{align}
%
gdzie $\tau_s = \Omega \tau_f$ to bezwymiarowy \emph{czas
zatrzymania}~\footnote{ang. dimensionless stopping time} i~$\xi =
((1+\epsilon)^2 + \tau_s^2)^{-1}$.  Linearyzacja równań \mref{eqc1}-\mref{eqm2},
polega na rozbiciu zmiennych na część stałą $\bar{\mathbf{q}}$ oraz zaburzenie
$\mathbf{q}^\prime$. W~rezultacie $\mathbf{q} =
\bar{\mathbf{q}} + \mathbf{q}^\prime$, gdzie $\mathbf{q}=[\rho_d, w_x, w_y, w_z,
\rho_g, u_x, u_y, u_z]$. Zakładamy równocześnie, że zaburzenie przyjmuje postać
osiowo-symetrycznej fali płaskiej:

\begin{equation}
   \label{eq:planar}
   \mathbf{q}^\prime(x,z,t) = \tilde{\mathbf{q}}
 \exp\left[i(k_x x + k_z z~-\omega t)\right]
\end{equation}

Po~podstawieniu liniowego zaburzenia układ równań przyjmuje następującą postać

\begin{align}
-i(\omega- k_x\bar{w}_x)\tilde{\rho}_d &= 
 - i~\bar{\rho}_d(k_x\tilde{w}_x + k_z\tilde{w}_z), \label{lin1}\\
-i(\omega- k_x\bar{u}_x)\tilde{\rho}_g &= 
 - i~\bar{\rho}_g(k_x\tilde{u}_x + k_z\tilde{u}_z), \label{lin2}\\
-i(\omega- k_x\bar{u}_x)\tilde{\mathbf{u}} &= 
 2\Omega \tilde{u}_y\hat{\mathbf{x}} - \frac{1}{2}\Omega \tilde{u}_x
 \hat{\mathbf{y}}
 -\frac{\epsilon}{\tau_f}(\tilde{\mathbf{u}} - \tilde{\mathbf{w}}) \notag\\
  -\frac{\tilde{\rho}_d}{\bar{\rho}_g\tau_f}
  (\bar{\mathbf{u}} &- \bar{\mathbf{w}})
  - \frac{c_s^2}{\bar{\rho}_g}ik_x\tilde{\rho}_g\hat{\mathbf{x}} -
  - \frac{c_s^2}{\bar{\rho}_g}ik_z\tilde{\rho}_g\hat{\mathbf{z}}, \label{lin3}\\
-i(\omega- k_x\bar{w}_x)\tilde{\mathbf{w}} &= 
 2\Omega \tilde{w}_y\hat{\mathbf{x}} - \frac{1}{2}\Omega \tilde{w}_x
 \hat{\mathbf{y}} 
 - \frac{1}{\tau_f} (\tilde{\mathbf{w}} - \tilde{\mathbf{u}}), \label{lin4}
\end{align}
%
gdzie $\epsilon = \bar{\rho}_d/\bar{\rho}_g$. Układ równań
\mref{lin1}-\mref{lin4} można zapisać jako

\begin{equation}
 \eurom{A}(k_x,k_z,\omega)\tilde{\mathbf{q}} = 0,
 \label{eq:linset}
\end{equation}
gdzie 
\begin{equation}
 A =
 \setlength\arraycolsep{2pt}
 \begin{bmatrix}
    -i\tilde{\omega}_d & i~k_x \bar{\rho}_d & 0 & i~k_z \bar{\rho}_d & 0 & 0 & 0 & 0 \\
    0 & \frac{1}{\tau_f} -i \tilde{\omega}_d & -2\Omega & 0 & 0 & \frac{1}{\tau_f} & 0 & 0 \\
    0 & \frac{1}{2}\Omega & \frac{1}{\tau_f} -i \tilde{\omega}_d & 0 & 0 & 0 & \frac{1}{\tau_f} & 0 \\
    0 & 0 & 0 & \frac{1}{\tau_f} -i \tilde{\omega}_d & 0 & 0 & 0 & \frac{1}{\tau_f} \\
    0 & 0 & 0 & 0 & -i\tilde{\omega}_g & i~k_x \bar{\rho}_g & 0 & i~k_z \bar{\rho}_g \\
    \frac{\bar{u}_x - \bar{w}_x}{\tau_f \bar{\rho}_g} & -\frac{\epsilon}{\tau_f} & 0 & 0 &
    \frac{c_s^2}{\bar{\rho}_g i~k_x} & \frac{\epsilon}{\tau_f}-i\tilde{\omega}_g &
    -2\Omega & 0\\
    \frac{\bar{u}_y - \bar{w}_y}{\tau_f \bar{\rho}_g} & 0 & -\frac{\epsilon}{\tau_f} & 0 & 0 &
    \frac{1}{2}\Omega & \frac{\epsilon}{\tau_f} -i \tilde{\omega}_g & 0 \\
    0 & 0 & 0 & -\frac{\epsilon}{\tau_f} & \frac{c_s^2}{\bar{\rho}_g i~k_z} & 0 & 0 &
    \frac{\epsilon}{\tau_f} -i \tilde{\omega}_g
 \end{bmatrix}
%
\end{equation}
zaś $\tilde{\omega}_d = \omega - k_x \bar{w}_x$ i~$\tilde{\omega}_g = \omega -
k_x \bar{u}_x$.
%
Nietrywialne rozwiązania układu liniowego \mref{eq:linset} istnieją wtedy i~tylko
wtedy gdy równanie dyspersyjne 
\begin{equation}
 \det|\eurom{A}(k_x,k_z,\omega)|=0.
 \label{eq:disprel}
\end{equation}
%
Dla zadanych wartości $(k_x, k_x)$ relację \mref{eq:disprel} można rozwiązać ze
względu na $\omega$ i~dzięki temu otrzymać związki pomiędzy amplitudami
składowych wektora $\tilde{\mathbf{q}}$.
Tempo wzrostu niestabilności definiujemy jako urojoną część zespolonej
częstotliwości $s=\Im(\omega)$.
%

Początkowe stadia ewolucji niestabilności strumieniowej są podobne dla
wszystkich wartości parametrów $\epsilon$ i~$a$ t.j. dominują mody liniowe.
Dokładna analiza fazy liniowej została przedstawiona w
podrozdziale~\ref{sec:simulation_analysis}.
