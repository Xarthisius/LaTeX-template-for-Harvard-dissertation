\begin{savequote}[75mm]
Nulla facilisi. In vel sem. Morbi id urna in diam dignissim feugiat. Proin molestie tortor eu velit. Aliquam erat volutpat. Nullam ultrices, diam tempus vulputate egestas, eros pede varius leo.
\qauthor{Quoteauthor Lastname}
\end{savequote}

\chapter{Wprowadzenie}
\section{Chronologia procesu formowania się planet}
\section{Ważne pojęcia}
Od gazowego dysku, stopniowe przejście do efektów pyłu

\subsection{Struktura dysku protoplanetarnego}
\subsection{Niestabilność magnetorotacyjna}
\subsection{Niestabilność grawitacyjna}
\subsection{Oddziaływanie pomiędzy gazem, a pyłem}
\subsection{Radialny dryf pyłu}
\subsection{Sedymentacja i pułapkowanie pyłu (KHI)}
\subsection{Koagulacja i wzrost rozmiarów}
\subsection{Niestabilność strumieniowa}
\section{Cel pracy}
Głównym celem pracy jest zbadanie niestabilności strumieniowej w bardziej
realistycznym przyblizeniu radialnie rozciągłego dysku.

\newthought{Skąd wzięły się planety?} Pytanie które nurtuję ludzkość od
zamierzchłych czasów 

\section{Luźne myśli}

Formowanie się planet jest złożonym procesem, który wymaga wzrostu rozmiaru
mikrometrowych ziaren pyłu o wiele rzędów wielkości. Pomimo tego, że najmniejsze
drobiny pyłu są silnie sprzężone z gazem, mogą one dryfować zarówno w kierunku
radialnym jak oraz wertykalnym i podlegać wzajemnym zderzeniom. Przy odpowiednio
niskiej prędkości względnej, taka kolizja może prowadzić do tworzenia coraz to
większych aglomeratów cząstek~\citep{BW08}. Z drugiej strony, ciała o rozmiarach
setek czy tysięcy metrów są na tyle duże, iż opór aerodynamiczny stawiany na nie
przez gaz jest całkowicie zaniedbywalne, zaś dominująca dynamicznie siłą są
wzajemne oddziaływania grawitacyjne~\citep{KKI06}.

\par Nierozwiązaną dotąd zagadką współczesnej astrofizyki jest pośredni etap
wzrostu centymetrowych ziaren pyłu do kilometrowych głazów stanowiących budulec
planet. Istnieje szereg procesów, które przeciwdziałają możliwemu wzrostowi
rozmiaru ziaren pyłu lub nakładają silne więzy czasowe na formację systemów
planetarnych. Najsilniejszym ograniczeniem jest szybki radialny dryf dla ziaren
pyłu luźno związanych z gazem, tj. takich dla których charakterystyczna skala
czasowa dla tarcia aerodynamicznego jest porównywalna z ich okresem
orbitalnym~\citep{W77}. Ponadto typowe prędkości drobiny pyłu o rozmiarach od
$1\textrm{ cm}$ do $1\textrm{ m}$ zawierają się w przedziale $1\div10\textrm{ m
s}^{-1}$, co sprawia że najbardziej prawdopodobnym rezultatem zderzenia jest
fragmentacja bądź odbicie~\citep{Z10}.

\par Jednym z możliwych scenariuszy formowania się planet jest szybki wzrost
gęstości pyłu na skutek sedymentacji ziaren, którą wymusza pionowa składowa
grawitacji pochodzącej od gwiazdy macierzystej. Po przekroczeniu wartości
krytycznej gęsta warstwa pyłu rozpada się pod własnym ciężarem~\citep{GW73}.
Należy jednak mieć na uwadzę, że sam proces sedymentacji może prowadzić do
wzbudzenia się niestabilności Kelvina-Helmholza~\cite{JHK06}, a to przeciwdziała
tworzeniu się cienkiej i ciężkiej warstwy pyłu w płaszczyźnie dysku. Ostatnie
badania pokazują że odpowiednio masywne i metaliczne dyski są nie wrażliwe na
ten proces~\citep{L10}.

{\bf więcej o niestabilnościach powodujących turbulencje}

\par Zdecydowaną wadę powyższej hipotezy jest całkowite zaniedbanie globalnej
turbulencji występującej w dyskach okołogwiazdowych, będącej jedynym
mechanizmem zdolnym do wyjaśnienia obserwowanych temp akrecji materii na
formujące się gwiazdy w ramach tzw. teorii $\alpha$-dysków~\citep{SS73}.
Obecnie za dominujący proces odpowiedzialny za turbulencję uważa się niestabilność
magnetorotacyjna~\citep{BH98}. Dopuszcza ona obecność w dysku obszarów pozbawionych
turbulencji np. w miejscach o niewystarczającym stopniu jonizacji gazu, jednakże
istnieje szereg innych zjawisk które mieszają płyn~\citep{LP10}.  {\bf Tu by się
pewnie przydało opisać}

\par Pomimo tych niesprzyjających warunków istnieje proces który dominuje
ewolucję pyłu w momencie w którym stosunek koncetracji ziaren pyłu do gęstości
gazu zbliża się do jedności. Tym mechanizmem jest {\it niestabilność
strumieniowa} po raz pierwszy przedstawiona w pracy~\cite{YG05}. Okazuje się, że
połączenie komulowanie się pyłu w lokalnych maksimach w rozkładzie ciśnienia
gazu i wzjamne oddziaływanie pomiędzy tymi dwoma składnikami dysku, prowadzi do
znacznego wzrostu koncentracji ziaren pyłu~\citep{J11}. Nawet zaniedbując efekt
samograwitacji w trakcie ewolucji niestabilności strumieniowej lokalna gęstość
pyłu może zwiększyć się tysiąckrotnie~\cite{JY07}, co może prowadzić do
wytworzenia się grawitacyjnie związanych obiektów~\cite{J07}. Ostatnie badania
niestabilności strumieniowej skupiały się na różnych aspektach fizycznych które
mają wpływ na jej rozwój t.j.: uwzględnienie szerokiego spektrum rozmiaru
cząstek pyłu~\cite{BS10a}, wpływ globalnego gradientu ciśnienia w dysku
okołogwiazdowym~\cite{BS10b}, stratyfikacja dysku~\cite{T12}. Niemniej jednak
wszystkie publikacje naukowe były ograniczone do lokalnego przybliżenia dysku.


%%%%%%%%%%%%%%%%%%%%%%%%%%%%%%%%%%%%%%%%%%%%%%%%%%%%%%%%%%%%%%%%%%%%%%%%%%%%%%%%
% vim: tw=80 ts=3: 
