\begin{savequote}[75mm]
   You know nothing, Jon Snow.
\qauthor{Ygritte, A Song of Ice and Fire by George R. R. Martin}
\end{savequote}

\chapter{Wprowadzenie}
\newthought{Skąd wzięły się planety?} Pytanie, które nurtuje ludzkość od
zamierzchłych czasów. Konkretne rozważania teoretyczne dotyczące pochodzenia
planet mają długą historię, sięgającą\linebreak przynajmniej XVIII wieku, kiedy
to Immanuel Kant wysunął ,,Hipotezę mgławicową''~\cite{ImmanuelKant.etal:2008}.
Już wtedy unikalność Układu Słonecznego stanowiła przedmiot debaty. Dopiero na
początku XX wieku pogląd, iż układy planetarne są czymś powszechnym we
Wszechświecie, został zaakceptowany przez środowisko naukowe, a w~roku 1992
została odkryta pierwsza, pozasłoneczna planeta orbitująca pulsar PSR
1257+12b~\cite{1992Natur.355..145W}. Dziś znamy ponad 1800 układów planetarnych,
orbitujących gwiazdy znajdujące się na różnorakich etapach ewolucji. Pomimo tego
bogactwa danych obserwacyjnych i~wieloletnich badań teoretycznych, odpowiedź na
pytanie \emph{skąd wzięły się planety} pozostaje niejednoznaczna.

\section{Paradygmat powstawania planet}
\label{sec:paradigm}
\subsection{Narodziny gwiazdy}
Formowanie się planet jest nierozerwalnie związane z~narodzinami gwiazd, które
biorą swój początek w~gęstych, pyłowo--gazowych obłokach materii. Zanurzone w
gorącym ośrodku międzygwiazdowym, początkowo w~stanie równowagi termodynamicznej
z otaczającym je gazem, obłoki takie często występują w~ogromnych kompleksach i
obserwowane są jako ciemne mgławice molekularne~\cite{Tielens05}. W~ich pobliżu
odnajdywane są gwiazdy~\emph{T~Tauri} -- obiekty zmienne o~jasności większej niż
wynikałoby to z~ich temperatur efektywnych, co sugeruje ich młody wiek,
nieprzekraczający 1~\Myr~\cite{H62}. Obserwowane temperatury efektywne sugerują,
iż we wnętrzach nie panują dostatecznie wysokie temperatury, aby mogły zachodzić
już reakcje spalania wodoru~\cite{CK79}. Część obserwowanych gwiazd T~Tauri jest
częściowo zanurzona w~małych, ciemnych i~gęstych obłokach materii. Jak pokazują
obserwacje na falach radiowych i~w~podczerwieni, obłoki te są na tyle gęste, że
siła wynikająca z gradientu ciśnienia jest w~stanie zrównoważyć siłę pochodzącą
od samograwitacji~\cite{WT02}. Ich wewnętrzna struktura jest wysoce
hierarchiczna, tzn. we wnętrzu pojedynczego gęstego obłoku o masach rzędu
tysięcy mas słonecznych, rozciągającego się na wiele parseków, znajdują się dużo
gęstsze obiekty o masach rzędu $1\Msun$ i~rozmiarach rzędu $0.1\pc$~\cite{M85,
LSM93}. Obserwacje rotacyjnych linii emisyjnych molekuły NH$_3$ pozwalają
szacować typowe koncentracje gazu w~obłokach na $10^{4}\cm^{-3}$~\cite{BM89}.
Nieregularny brzeg obłoków, w~połączeniu z~ich w~przybliżeniu fraktalną
strukturą, interpretowany jest jako obecność silnej turbulencji w samych
obłokach~\cite{E00, FPW91}. Nie jest jasne, czy ta struktura jest przejściowym
etapem ewolucji całego kompleksu obłoków, czy też quasi-stacjonarnym
stanem~\cite{L94}. Źródła turbulencji można upatrywać w supernowych, wiatrach
gwiazdowych, promieniowaniu masywnych gwiazd oraz niestabilnościach związanych
z polem magnetycznym~\cite{NP03, MLK04}. 
Typowe temperatury obłoków molekularnych wynoszą od
10 do 20\K. Za efektywne chłodzenie początkowo odpowiada emisja w~podczerwieni
molekuły CO~\cite{MSWG82}, jednakże w~trakcie kolapsu grawitacyjnego gaz sprzęga
się termicznie z~pyłem, który wypromieniowuje nadwyżkę energii
w~podczerwieni~\cite{HN65, MI00} przez co temperatura całego zapadającego się
obłoku pozostaje stała w czasie.

\par Procesy zachodzące w~samych obłokach, tj. turbulencja, samograwitacja, lub
w ośrodku zewnętrznym, tj. wybuchy supernowych mogą powodować wzrost gęstości
poszczególnych zagęszczeń w~obłoku. W~momencie, w~którym obszar gęstej materii
przekroczy  masę krytyczną, nazywaną masą Jeansa $M_J$~\cite{J1902, J1928},
grawitacja przeważa i~chmura zaczyna się zapadać (panel~a
rys.~\ref{fig:planet}). Masa Jeansa zależy od temperatury kinetycznej ośrodka
$T$ oraz jego gęstości $\rho$~\cite{H64}:
%
\begin{equation} M_J \sim
   \left( \frac{k_B T}{G} \right) ^\frac{3}{2} {\rho}^{-\frac{1}{2}},
\end{equation}
%
gdzie $k_B$ jest stałą Boltzmanna, a $G$ jest stałą grawitacji.
Dla wymienionych powyżej typowych warunków panujących wewnątrz obłoków materii
międzygwiazdowej~\cite{BM89}, masa Jeansa przyjmuje wartość:
%
\begin{equation}
 M_J \approx 2.9 M_{\odot} \left(\frac{T}{10\K}\right)^{1.5} 
 \left(\frac{n}{10^4\cm^{-3}}\right)^{-0.5},
\end{equation}
%
gdzie $n = \rho_g / \mu \mH$ jest koncentracją cząstek materii.  Gdy masa obłoku
przekracza masę Jeansa, gradient ciśnienia gazu nie wystarcza do skompensowania
siły przyciągania grawitacyjnego i rozpoczyna się kolaps grawitacyjny w
charakterystycznej skali czasowej swobodnego spadku~\cite{Spitzer1978}:
%
\begin{equation}
   t_{\textrm{ff}} \sim \frac{1}{\sqrt{G\rho}} \sim 10^5\yr
   \left(\frac{n}{10^4\thinspace \cm^{-3}}\right)^{-0.5}.
\end{equation}
%
W~rzeczywistości gradient ciśnienia w~niewielkim tylko stopniu spowalnia
zapadanie się materii~\cite{T82}. Obliczenia numeryczne pokazują, że profil
gęstości materii w~zapadającym się, sferycznie symetrycznym obłoku, pozbawionym
pola magnetycznego i turbulencji, asymptotycznie zbiega do funkcji
proporcjonalnej do $r^{-2}$~\cite{L69}. W~rezultacie tylko niewielka część masy
obłoku formuje protogwiazdę, reszta materii zostaje uwięziona w~formie
rozciągniętej otoczki opadającej na obiekt centralny. Przy braku rotacji
i~zaniedbaniu wpływu pola magnetycznego, otoczka opada radialnie, w~tempie
proporcjonalnym do $c_s^3 / G$, gdzie $c_s$ jest izotermiczną prędkością
dźwięku.  Stała proporcjonalności wynosi od około jedności~\cite{S77} do
kilkudziesięciu~\cite{H77}.

\par Jak już wcześniej wspomniano, pole magnetyczne odgrywa istotną rolę w
dynamicznej ewolucji całego kompleksu obłoków molekularnych i~należy spodziewać
się obecności silnego pola magnetycznego w~zapadającym się obłoku molekularnym i
jego przeciwdziałania sile samograwitacji~\cite{MC99}. Jednym z możliwych
mechanizmów zapoczątkowujących kondensację materii w~centrum grawitacji jest
dyfuzja ambipolarna, która zachodzi w~skalach czasowych rzędu
$10^7$~lat~\cite{MZGH93}.  Dzięki stopniowemu zwiększaniu masy, centralna część
podtrzymywanego przez pole magnetyczne obłoku ulega powolnej kontrakcji, do
momentu osiągnięcia koncentracji materii rzędu $10^{5}\cm^{-3}$, dla której
siła samograwitacji przeważa nad ciśnieniem magnetycznym i~rozpoczyna się
niepohamowany kolaps~\cite{BM94, CB00}. Niedawne eksperymenty
numeryczne~\cite{JHCF13} sugerują, że uwzględnienie wpływu turbulencji podczas
kolapsu pozwala efektywniej przełamać przeciwdziałający samograwitacji wpływ
pola magnetycznego, nawet dla silnie namagnesowanego ośrodka.

\par W powyższych rozważaniach całkowicie zaniedbano wpływ rotacji na ewolucję
formującej się protogwiazdy. Większość obserwowanych obłoków materii, z której
formują się potem gwiazdy, rotuje~\cite{GBFM93}, co wydaje się być naturalną
konsekwencją turbulencji obecnej w~obłokach~\cite{BB00}. Typowy moment pędu,
szacowany w~obłokach protogwiazdowych, jest przynajmniej o rząd większy niż
moment pędu, który posiadałaby gwiazda rotując z~maksymalną prędkością
równoważącą siłę samograwitacji. 

Fakt ten implikuje konieczność uwzględnienia mechanizmu odpowiedzialnego za
radialny transport momentu pędu. Rotacja powoduje, że materia nie opada
centralnie na centrum grawitacji, lecz formuje dysk podtrzymywany przez
równowagę pomiędzy radialną składową siły grawitacji oraz siłę
odśrodkową~\cite{TSC84}. Biorąc pod uwagę masę opadającą z~otoczki, formujący
się dysk jest marginalnie stabilny, bądź jest całkowicie niestabilny
grawitacyjnie~\cite{SKBT94}. W~rezultacie tworzą się w~nim spiralne fale
gęstości, które pod wpływem momentu siły, wynikającego z~oddziaływania
grawitacyjnego porcji gazu tworzącego dysk, napędzają akrecję materii na
protogwiazdę~\cite{St00}.
Akrecja materii przebiega w sposób niezaburzony jeżeli fluktuacje gęstości
wysycają się na odpowiednio niskim poziomie. W przeciwnym razie dysk może
rozpaść się, tworząc układ podwójny lub wielokrotny.

\par Formujące się w~centrum zagęszczenie staje się nieprzezroczyste dla
termicznej emisji pyłu przy gęstościach gazu większych niż $10^{-13}\g\cm^{-3}$
$(2\cdot10^{10}$ H$_2\cm^{-3})$~\cite{L69}, w~wyniku czego temperatura wnętrza
obłoku zaczyna rosnąć. Kończy to etap izotermicznego kolapsu. Nieprzezroczyste
jądro obłoku staje się adiabatyczne (wykładnik adiabatyczny H$_2$: $\gamma =
7/5$~\cite{L69}) dla gęstości powyżej $10^{-12}\g\cm^{-3}$. W adiabatycznie
ściskanym gazie ciśnienie prowadzi do praktycznie
całkowitego zatrzymania kolapsu dla gęstości centralnej rzędu $2\cdot
10^{-10}\g\cm^{-3}$ i~osiągnięcia przez gaz równowagi hydrostatycznej. Gaz
z~otoczki nie przestaje być jednak akreaowany i~gęstość oraz temperatura jądra
cały czas rosną. Krytycznym momentem jest osiągnięcie przez gaz temperatury
$2000\K$, dla której następuje dysocjacja molekuły H$_2$ i~gwałtowny spadek
wykładnika adiabatycznego poniżej wartości $4/3$. Ta druga faza kolapsu zachodzi
podobnie jak początkowy kolaps izotermiczny. Faza ta trwa aż do momentu, kiedy
wodór zostanie zjonizowany przez wzrost temperatury i~wykładnik adiabatyczny
gazu wzrośnie do wartości $5/3$.  Wzrost ciśnienia prowadzi do uformowania
drugiego jądra, pozostającego w równowadze hydrostatycznej. Posiada ono dość
niewielką masę rzędu $10^{-3}\Msun$ i~rozmiar około $1\Rsun$~\cite{MI00}.
Całkowita masa formującej się protogwiazdy nie przekracza na tym etapie
$10^{-2}\Msun$.  Dominującym procesem staje się akrecja materii, która musi
dostarczyć pozostałe 99\% masy do rodzącej się gwiazdy. 

\begin{figure}
   \includegraphics[width=0.9\textwidth]{figures/chap1_sed.png}
   \caption[Rozkład widma energii dla gwiazdy HD 34282.]
     {Rozkład widma energii dla gwiazdy HD 34282~\cite{MME04}. Symbole oznaczają
      obserwacje różnymi metodami dla odpowiednich długości fali. Do obserwacji
      dopasowano następujące modele: linia przerywana model widma gwiazdowego
      dla obiektu typu A3V $(T\sim 8600\K)$, cienka ciągła
      linia model ciała doskonale czarnego dla $T=1400\K$,
      linia kropkowana reprezentuje model dysku o nachyleniu $i=56^o$, tempie
      akrecji $\dot{M} = 8.2\times10^{-9}\thinspace\Msun\yr^{-1}$
      rozciągającym się od $0.31\AU$ do $705\AU$.
   Obrazek pochodzi z~pracy~\cite{MME04}}
   \label{fig:sed}
\end{figure}

W~trakcie akrecji protogwiazda cały czas pozostaje niewidoczna dla obserwatora,
ponieważ jest przysłonięta przez pył znajdujący się w~opadającej otoczce.
Klasyfikacja obserwacyjna takiego układu określa obiekty na tym etapie mianem
\emph{klasy zero}~\cite{andre} (panel~b rys.~\ref{fig:planet}). Ich obserwowana temperatura jest
stosunkowo niska $(T \lesssim 30\K)$. Obiekty te charakteryzują się maksimum
w~rozkładzie energii promieniowania wypadającym w~dalekiej podczerwieni, bez
wykrywalnej nadwyżki w~bliskiej podczerwieni. Z czasem, kiedy z~otoczki ubywa
materii, obszar optycznie gruby staje się coraz mniejszy i~maksimum
wypromieniowywanej energii przesuwa się w~kierunku bliższej podczerwieni.
Ostatecznie światło samej gwiazdy przebija się przez pozostałość otoczki, dzięki
czemu można zaobserwować charakterystyczne, dwuskładnikowe widmo (patrz
Rysunek~\ref{fig:sed}). Obserwatorzy wprowadzili prostą klasyfikację tak młodych
obiektów dzieląc je na 4 klasy: 0, I, II i~III, które oznaczają położenie
maksimum w~rozkładzie energii promieniowania odpowiednio na falach:
submilimetrowych, dalekiej podczerwieni, bliskiej podczerwieni i~w~zakresie
widzialnym. Poszczególne klasy odpowiadają też różnym fazom akrecji materiału i
różnią się długością trwania. Protogwiazdy znajdują się w~klasie 0 (panel
b~rys.~\ref{fig:planet}) przez około
kilkadziesiąt tysięcy lat~\cite{FSSK06}. W~tym czasie następuje gwałtowna
akrecja materii.  Czas ewolucji protogwiazd przypisywanych klasie I (panel~c
rys.~\ref{fig:planet}) jest o rząd wielkości dłuższy (kilka $10^5$ lat), zaś
maksymalny wiek obserwowanych gwiazd T~Ta\-u\-ri (klasa II, panel d
rys.~\ref{fig:planet}) wynosi $10^6$ lat~\cite{HCGD98}.  Dla obiektów klasy III
(panel e rys.~\ref{fig:planet}) w~widmach zanikają wszelkie struktury
przypisywane materii okołogwiazdowej\footnote{ang. \emph{weak-line T~Tauri
stars}}. Pro\-to\-gwia\-zda traci materię z~dysku na skutek jego
fotoewaporacji~\cite{ACP06} i~wiatru gwiazdowego~\cite{PN86}. Wiele
obserwowanych gwiazd T~Tauri wykazuje w~obserwowanych widmach już tylko
szczątkową akrecję od $10^{-8}$ do $10^{-7} \Msun\rok^{-1}$~\cite{Hart98}. Tak
niski poziom akrecji nie jest już w~stanie znacząco wpłynąć na masę gwiazdy
i~proces jej formowania można uznać za zakończony.

\begin{figure}[p]
\centering 
\includegraphics[width=0.77\textwidth]{figures/planetformation.png}
\caption[Fazy formowania się mało masywnych gwiazd.]
   {Ilustracja przedstawia kolejne fazy formowania się mało masywnej gwiazdy
   wraz z~systemem planetarnym: a) kolaps grawitacyjny gęstego obłoku; b)
   oddziaływanie centralnego pola grawitacyjnego oraz siły odśrodkowej powoduje
   opadanie materii i~formowanie się dysku; c) faza FU Orionis: silna akrecja w
   dysku oraz wypływ materii w~okolicach osi obrotu; d) faza T~Tauri: zmniejsza
   się tempo akrecji do rzędu $10^{-8}\Msun\rok^{-1}$ oraz wypływu materii,
   rozpoczyna się proces formowania planet; e) zanika składowa gazowa, planety
otwierają przerwy w~dysku, następuje również ich migracja; f) cały gaz oraz
mniejsze ciała zostają pochłonięte przez planety lub usunięte z~dysku, układ
planetarny przyjmuje ostateczny kształt. Obrazek zamieszczony dzięki uprzejmości
Joanny Drążkowskiej}

\label{fig:planet}
\end{figure}

\subsection{Powstawanie planet}

Powstawanie planet najprawdopodobniej nie jest pojedynczym procesem,
lecz złożeniem wielu różnych etapów, w~których każdy kolejny krok bazuje na
produktach poprzedniego. Już w~połowie XX wieku Carl von
Weizsacker~\cite*{1943ZA.....22..319W} i~Gerard
Kuiper~\cite*{1951PNAS...37....1K} zapostulowali, iż dynamiczna ewolucja obłoku
protoplanetarnego prowadzi do utworzenia się lokalnych zagęszczeń gazu i~py\-łu.
Postulowany model zakłada, że mikroskopijne cząsteczki pyłu sklejają się ze sobą
tworząc coraz to większe ciała --- \emph{planetezymale}. Kiedy planetezymale
osiągną rozmiary rzędu $10\m$ ich wzajemne oddziaływanie grawitacyjne zaczyna
przeważać nad siłami tarcia aerodynamicznego. Następnie planetezymale zaczynają
się zderzać, formując obiekty o rozmiarach rzędu tysięcy kilometrów ---
protoplanety. Protoplanety zaś są na tyle masywne, iż są w~stanie akreować
otaczający je gaz. Bardziej szczegółowo można podzielić formowanie się planet na
4 etapy:

\begin{description}
   \item[i) koagulacja ziaren pyłu $\left(\mum \rightarrow \km\right):$] 
      Z doświadczeń laboratoryjnych~\cite{BW08} wynika, że \linebreak drobne
      cząsteczki pyłu mogą na skutek wzajemnych zderzeń zwiększać swoje
      rozmiary. ,,Spoiwem'' stają się siły van der Waalsa bądź oddziaływanie
      elektrostatyczne. Opierając się na analizie drogi swobodnej jednorodnej
      frakcji cząstek pyłu o promieniu $a$, można określić charakterystyczną
      skalę czasową koagulacji jako:
%
   \begin{equation}\label{coag} 
      t_{\textrm{coag}} % = \frac{1}{n_d \sigma \Delta v}
      \sim \frac{a}{\Delta v}\frac{\rho_p}{\rho_d} \approx 
      10^{-12} \rho_d^{-1}\yr\thinspace
      \left(\frac{a}{1\mum}\right)
      \left(\frac{\Delta v}{0.1\m\s^{-1}}\right)^{-1}
      \left(\frac{\rho_p}{3\g\cm^{-3}}\right),
   \end{equation}
%
   gdzie $\Delta v$ jest średnią prędkością względną cząstek, $\rho_p$ jest
   gęstością materiału budującego cząstki, natomiast $\rho_d$ jest gęstością
   ośrodka pyłowego w~dysku.  Biorąc pod uwagę typowe gęstości pyłu w~obłokach
   gwiazdowych $(\rho_d \sim 10^{-20}\g\cm^{-3})$, proces ten zachodzi w~skali
   czasowej milionów lat. Dla dysków protoplanetarnych typowe gęstości materii
   są rzędu $10^{-10}\g\cm^{-3}$. Jest to całkowita gęstość, uwzględniająca
   obydwa składniki: gaz i~pył. Przyjmuje się, że kanoniczna wartość stosunku
   gęstości pyłu do gęstości gazu $\epsilon$, wynosi 0.01. W rzeczywistości,
   wielkość ta nie jest znana, a przyjęta wartość jest stosunkiem gęstości pyłu
   do gęstości gazu, obserwowanym w~ośrodku międzygwiazdowym~\cite{FS03}. Przy
   tym założeniu, gęstość pyłu w dysku jest rzędu $10^{-12}\g\cm^{-3}$. Zatem
   proces koagulacji zachodzi w~skalach lat lub nawet dziesiątek lat, więc
   zgodnie ze wzorem~\mref{coag} należałoby oczekiwać powstania planetezymali.
   W~rzeczywistości, dla ziaren pyłu o rozmiarach decymetrów czy metrów pojawia
   się szereg procesów przeciwdziałających dalszemu wzrostowi. Ulega także
   zmianie średnia prędkość względna cząstek pyłu, modyfikując
   prawdopodobieństwo wyniku kolizji na korzyść fragmentacji, a nie
   koagulacji. Szczegółowy opis tych procesów znajduje się w kolejnych
   podrozdziałach.

\item[ii) oligarchiczny wzrost $\left(\km \rightarrow
   10^3\km\right)$:]
   Faza druga formowania się planet rozpoczyna się w~momencie, w~którym przeważa
   wzajemne oddziaływanie grawitacyjne pomiędzy planetezymalami. Spoiwem
   łączącym zderzające się obiekty staje się grawitacja. W~trakcie bliskich
   przelotów (ang. \emph{close encounters}) mało masywne planetezymale doznają
   znacznych przyspieszeń, przez co rosną ich prędkości względne, w
   przeciwieństwie do masywnych obiektów~\cite{WS93}. Dla ciał poruszających się z
   podobnymi prędkościami prawdopodobieństwo zderzenia jest większe, jako że
   grawitacja jest w~stanie zbliżyć ich trajektorie. W~rezultacie, obiekty
   najmasywniejsze w~danym obszarze dysku przybierają na masie najszybciej i
   zaczynają wywierać dominujący wpływ na dynamikę planetezymali w swoim
   sąsiedztwie~\cite{IM93}. 
   Ostatecznie, w~poszczególnych obszarach dysku
   protoplanetarnego formuje się pojedynczy obiekt -- jądro protoplanetarne
   oraz pozostaje pewna populacja mniejszych planetezymali~\cite{KI98}. Jądra
   protoplanetarne są też nazywane ,,oligarchami'', a etap ten określany jest
   jako ,,oligarchiczny wzrost''.
   Tempo przyrostu masy oligarchów silnie zależy od typowych rozmiarów
   planetezymali, które pozostały w~dysku. Mniejsze ciała silnie oddziałują z
   gazem poprzez tarcie aerodynamiczne, które ukoławia ich orbity, zwiększając
   równocześnie prawdopodobieństwo wychwytu przez pobliskie jądra
   protoplanetarne na skutek soczewkowania
   grawitacyjnego\footnote{soczewkowanie grawitacyjne należy rozumieć jako
      zwiększenie przekroju czynnego na zderzenie, spowodowanego oddziaływaniem
   grawitacyjnym, ang. \emph{gravitational focusing}, nie zaś jako zakrzywienie
promieni świetlnych w~polu grawitacyjnym masywnego ciała ang.
\emph{gravitational lensing}.}~\cite{R04}.
   Kiedy jądra protoplanetarne osiągają rozmiary rzędu $10^3\km$ ich oddziaływanie
   grawitacyjne jest na tyle silne, że znacząco zwiększa względne prędkości
   masywnych planetezymali. Duże prędkości względne podczas zderzeń powodują rozpad
   planetezymali na mniejsze obiekty~\cite{KB04}. Powstały ,,gruz'' jest dużo
   efektywniej akreaowany przez oligarchów~\cite{WS93}.
\item[iii) akrecja gazu]
   Po osiągnięciu rozmiarów rzędu $10^3\km$ i~mas rzędu $0.1\Mearth$, jądra
   planetarne są w~stanie wiązać grawitacyjnie gaz na swoich powierzchniach.
   Mniejsze planetezymale, przechodząc przez powstałą gazową otoczkę, są
   spowalniane przez tarcie aerodynamiczne, co zwiększa prawdopodobieństwo ich
   schwytania przez protoplanetarne jądro~\cite{II03}. Po osiągnięciu przez
   jądro masy rzędu $10\Mearth$, jest ono w~stanie efektywnie akreować gaz i
   tworzyć masywne otoczki, tworząc planety Jowiszo-podobne~\cite{Petal96}.
   Globalne oddziaływanie dysk -- protoplanety staje się istotne i~może
   prowadzić z~jednej strony do migracji protoplanet~\cite{Papa07}, a z~drugiej
   strony do otworzenia się przerw w~dysku~\citep{KKI06}. Model ten nosi nazwę
   nosi nazwę ,,akrecji na jądra'' (ang. \emph{core-accretion}). 
%
\item[iv) długoskalowa ewolucja dynamiczna:]
   Jest to etap zdominowany przez wzajemne oddziaływanie
   grawitacyjne pomiędzy utworzonymi planetami, a także z~gwiazdą
   macierzystą~\cite{CW98}.  Układ planetarny może na tym etapie utracić
   znaczną część masy składników stałych poprzez pochłonięcie planety przez
   gwiazdę, bądź wprowadzenie jej na orbitę hiperboliczną~\cite{DAA13}.
%
\end{description}
Kluczowym, aczkolwiek najmniej zbadanym etapem powyższego scenariusza, jest
formowanie się planetezymali. Założenie, że proces koagulacji zderzeniowej bez
przeszkód prowadzi do wzrostu rozmiarów ziaren pyłu od mikrometrów do
kilometrów, nie ma silnych podstaw doświadczalnych. Eksperymenty laboratoryjne
pokazują, że efektywność tego procesu zależy od szeregu czynników, takich jak:
skład chemiczny, porowatość, ładunek elektryczny i~wielu innych~\cite{SBT97,
GBZ10}.  Wzrost planetezymali może być znacznie przyspieszony przez tzw.
mechanizm Goldreicha i~Warda~\cite{GW73}, czyli fragmentację grawitacyjną
gęstego ,,poddysku'' pyłowego, tworzącego się na skutek opadania pyłu na
płaszczyznę dysku wokółgwiazdowego i~jego radialnego dryfu (patrz
Rysunek~\ref{fig:GW}).

\begin{figure}[h]
   \centering
   \includegraphics[width=0.9\textwidth]{figures/sedymentacja.png}
   \caption[Mechanizm Goldreicha-Warda.]
      {Ilustracja mechanizmu Goldreicha-Warda. Efektem sedymentacji i
      radialnego dryfu pyłu (a) może być powstanie cienkiej i masywnej warstwy
      składników stałych (b), która jest niestabilna grawitacyjnie.
      Fragmentacja pod wpływem samograwitacji może
      prowadzić do wytworzenia się planetezymali w~lokalnych zagęszczeniach pyłu
      (c). Obrazek pochodzi z~pracy~\cite{armitage}
   }
   \label{fig:GW}
\end{figure}

\par Aby zrozumieć niedostatki powyższej hipotezy, należy
dokładnie przeanalizować procesy zachodzące podczas ewolucji gazowo-pyłowego
dysku okołogwiazdowego. Syntetyczne zestawienie najbardziej istotnych
mechanizmów zostało przedstawione w~dalszej części tego rozdziału.

\par Dla gazowych olbrzymów powstających w masywnych dyskach
protoplanetarnych, alternatywna teoria zakłada kolaps i~fragmentację
grawitacyjną całego gazowo--pyłowego dysku~\cite{Boss97}. O tym, czy zaburzenia
w~gęstości materii w~dysku będą wzrastać nieograniczenie, decyduje równowaga
pomiędzy destabilizującym wpływem samograwitacji, a stabilizującymi
właściwościami ciśnienia oraz rotacji.  Wartość graniczna dla stabilności
cienkiego, osiowo symetrycznego dysku została pierwszy raz wyprowadzona przez
Toomre'a~\cite{T64} jako:
%
\begin{equation}
   Q = \frac{c_s\kappa}{\pi G \Sigma}\sim 1,
   \label{eq:toomre}
\end{equation}
%
gdzie $c_s$ to prędkość dźwięku, $\kappa$ częstość epicykliczna, $\Sigma$
gęstość powierzchniowa gazu. Kryterium to jest stosowane do globalnych,
stratyfikowanych dysków podlegających nieosiowosymetrycznym zaburzeniom.
Symulacje numeryczne wykazują fragmentację takich dysków dla $Q\sim
1$~\cite{NBAA98}. W ogólnym przypadku niestabilności grawitacyjnej dysku,
kryterium stabilności jest zależne od azymutalnej liczby falowej
$m$~\cite{BT87}. Parametr Toomre'a~\mref{eq:toomre} jest zdefiniowany dla
zaburzeń osiowosymetrycznych.
Symulacje numeryczne~\cite{Duris07} pokazują, że dla $Q\lesssim 1.7$ drobne
zaburzenia w dysku są wzmacniane i formują spiralne fale gęstości. 
Fale spiralne są źródłem dodatkowych momentów sił i fal
uderzeniowych, które odpowiadają za transport masy i~momentu pędu oraz
podgrzewanie gazu~\cite{YC85}, stabilizując dysk. Kryterium~\mref{eq:toomre} nie
uwzględnia również destabilizujących efektów promienistego (bądź konwekcyjnego)
chłodzenia gazu~\cite{BMD06}. Dokładny przebieg niestabilności grawitacyjnej
w~realistycznych modelach dysków okołogwiazdowych jest ciągle
niejasny~\cite{MB11, LC11}. Niemniej jednak symulacje numeryczne sugerują, że w
masywnych dyskach niestabilność grawitacyjna jest w~stanie wytworzyć związane
obłoki materii o masach rzędu od $1$ do $10\Mjup$~\cite{BHM10, FR11}. Ich dalsza
ewolucja przebiega podobnie jak w~modelu akrecji jąder, tj. pył sedymentuje na
centrum grawitacji, tworząc skaliste jądro w~skalach czasowych od $10^3$ do
$10^6$ lat~\cite{HB11, GHB12}. Skalując parametry dysku względem typowych
wartości parametrów fizycznych, obserwowanych w~dyskach protoplanetarnych,
parametr $Q$ można wyrazić jako:
%
\begin{equation}
   Q \sim 10^2 
   \left(\frac{T}{100\K}\right)^{0.5}
   \left(\frac{\Sigma}{10^3\g\cm^{-3}}\right)^{-1}
   \left(\frac{R}{1\AU}\right)^{-1.5}.
   \label{eq:Qemp}
\end{equation}
%
Z tego względu, niestabilność grawitacyjna ma znaczenie tylko dla zewnętrznych
obszarów dysku protoplanetarnego. Co więcej, związane grawitacyjnie obłoki
podlegają silnemu oddziaływaniu pływowemu~\cite{VH12}, które może prowadzić do
ich rozerwania, a także gwałtownej migracji w~kierunku protogwiazdy~\cite{BMP11}.
\par Niestabilność grawitacyjna niekoniecznie wyklucza się z~modelem akrecji na
jądra, ponieważ może zachodzić już w~trakcie początkowych etapów formowania się
układu protogwiazda -- dysk (pierwsze kilkaset tysięcy lat), w~przeciwieństwie
do paru milionów lat w~przypadku akrecji na jądra. Model oparty na działaniu
niestabilności grawitacyjnej tłumaczy powstawanie bardzo masywnych planet, które
w~wypadku modelu ,,akrecji na jądra'' wymagają czasu porównywalnego z czasem
życia dysku protoplanetarnego\cite{HBP13}.

\section{Ewolucja dysku protoplanetarnego}
Planety formują się w gazowo--pyłowym dysku, który powstaje podczas kolapsu
obłoku protogwiazdowego. Pewne szczególne mechanizmy oraz globalna dynamika mogą
temu procesowi pomagać, bądź mu przeciwdziałać. Poniższe akapity pokrótce
opisują strukturę dysku protoplanetarnego, najważniejsze efekty dynamiczne
związane z~samym gazem, a także wpływ dynamiki gazu na dynamikę
i~ewolucję pyłu. Pozwoli to wskazać problemy z~jakimi boryka się model ,,akrecji
na jądra'' i~naturalnie przejść do celu tej rozprawy.

\subsection{Struktura dysku protoplanetarnego}
Zachowanie momentu pędu w trakcie kolapsu grawitacyjnego implikuje, że w
rotującym obłoku materia nie opada bezpośrednio na obiekt centralny, lecz formuje dysk
w~płaszczyźnie prostopadłej do wektora całkowitego momentu pędu. Aby określić
przybliżone warunki fizyczne w~formującym się dysku, możemy posłużyć się
równaniami hydrodynamiki:
%
\begin{gather}
   \partial_t \rho_g + \nabla\cdot\left(\rho_g\mathbf{u}\right) = 0,
   \label{eq:hd1}\\
\partial_t \mathbf{u} + \left(\mathbf{u}\cdot\nabla\right)\mathbf{u} = 
-\nabla\Phi + -\frac{1}{\rho_g} \nabla P, \label{eq:hd2}
\end{gather}
%
gdzie $\rho_g$ jest gęstością gazu, $P$ ciśnieniem, a $\Phi$ potencjałem
grawitacyjnym. Jeżeli ponadto założymy, że dysk jest izotermiczny to:
%
\begin{equation}
   P = n k_{\textrm{B}} T = \frac{\rho k_{\textrm{B}} T}{m_{\textrm{H}}} = \rho
   c_s^2 
\end{equation}
gdzie $c_s$ jest izotermiczną prędkością dźwięku.
Równanie równowagi hydrostatycznej przybiera postać:
\begin{equation}
   -\frac{1}{\rho_g}\nabla P = -c_s^2\nabla\ln\rho_g.
\end{equation}
%Przy założeniu stacjonarności równanie~\mref{eq:hd2} 
%
Gdy dysk znajduje się w~równowadze hydrostatycznej, to wertykalne
przyspieszenie grawitacyjne:
%
\begin{equation}
   \partial_z \Phi = g_z = \frac{GM_\star}{r^2}\frac{z}{r} = \Omega^2 z,
\end{equation} 
%
gdzie $M_\star$ to masa gwiazdy macierzystej, $\Omega$ -- orbitalna częstość
keplerowska, $G$ -- stała grawitacji Newtona, jest równoważone przez gradient
ciśnienia gazu $\partial_z P / \rho_g$.
Łącząc powyższe założenia otrzymujemy rozkład gęstości gazu:
%
\begin{equation} \label{eq:zeq}
   \rho_g(z) = \frac{\Sigma_G}{H\sqrt{2\pi}} \exp \left[
   \frac{1}{2}\left(\frac{z}{H}\right)^2 \right],
\end{equation}
%
gdzie $\Sigma_g = \int \rho_g(z) dz$ jest gęstością powierzchniową, a
$H=\frac{c_s}{\Omega}$ to charakterystyczna skala grubości dysku.
%
%\par Zakładając w~pierwszym przybliżeniu że dysk jest optycznie gruby, t.j.
%absorbuję całkowicie promieniowanie pochodzące od gwiazdy i~następnie reemituje
%je jako ciało doskonale czarne, można pokazać~\cite{armitage07} że $T \propto
%r^{-3/4}$ i~co za tym idzie $c_s \propto r^{-3/8}$. Dokładniejsze szacunki,
%które lepiej oddają obserwowane dystrucje spektralne energii, można znaleźć w
%pracach~\cite{KenyonHART87, ChaingGold97} {\bf patrz armitage}.
%{\bf Tu raczej trzeba przedstawić tę wersję z~której wynika $T \propto
%r^{-1/2}$}
%
\par W~modelu stacjonarnym warunek równowagi sił radialnych można zapisać
równaniem:
%
\begin{equation}\label{eq:radial_balance}
\frac{u_\phi^2}{r} = \frac{GM_\star}{r^2} +
  \frac{1}{\rho_g}\frac{\textrm{d}P}{\textrm{dr}},
\end{equation}
gdzie $u_\phi^2$ jest prędkością orbitalną gazu.  Kolejne wyrazy
równania~\mref{eq:radial_balance} reprezentują: siłę odśrodkową, siłę
grawitacji, gradient ciśnienia. Wpływ tego ostatniego na globalny rozkład
prędkości jest rzędu $O(H/r)^2$. Dlatego dla cienkich dysków $(H/r \ll 1)$
z~dobrym przybliżeniem można przyjąć, że właściwy moment pędu gazu jest równy
momentowi pędu odpowiadającemu ruchowi keplerowskiemu. Z równania
\mref{eq:radial_balance} wynika zatem, że moment pędu jest monotonicznie rosnącą
funkcją promienia:
%
\begin{equation}\label{eq:angmom}
l = r^2\Omega = \sqrt{GM_\star r}.
\end{equation}
%
Aby materiał z~dysku mógł być akreowany przez gwiazdę macierzystą, w~układzie
musi działać mechanizm powodujący utratę, bądź chociaż redystrybucję momentu
pędu.

\subsection{Transport momentu pędu}
Dużą jasność fotometryczną oraz rozkład energii w widmie dla obserwowanych
obiektów protoplanetarnych~\citep{MME04} tłumaczy się poprzez konwersję energii
potencjalnej materii w~ciepło, które jest następnie wypromieniowywane. Dla
obiektów klasy I obserwowana jasność jest na tyle duża, że spodziewane tempo
akrecji może sięgać poziom rzędu $10^{-5}\Msun\yr^{-1}$.  Aby był możliwy radialny
przepływ materii w~kierunku gwiazdy macierzystej, potrzebny jest mechanizm
transportu momentu pędu. W~tym celu trzeba założyć, że ośrodek jest lepki. Zatem
należy zastosować równania Naviera--Stokesa, w których transport momentu pędu
jest konsekwencją siły lepkiej w różnicowo rotującym dysku (ostatni wyraz po
prawej stronie równania~\mref{eq:ns2}):

\begin{gather}
   \partial_t \rho_g + \nabla\cdot\left(\rho_g\mathbf{u}\right) = 0,
   \label{eq:ns1}\\
\partial_t \mathbf{u} + \left(\mathbf{u}\cdot\nabla\right)\mathbf{u} = 
-\nabla\Phi -\frac{1}{\rho_g} \nabla P + \frac{1}{\rho_g} \nabla \cdot \Pi.
\label{eq:ns2}
\end{gather}
%
Jeżeli dodatkowo założymy, że mamy do czynienia z~płynem newtonowskim, tensor
naprężeń można zredukować do postaci $\Pi = (\rho_g \nu)\nabla\cdot\mathbf{u}$,
gdzie $\nu$ jest lepkością kinematyczną. Całkując układ
równań~\mref{eq:ns1}-\mref{eq:ns2} w~kierunku wertykalnym i~dokonując prostych
przekształceń można otrzymać:
\begin{equation}\label{eq:sigma}
   \partial_t \Sigma_g =
   \frac{3}{R}\partial_R\left(\frac{1}{R\Omega}\partial_R\left(R^2\Sigma_g \nu
         \Omega\right)\right).
\end{equation}
Rozwiązaniem stacjonarnym równania~\mref{eq:sigma} jest warunek $\Sigma_g\nu =
\textrm{const}$, co przekłada się na tempo akrecji $\dot{M} = 3\pi\Sigma_g\nu$.
Z powyższego warunku wynika, że lepkość jest parametrem określającym akrecję
dyskową. Problemem pozostaje wskazanie mechanizmu fizycznego, który daje
przyczynek do lepkości.
\par Podstawowa lepkość, tj. lepkość molekularna $\nu_{\textrm{m}} \sim c_s
\lambda$, gdzie $\lambda = 1 / n\sigma$ to średnia droga swobodna molekuł gazu,
zaś $n$ to koncentracja molekuł gazu, a $\sigma$ ich przekrój czynny, dla
typowych wartości gęstości i~temperatury dysków protoplanetarnych wynosi
$\nu_{\textrm{m}}\sim10^5\cm^2\s^{-1}$~\cite{armitage}. Przekłada się to na
tempo akrecji na poziomie $\dot{M}\sim 10^{-17}\Msun\rok^{-1}$. 
Charakterystyczna skala czasowa takiego procesu $\tau \simeq R^2 /
\nu_{\textrm{m}}$ wynosiłaby $10^{13}\yr$. Z tego względu lepkość molekularną można
całkowicie zaniedbać w~dalszych rozważaniach.
\par W~słynnej pracy, Shakura i~Sunyaev~\citep{SS73} zaproponowali, że turbulencja
w dysku może być źródłem lepkości, znacząco przewyższającym
lepkość molekularną. Dla izotropowej turbulencji, maksymalna skala wirów w~dysku
powinna być proporcjonalna do charakterystycznej skali grubości dysku $H$, zaś
maksymalna prędkość ruchu turbulentnych nie powinna przekraczać prędkości
dźwięku, ponieważ fale uderzeniowe bardzo szybko dyssypują energię kinetyczną.
Shakura i~Sunyaev zaproponowali parametryzację lepkości turbulentnej:

\begin{equation}\label{eq:alpha}
\nu = \alpha c_s H,
\end{equation}
%
gdzie $\alpha$ jest bezwymiarowym parametrem określającym wydajność
turbulencji w~transporcie momentu pędu. Aby wyjaśnić obserwowane tempo akrecji
dla gwiazd \emph{T~Tauri}, parametr $\alpha$ powinien być rzędu $10^{-2}$.
Problemem pozostaje wyjaśnienie mechanizmu powstawania turbulencji w~dysku. Z
kryterium Rayleigh'a~\cite{C61}:
%
\begin{equation}
   \frac{\mathrm{d}}{\mathrm{d}r} j =
   \frac{\mathrm{d}}{\mathrm{d}r}\left(r^2\Omega\right) > 0,
\end{equation}
%
wynika, że hydrodynamiczny dysk keplerowski jest liniowo stabilny. Sytuacja
diametralnie się zmienia jeżeli uwzględnimy obecność w~układzie 
słabego pola magnetycznego. Balbus i~Hawley~\citep{BH91} pokazali, że przepływ
magnetohydrodynamiczny jest stabilny liniowo wtedy i~tylko wte\-dy, gdy:
%
\begin{equation}\label{eq:mri}
   \frac{\mathrm{d}}{\mathrm{d}r}\left(\Omega^2\right) > 0.
\end{equation}
%
Warunek \mref{eq:mri} \emph{nie jest} spełniony dla dysków keplerowskich. W
rezultacie, nawet słabe pole magnetyczne jest w~stanie wzmocnić wykładniczo
zaburzenia prędkości oraz pola magnetycznego w~gazie w~czasie kilku okresów
orbitalnych, powodując silną turbulencję. Proces ten określany jest mianem
niestabilności magnetorotacyjnej (MRI). Co więcej, zarówno symulacje
lokalne~\cite{DSP10}, jak i globalne~\cite{FD11} umożliwiają oszacowanie
parametru lepkości $\alpha$ dla dysku, w którym turbulencja została wzbudzona
przez MRI na $\alpha\sim 10^{-2}$.

\par Należy zauważyć, że niestabilność magnetorotacyjna wymaga choćby
szczątkowej jonizacji gazu. Jeżeli założymy, że źródłem jonizacji jest wysoko
energetyczne promieniowanie, pochodzące od gwiazdy macierzystej i~weźmiemy pod
uwagę strukturę pionową dysku~\mref{eq:zeq}, to dojdziemy do wniosku, iż
promieniowanie jest w stanie zjonizować materię tylko do pewnej głębokości
$h_{\textrm{ion}}$. Materia znajdująca się poniżej $h_{\textrm{ion}}$ jest
efektywnie osłaniana przez warstwy gazu o $h > h_i$.  Obszar $h < h_i$ jest
zatem \emph{martwą strefą}, w której niestabilność magnetorotacyjna nie
występuje~\cite{DFT10} i dlatego nie może być źródłem turbulencji. 
Obecność obszarów niepodatnych na niestabilność magnetorotacyjną w dysku
wymusiła potrzebę znalezienia innego mechanizmu odpowiedzialnego za generację
turbulentnej lepkości w martwych strefach. Jednym z obiecujących procesów jest
niestabilność baroklinowa~\cite{KB03, Kl04}. Przepływ baroklinowy występuje,
kiedy ciśnienie płynu zależy równocześnie od jego gęstości, jak i temperatury, w
przeciwieństwie do przepływu barotropowego, w którym ciśnienie jest funkcją
tylko gęstości. Aplikując obustronnie do równania~\mref{eq:hd2} operator
wektorowej rotacji, pomijając siły zewnętrzne i dokonując prostych przekształceń
matematycznych, otrzymujemy:
%
\begin{equation}
   \partial_t \mathbf{\omega} + (\mathbf{u}\cdot\nabla)\mathbf{\omega} =
   (\mathbf{\omega}\cdot\nabla)\mathbf{u} -
   \mathbf{\omega}\left(\nabla\cdot\mathbf{u}\right) + \rho^{-2}\nabla \rho \times
      \nabla p,
   \label{eq:vort}
\end{equation}
%
gdzie $\mathbf{\omega}=\nabla\times\mathbf{u}\;$ jest wirowością.  W wypadku
istnienia w dysku radialnego gradientu entropii, nachylenie pomiędzy izopykną,
tzn. powierzchnią o stałej gęstości płynu, a izobarą, tzn. powierzchnią o stałym
ciśnieniu, sprawia że ostatni wyraz w równaniu~\mref{eq:vort} jest
niezerowy~\cite{LK11}. Baroklinowość jest m.in. odpowiedzialna za tworzenie się
cyklonów i antycyklonów w atmosferze Ziemi. Istotną rzeczą przy przepływie
baroklinowym w dyskach okołogwiazdowych jest skończona bezwładność
cieplna~\cite{P07a, P07b}. W przypadku, kiedy czas potrzebny płynowi na
zareagowanie na zmianę temperatury jest porównywalny z okresem obrotu
wytworzonego w nim wiru, wir ten jest w stanie wytworzyć gradient entropii wokół
siebie. Lokalny gradient entropii równoważy globalny gradient entropii, który
był początkowo odpowiedzialny za utworzenie wiru~\cite{KB03}. Gradient entropii
oddziałuje zwrotnie z wirem i dzięki wyporności generuje więcej wirowości.
Dochodzi do dodatniego sprzężenia zwrotnego i wir rośnie. Trójwymiarowe
symulacje~\cite{KB03} pokazują, że niestabilność baroklinowa może być źródłem
turbulencji zapewniającej parametr lepkości $\alpha$ na poziomie od
$10^{-3}$ do $10^{-2}$.

\subsection{Oddziaływanie pomiędzy gazem, a pyłem}
Oddziaływanie pomiędzy pyłem a gazem odbywa się poprzez tarcie aerodynamiczne.
Charakterystyczną skalę czasową tego procesu można wyrazić jako~\cite{W73}:
%
\begin{equation}
   \tau_f = \frac{mv_\textrm{pg}}{F_\textrm{D}},
\end{equation}
%
gdzie $m$ i~$v_{\textrm{pg}}\equiv|\mathbf{u} - \mathbf{v}|$ to odpowiednio masa
i prędkość cząstek pyłu względem gazu, zaś $F_\textrm{D}$ to siła tarcia.
$\tau_f$ można interpretować jako czas potrzebny do wytracenia pędu cząstki pyłu
i zrównania jej prędkości z~gazem. Siła $F_\textrm{D}$ jest definiowana w~różny
sposób, w~zależności od rozmiaru ziaren pyłu. Dla cząstek pyłu o promieniu $a < 9
\lambda / 4$ siła tarcia wyraża się prawem Epsteina~\cite{W77}:
%
\begin{equation}
   F_\textrm{D} = \frac{4}{3}\pi a^2 \rho_\bullet \rho_G c_s v_\textrm{pg}, 
\end{equation}
%
gdzie $\rho_\bullet = 1.6\g\cm^{-3}$ jest gęstością materiału, z~którego
zbudowany jest pył. Dla ziaren pyłu o rozmiarach porównywalnych z~drogą swobodną
molekuł gazu, siła tarcia wyraża się poprzez prawo Stokesa~\cite{W77}:
\begin{equation}
  F_\textrm{D} = C_\textrm{D} \pi a^2 \rho \frac{v^2}{2},
\end{equation}
gdzie $v$ to prędkość gazu, zaś współczynnik $C_\textrm{D}$ jest zależny od
liczby Reynoldsa~\cite{W77}:
\begin{equation}
   \textrm{Re} = 2 a \rho \frac{v}{\eta}.
   \label{eq:Re}
\end{equation}
Parametr $\eta$ w równaniu \mref{eq:Re} jest lepkością gazu.
W~niniejszej pracy skupiono się
na obszarach dysków rozciągających się od relatywnie dużych promieni, tj.
$>2\AU$.  Biorąc pod uwagę, iż średnią drogę swobodną możemy wyrazić przez
$\lambda_g = 4.2\times 10^4\textrm{ cm} (10^{-14}\textrm{ g cm}^{-3}/\rho_g)
\approx (R/1 \textrm{AU})^{2.75}\cm$~\citep{W77,BT09}, gdzie $R$ jest odległością
radialną od centrum dysku, prawo Epsteina ma zastosowanie dla dominującej części
domeny obliczeniowej nawet dla największych symulowanych przez nas ziaren pyłu.
Skalę czasową tarcia można wyrazić wzorem:
%
\begin{equation} 
   \tau_f = \frac{\rho_\bullet a} {\rho_g \sqrt{c_s^2 +
      |\mathbf{u} - \mathbf{w}|^2 }}. \label{eq:tauf}
\end{equation}
%
Powodem różnicy prędkości pomiędzy pyłem a gazem może być wiele procesów,
przede wszystkim turbulencja obecna w~gazie, ale także radialny dryf pyłu (o
którym będzie mowa w~dalszej części tego rozdziału) oraz ruchy Browna.  Te
ostatnie są wynikiem transportu momentu pędu na skutek zderzeń ziaren pyłu z
cząsteczkami gazu~\cite{E1905}.  Stochastyczność tych zderzeń przekłada się na
losowe błądzenie ziaren pyłu ze średnią prędkością:
%
\begin{equation}
   \Delta v_{\textrm{BM}} = \sqrt{\frac{8k_{\textrm{B} T}}{\pi m_\mu}},
\label{eq:bm}
\end{equation}
%
gdzie $m_\mu = m_1m_2/(m_1+m_2)$ to zredukowana masa cząstek. Z
równania~\mref{eq:bm} wynika, że różnica prędkości maleje z potęgą $-3/2$
promienia ziaren pyłu. Z tego względu ruchy Browna są istotne tylko dla najmniej
masywnych ziaren pyłu o rozmiarach rzędu $\mu m$ i odgrywają znaczącą rolę dla
procesów koagulacji pyłu w~początkowej fazie formacji protoplanet~\citep{DD05}. 
Turbulencja i radialny dryf mają większe znaczenie dla cząstek o
rozmiarach pośrednich, które są słabiej sprzężone z chaotycznymi ruchami gazu.


\subsection{Radialny dryf pyłu}
Jedno z przybliżeń stosowanych w opisie dysków gazowo-pyłowych opiera się na
założeniu, że pył w~dysku protoplanetarnym można traktować jako bezciśnieniowy płyn. 
Jako że gęstość materii w~dysku zgodnie z założeniami jest malejącą funkcją
promienia, to dla izotermicznego gazu gradient ciśnienia jest ujemny dla całej
rozciągłości dysku. W~rezultacie, radialna składowa siły grawitacyjnej dla gazu
jest pomniejszona o gradient ciśnienia. Równanie równowagi sił radialnych:
%
\begin{equation}
   R\Omega_g^2 = \partial_R P / \rho_G + R\Omega_K^2
\end{equation}
%
implikuje, że prędkość orbitalna dla gazu jest nieznacznie mniejsza niż prędkość
keplerowska. Niedobór prędkości rotacji gazu $\delta v = \eta v_\textrm{K}$
względem prędkości orbitalnej, opisanej prawem Keplera, można określić 
bezwymiarowym parametrem~\cite{N86} wyrażonym jako:
%
\begin{equation}
   \label{eq:eta}
   \eta = \frac{\partial_R P}{2\rho_\textrm{G} R \Omega^2} = \frac{1}{2}
   \frac{c_s^2}{v_\textrm{K}^2} \partial_{\ln r} \ln \rho_{\textrm{G}} \approx
   \frac{c_s^2}{v_\textrm{K}^2}.
\end{equation}
%
Dla typowych wartość parametrów, na orbicie o promieniu $1\AU$, przy
$v_\textrm{K}\approx 30\km\s^{-1}$ i~$\eta \approx 10^{-3}$, różnica w~prędkości
orbitalnej między gazem, a pyłem, wynosi $\Delta v \approx 33\m\s^{-1}$.
W~efekcie cząstki pyłu odczuwają permanentny ,,wiatr w~oczy'' skierowany
przeciwnie do kierunku ruchu orbitalnego i~na skutek tarcia trąca swój moment
pędu. Efektywność tego procesu silnie zależy od rozmiaru ziaren pyłu. Drobne
cząstki są silnie związane z~gazem i przez to poruszają się z prędkością bliską
prędkości gazu. W rezultacie, efekt oporu aerodynamicznego jest zaniedbywalny.
Podobnie dla dużych i~masywnych obiektów ze względu na ich bezwładność.
Najsilniejsza utrata momentu pędu następuje dla ziaren pyłu spełniających
relację $\Omega \tau_f \sim 1$, co odpowiada cząstkom pyłu o rozmiarach $1\m$ na
orbicie o promieniu $1\AU$, bądź $10\cm$ w~zewnętrznych obszarach dysku (patrz
rysunek~\ref{fig:chap1_drift}).
%
\begin{figure}
   \includegraphics[width=0.9\textwidth]{figures/chap1_drift.png}
   \caption[Zależność pomiędzy rozmiarem pyłu, a maksymalną prędkością
   radialnego dryfu.]
   {Schematyczny rysunek zależności pomiędzy rozmiarem ziaren pyłu,
   a~maksymalną prędkością radialnego dryfu. Obrazek pochodzi
   z~pracy~\cite{W77}}
   \label{fig:chap1_drift}
\end{figure}

Ziarna te są w~stanie osiągać prędkości radialne rzędu
od $10^2$ do $10^3\cm\s^{-1}$~\cite{W77}, co przekłada się na ucieczkę tych cząstek w
kierunku centralnej gwiazdy w~skali czasowej rzędu setek lat!  Kolejną
konsekwencją zależności radialnego dryfu od rozmiaru cząstek jest zróżnicowanie
ich wzajemnej prędkości. Cząstki pyłu sklejają się dzięki oddziaływaniom
międzycząsteczkowym, które są szczególnie wydajne w~przypadku małych cząstek
i~dużego ich zagęszczenia. Wraz ze wzrostem cząstek proces zlepiania zachodzi
coraz wolniej.  Siły elektrostatyczne faworyzują małe cząstki, które mają większy stosunek
powierzchni do masy. Przy odpowiednio niskiej prędkości względnej, kolizje
prowadzą do tworzenia coraz to większych aglomeratów cząstek~\citep{BW08}, ale
dla dużych różnic w prędkościach najbardziej prawdopodobnym rezultatem zderzenia
jest fragmentacja bądź odbicie~\citep{Z10}. 

\par Czas, w którym ziarna pyłu o rozmiarach od $0.1$ do $10\m$ mogły
by rosnąć, jest niezmiernie krótki, ale nie jesteśmy w~stanie wskazać żadnego
mechanizmu powodującego dalszy wzrost rozmiarów pyłu. Problem ten nosi nazwę
\emph{metrowej bariery wzrostu} i~jest najważniejszą niewyjaśnioną kwestią
obecnego paradygmatu formacji planet. 

\par W~ogólności dryf radialny przemieszcza pył w~kierunku maksimum ciśnienia 
w~gazie. Dla modelu dysku, w którym rozkład gęstości jest opisany funkcją
wykładniczą jest to tożsame\linebreak z~centrum grawitacji. Należy jednak pamiętać, że
turbulentny ośrodek może wytwarzać lokalne, przejściowe maksima w~ciśnieniu
gazu, dla których gradient będzie dużo większy niż gradient globalny. Z tego
względu obszary o podwyższonym ciśnieniu będą działać jako swoiste pułapki na
pył, przejściowo zwiększając jego gęstość (patrz rysunek~\ref{fig:chap1_trap}).
Cuzzi, Hogan i~Shariff~\citep{CHS08} postulują, iż przejściowe zagęszczenia pyłu
na skutek ruchów turbulentnych są wyzwalaczem dla dalszych procesów formowania
się planetezymali.
%
\begin{figure}
   \centering
   \includegraphics[width=0.46\textwidth]{figures/chap1_gasdisk.png}
   \includegraphics[width=0.46\textwidth]{figures/chap1_dustdisk.png}
   \caption[Pułapkowanie pyłu w lokalnych maksimach ciśnienia gazu.]
     {Symulacja dwuskładnikowego, izotermicznego dysku
      pro\-to\-pla\-ne\-tar\-ne\-go.
      Lewy panel przedstawia rozkład ciśnienia gazu, prawy zaś rozkład gęstości
      gazu. Wy\-raź\-nie widać jak pył jest pułapkowany w~lokalnych maksimach
      ciśnienia. Obrazek za\-po\-ży\-czo\-ny~z pracy~\cite{RLP2006}}
   \label{fig:chap1_trap}
\end{figure}

\subsection{Sedymentacja i~niestabilność Kelvina-Helmholtza (KHI)}
W~dotychczasowych rozważaniach pominęliśmy wpływ pionowej składowej siły
grawitacji, pochodzącej od gwiazdy macierzystej, na dynamikę ziaren pyłu. O ile
gaz zostaje ściśnięty przez grawitację do momentu osiągnięcia równowagi
hydrostatycznej z gradientem ciśnienia~\mref{eq:zeq}, o tyle opadaniu pyłu nie
przeciwdziała żaden proces fizyczny\footnote{na chwilę pomińmy fakt, że gaz
jest ośrodkiem turbulentnym i~przez wzajemne sprzężenie wpływa na dynamikę
pyłu}. Sedymentacja mogłaby zatem prowadzić do wytworzenia się cienkiej i~bardzo
gęstej warstwy pyłu, która po przekroczeniu wartości krytycznej rozpadłaby się
pod własnym ciężarem~\citep{GW73}. Zauważmy jednak, że osiadanie pyłu prowadzi
do stopniowego zwiększenia stosunku gęstości pyłu do gazu w~płaszczyźnie dysku.
Ponadto, jak już było wspomniane w~poprzednim podrozdziale, pył porusza się na
wybranej orbicie z~prędkością keplerowską, natomiast gaz pod wpływem radialnego
gradientu ciśnienia -- z~prędkością podkeplerowską. Duża koncentracja pyłu
w~płaszczyźnie powoduje, że gaz jest efektywniej ,,pchany'' przez pył i~zaczyna
poruszać się szybciej niż warstwy gazu leżące poniżej i~powyżej płaszczyzny
symetrii dysku. Prowadzi to do wytworzenia się pionowego gradientu prędkości
azymutalnej gazu. Ruch płynu w płaszczyźnie $\phi$-\emph{z} można w takiej sytuacji
przybliżyć jako laminarny, stacjonarny przepływ. Eliminując wyrazy zależne od
czasu w równaniu~\mref{eq:vort}, otrzymujemy różniczkową postać prawa
Bernoulliego:
\begin{equation}
   \nabla\left(\frac{1}{2}u^2\right) -
   \mathbf{u}\times\left(\nabla\times\mathbf{u}\right) =
-\frac{1}{\rho}\nabla p - \nabla\Phi.
   \label{eq:bern}
\end{equation}
Można pokazać~\cite{C98}, że drobne zaburzenia przepływu ścinanego dla płynu o
różnej gęstości, opisywanego równaniem~\mref{eq:bern}, są efektywnie wzmacniane.
Mechanizm ten nosi miano niestabilności Kelvina-Helmholtza i~w~warunkach
opisanych powyżej prowadzi do pojawienia się turbulencji w~gazie znajdującym się
dysku protoplanetarnym. Na skutek wzajemnego sprzężenia, turbulentne ruchy gazu
skutkują mieszaniem składnika pyłowego i~hamowaniem sedymentacji~\cite{JHK06}.
Niedawne badania pokazują, że tylko bardzo masywne dyski, o~metaliczności dużo
większej niż słoneczna, nie są wrażliwe na ten proces~\citep{L10}. Podobnie jak
w~przypadku niestabilności magnetorotacyjnej, turbulentne ruchy gazu skutkują
analogicznymi ruchami pyłu, jednakże i~w~tym wypadku ustala się pewien stan
równowagi dynamicznej między sedymentacją pyłu, a turbulencją~\cite{JHK06}. 

\subsection{Niestabilność strumieniowa}
Pomimo działania turbulencji wywołanej poprzez szereg opisywanych wcześniej
niestabilności płynowych, w~dysku protoplanetarnym może uformować się warstwa
pyłu o skończonej grubości, dużo mniejszej niż grubość dysku gazowego. Pył ten
jest jednak zbyt rzadki, aby wzbudziła się w~nim niestabilność
grawitacyjna~\cite{JK05,JHK06}.
%
\par Pomimo tych niesprzyjających warunków, istnieje proces, który dominuje
ewolucję pyłu w~momencie, kiedy stosunek gęstości pyłu do gęstości gazu zbliża
się do jedności. Tym mechanizmem jest {\it niestabilność strumieniowa}, 
wynikająca z oddziaływania pyłu z gazem (istnieje  również podobna, znana od
dawna  niestabilność strumieniowa w plazmie),
po raz pierwszy przedstawiona w~pracy~\cite{YG05}.  Wraz ze wzrostem gęstości pyłu
spowodowanym dryfem w~kierunku najbliższego maksimum ciśnienia gazu,
wzrasta wypadkowa siła, z~jaką pył oddziałuje na gaz.  Prowadzi to zwiększenia
prędkości gazu i~wzrostu ciśnienia na skutek zagarniania obszarów gazów
poruszających się wolniej. Zwiększa to maksimum ciśnienia w~gazie i przyspiesza
dryft pyłu z~otaczających go obszarów. W~przypadku braku działania dodatkowych
sił, gradient ciśnienia skierowany na zewnątrz gęstniejącego obszaru powodowałby
jego szybkie rozmycie, ale w~wypadku rotującego dysku połączenie ciągłego dryfu
pyłu i~siły Coriolisa prowadzi do wytworzenia się równowagi
geostroficznej~\cite{JBL11}. Nasycenie gęstości pyłu następuje w~momencie, gdy
porcje pyłu rozpędzają się do wystarczająco dużych prędkości i są w~stanie
wyrwać się z~lokalnych maksimów gazu. Nawet zaniedbując efekt samograwitacji
w~trakcie ewolucji niestabilności strumieniowej, lokalna gęstość pyłu może
zwiększyć się tysiąckrotnie~\cite{JY07}, co może prowadzić do wytworzenia się
grawitacyjnie związanych obiektów~\cite{J07}.  Niedawne badania niestabilności
strumieniowej skupiały się na różnych aspektach fizycznych, które mają wpływ na
jej rozwój, tj. uwzględnienie szerokiego spektrum rozmiaru cząstek
pyłu~\cite{BS10a}, wpływ globalnego gradientu ciśnienia w~dysku
okołogwiazdowym~\cite{BS10b}, stratyfikacja dysku~\cite{T12}.  Niemniej jednak
wszystkie publikacje były ograniczone do lokalnego przybliżenia dysku.  Lokalne
przybliżenie prowadzi do szeregu uproszczeń m.in. zaniedbania globalnej migracji
pyłu, traktowania globalnego, radialnego ciśnienia gazu jako stałego i
niezmiennego współczynnika~\cite{N86}, czy stosowania bezwymiarowego czasu
zatrzymania przy obliczaniu wzajemnego oddziaływania obu płynów~\cite{YG05}.
Dopiero symulacje uwzględniające zmienność warunków fizycznych w~kierunku
radialnym dysku umożliwią odpowiedź na pytanie, czy niestabilność strumieniowa
może mieć znaczący przyczynek do procesu formowania się planet.

\section{Cel i plan pracy}
Celem badań jest oszacowanie i~opis wpływu niestabilności strumieniowej na
proces tworzenia zagęszczeń pyłowych w~globalnym dysku gazowo-pyłowym.  Pył jako
ważny składnik dysku okołogwiazdowego można opisywać w~przybliżeniu płynowym
(funkcja już zaimplementowana w~kodzie \textsc{PIERNIK}~\cite{piernik2}) lub
w~przybliżeniu punktów materialnych. Obecnie w literaturze naukowej częściej
stosowana jest druga metoda, ważna jest więc jakościowa i~ilościowa weryfikacja
własności niestabilności strumieniowej w przybliżeniu płynowym.  Główna część
badań przedstawionych w niniejszej pracy jest oparta na technice płynowych
symulacji numerycznych w ujęciu wielopłynowym. W kolejnym 2. rozdziale omówiono
w skrócie podstawowy formalizm liniowego opisu niestabilności strumieniowej w
gazowo-pyłowych dyskach protoplanetarnych.  Liniowa analiza niestabilności
strumieniowej zostanie wykorzystana do obliczenia tempa wzrostu oraz liczb
falowych najbardziej niestabilnych modów niestabilności strumieniowej. Wyniki
liniowej analizy stabilności zostaną wykorzystane do weryfikacji wyników
symulacji numerycznych.  Drugim, niemniej ważnym celem pracy jest zbadanie, czy
niestabilność strumieniowa z uwzględnieniem samograwitacji płynu w stabilnym
grawitacyjnie globalnym dysku gazowo-pyłowym może prowadzić do wytworzenia się
grawitacyjnie związanych obiektów, które następnie posłużą jako materiał do
formowania planet.
\par Plan pracy obejmuje przedstawienie liniowej analizy stabilności układu
opisującego dwuskładnikowy model dysku protoplanetarnego w dwuwymiarowym
przybliżeniu kostki ścinanej w~rozdziale drugim. W rozdziale trzecim zostały
opisane użyte metody numeryczne oraz zastosowane warunki początkowe. Rozdział
czwarty przedstawia wyniki przeprowadzonych eksperymentów numerycznych, zarówno
dwu- jak i~trójwymiarowych, z~udziałem jak i~bez udziału samograwitacji. Ponadto
w~tymże rozdziale porównano wyniki symulacji z~liniową analizą stabilności. W
rozdziale piątym przedyskutowano otrzymane wyniki w~kontekście dalszej ewolucji
układu i~formowania się planet. Praca kończy się krótkim podsumowaniem
w~rozdziale szóstym.

%%%%%%%%%%%%%%%%%%%%%%%%%%%%%%%%%%%%%%%%%%%%%%%%%%%%%%%%%%%%%%%%%%%%%%%%%%%%%%%%
% vim: tw=80 ts=3: 
