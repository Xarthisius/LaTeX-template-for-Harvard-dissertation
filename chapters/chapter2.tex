\begin{savequote}[75mm]
Strike hot iron and call forth sparks; strike a man and call forth fury; to shape man or metal to thy will, thou must
strike with force.
\qauthor{Collected Sermons of Carras, Thief by Looking Glass Studios}
\end{savequote}

\chapter{Eksperyment numeryczny}
\newthought{Lorem ipsum dolor sit amet}, consectetuer adipiscing elit. Morbi commodo, ipsum sed pharetra gravida, orci
\section{Metodyka}
W ramach poniższej pracy doktorskiej przeprowadzono szereg eksperymentów
numerycznych przy użyciu autorskiego kodu siatkowego PIERNIK, który opiera się
na:
\begin{itemize}
   \item zachowawczym schemacie numerycznym Relaxing TVD~\cite{jin-xin-95} w
      połączeniu z dzielonym, kierunkowym całkowaniem przestrzennym i czasowym
      przy użyciu algorytmu Runge-Kutta drugiego
      rzędu~\cite{2003PASP..115..303T,2003ApJS..149..447P}.
   \item implementacji wielopłynowości, tj. możliwości symulowania wielofazowego
      ośrodka np. płynu neutralnego i płynu bezciśnieniowego (pył) z
      uwzględnieniem oddziaływań międzypłynowych~\cite{piernik1,piernik2}
   \item solwerze multigridowym pozwalającym efektywnie rozwiązywać paraboliczne
      i eliptyczne równania różniczkowe, a w szczególności równanie Poissona
      opisujące samograwitację pyłu
   \item solwerze multipolowym pozwalającym na szybkie określenie warunków
      brzegowych dla potencjału grawitacyjnego.
   \item implementacji cylindrycznego układu współrzędnych w postaci
      zachowującej moment pędu~\cite{M07,SO10}.
   \item implementacji algorytmu szybkiego transportu
      eulerowskiego~\footnote{ang. \emph{Fast Advection in Rotating Gaseous Objects --
      FARGO}}~\citep{fargo} w ujęciu wielopłynowym.

\end{itemize}
Ponadto PIERNIK korzysta z algorytmu typu \emph{Constraint Transport}
zapewniającego bez źródłowość ewolucji pola magnetycznego, a także możliwość
wykonywania symulacji przy użyciu siatek adaptywnych~\footnote{ang.
\emph{Adaptive Mesh Refinement -- AMR}}.
\par Obliczenia opisywane w tej pracy zostały przeprowadzone z
wykorzystaniem:

\begin{itemize}
   \item infrastruktury PL-Grid, w szczególności klastrów
      komputerowych Galera+ (TASK), Hydra (ICM), Zeus (ACK CYFRONET), Inula
      (PCSS) w ramach grantu \emph{plggpiernik},
   \item klastrów PRACE, w szczególności klastrów komputerowych
      Cartesius (SurfSara), Fionn (ICHEC) w ramach grantu \emph{PIERNIK-SI} w
      projekcie DECI-11
\end{itemize}
Sumaryczne zużycie wyniosło kilka milionów CPUh.


%We conduct numerical simulations with the aid of a
%parallel MHD code PIERNIK using the cylindrical coordinate system. 


\subsection{Podstawowe równania}
Globalna dynamika dysku okołogwiazdowego można opisać poprzez dwa, wzajemnie ze
sobą oddziałujące płyny: neutralny gaz podlegający izotermicznemu równaniu
stanu oraz pył jako bezciśnieniowy płyn. Równania hydrodynamiki przyjmują dla
takiego modelu następującą postać:

% CONSERVATIVE FORM
\begin{align}
\partial_t \rho_g &+ \nabla\cdot\left(\rho_g\mathbf{u}\right) = 0,\\
\partial_t \rho_d &+ \nabla\cdot\left(\rho_d\mathbf{w}\right) = 0,\\
\partial_t \left(\rho_g\mathbf{u}\right) &+
   \nabla\cdot(\mathbf{u}\otimes(\rho_g\mathbf{u})+P) \notag\\
 &= -\rho_g\left(\nabla\Phi +
\frac{\rho_d}{\tau_f\rho_g}(\mathbf{u}-\mathbf{w})\right),\label{eq3}\\
\partial_t \left(\rho_d\mathbf{w}\right) &+
\nabla\cdot(\mathbf{w}\otimes(\rho_d\mathbf{w})) \notag\\
 &= -\rho_d\left(\nabla\Phi + \frac{1}{\tau_f}(\mathbf{w}-\mathbf{u})\right)
\label{eq4}.
\end{align}
% NON-CONSERVATIVE FORM
%\begin{align}
%\partial_t \rho_g &+ \nabla\cdot\left(\rho_g\mathbf{u}\right) = 0,\\
%\partial_t \rho_d &+ \nabla\cdot\left(\rho_d\mathbf{w}\right) = 0,\\
%\partial_t \mathbf{u} &+ \left(\mathbf{u}\cdot\nabla\right)\mathbf{u} = 
% -\nabla\Phi + \frac{\rho_d}{\tau_f\rho_g}(\mathbf{w}-\mathbf{u})
% -c_s^2\nabla\ln\rho_g,\label{eq3} \\
%\partial_t \mathbf{w} &+ \left(\mathbf{w}\cdot\nabla\right)\mathbf{w} = 
% -\nabla\Phi - \frac{1}{\tau_f}(\mathbf{w}-\mathbf{u}),\label{eq4}
%\end{align}

\noindent gdzie $\rho_g$, $\rho_d$ to odpowiednio gęstości gazu i pyłu,
$\mathbf{u}$, $\mathbf{w}$ ich prędkości, $P$ to ciśnienie gazu, $\tau_f$ jest
skalą czasową tarcia~\footnote{ang. friction time}~\mref{eq:tauf}, a $\Phi$ to
potencjał grawitacyjny.

\par Ze względu na specyfikę zagadnienia badanego w niniejszej pracy
najwygodniejszym układem współrzędnych jest układ cylindryczny. W kodzie
PIERNIK geometria cylindryczna jest zaimplementowana w formie zachowującej
moment pędu~\cite{M07,SO10}, która wprowadza tylko jeden, dodatkowy wyraz
źródłowy do równań ruchu~\mref{eq3} - \mref{eq4}: odpowiednio
$\left((\rho_g u_\phi + P) / R\right)\mathbf{\hat{R}}$ oraz $(\rho_d w_\phi / R)
\mathbf{\hat{R}}$.
\par Sprzężenie pomiędzy gazem a pyłem, które przyjmuje postać
$\rho_d/\tau_f\rho_g(\mathbf{u}-\mathbf{w})$ i
$1/\tau_f(\mathbf{w}-\mathbf{u})$ odpowiednio dla równań \mref{eq3} i \mref{eq4}
 w zależności od tego jak zostanie potraktowane, może prowadzić do
 znacznego skrócenia kroku czasowego. Z tego względu zdecydowałem o
 implementacji pół niejawnego schematu modyfikującego bezpośrednio prędkości gazu
 i pyłu, zaproponowanego przez~\cite{TB09}.

%W ramach pracy skupiono się na dyskach rozciągających się relatywnie dużych
%promienii tj. 2~AU. Zakładając, za pracą~\cite{CD93}, że przejście do
%reżimu Stokesa zachodzi dla ziaren pyłu o promieniu większym niż
%$a = 9/4\lambda_g$ gdzie $\lambda_g = 4.2\times 10^4\textrm{
%cm} (10^{-14}\textrm{ g cm}^{-3}/\rho_g) \approx (R/1 \textrm{AU})^{2.75}$~cm 
%jest średnią drogą swobodną molekuł gazu~\citep{W77,BT09}, zaś $R$ jest
%odległością radialną od centrum dysku. Przy tych założeniach, reżim Epstein ma
%%zastosowanie dla dominującej części domeny obliczeniowej nawet dla największych
%symulowanych przez nas ziaren pyłu. Skala czasowa tarcia przyjmuję zatem
%następującą postać:
%
%
\subsection{FARGO}
Ze względu na obecność rotacji, dyski keplerowskie stanowią mało wdzięczny
obiekt badań numerycznych, szczególnie w wypadku kiedy charakterystyczne
prędkości płynu osiągane w interesujących procesach są drobnym ułamkiem
prędkości rotacji. Zgodnie z warunkiem Couranta-Friedrichsa-Lewy'ego, stabilność
schematu numerycznego jest zapewniona wtedy i tylko wtedy, kiedy w jednym kroku
całkowania numerycznego sygnał nie propaguję się dalej niż o jedną komórkę
obliczeniową. W związku z tym, że zarówno prędkość rotacji dysku keplerowskiego
jak i azymutalny rozmiar komórki na siatce cylindrycznej jest malejącą funkcją
promienia, to najsilniejsze ograniczenie na rozmiar kroku czasowego wprowadza
dynamika gazu na najkrótszej symulowanej orbicie. Jedną z technik pozwalających
uniknąć powyższych więzów jest algorytm FARGO~\citep{Masset00}. Oryginalnie
został on zaprojektowany dla dwuwymiarowych dysków, lecz został rozszerzony
przez innych autorów~\cite{fargo2} do przypadków trójwymiarowych. Poniższa praca
rozwija go w kontekście wielu płynów.

\par FARGO opiera się na kierunkowym podziale części adwekcyjnej równań
hydrodynamiki. W kierunku radialnym i wertykalnym stosuje się klasyczny solwer
(w przypadku PIERNIKa RTVD), natomiast w kierunku azymutalnym rozbija się
adwekcję na trzy etapy:
\begin{enumerate}
   \item obliczenie średniej prędkości kątowej $\bar{\omega}_i$ dla każdego
      płynu i każdego promienia o indeksie $i$
      \begin{equation}
         \bar{\omega}_i = \frac{1}{N_\varphi~N_z} ~ \sum_{j,k} \omega_{i,j,k},
      \end{equation}
      gdzie $N_\varphi,\,N_z$ to odpowiednio liczba komórek w kierunku
      azymutalnym i wertykalnym.

   \item  obliczenie całkowitej liczby komórek dla przesunięcia w kierunku
      azymutalnym
      \begin{equation}
         n_i = {\tt Nint} \left( \bar{\omega}_i \Delta t/\Delta \varphi \right),
      \end{equation}
      gdzie {\tt Nint} oznacza funkcję określającą \emph{najbliższą liczbę
      całkowitą}. Co można wyrazić jako ,,prędkość przesunięcia''
      \begin{equation}
         \omega_i^{\rm SH} = n_i \frac{\Delta \varphi}{\Delta t}.
      \end{equation}
   \item obliczenie stałej wartości prędkości ,,rezydualnej'' dla każdego
      promienia
      \begin{equation}
         \omega_i^{\rm cr}= \bar{\omega}_i - \omega_i^{\rm SH}
      \end{equation}
   \item obliczenie właściwej prędkości ,,rezydualnej'' dla każdej komórki
      \begin{equation}
         \omega_{i,j,k}^{\rm res} = \omega_{i,j,k} - \bar{\omega}_i
      \end{equation}
\end{enumerate}

Powyższe cząstkowe prędkości kątowe sumują się do wyjściowej prędkości dla
poszczególnych komórek

\begin{equation}
   \omega_{i,j,k} = \omega_{i,j,k}^{\rm res} + \omega_i^{\rm cr} + \omega_i^{\rm SH}.  
\end{equation}

Adwekcja jest następnie wykonywana w trzech krokach, odpowiednio dla każdej
prędkości kątowej $\omega_i^{\rm SH}, \omega_i^{\rm cr}, \omega_{i,j,k}^{\rm
res}$: ostatnie dwie prędkości przy użyciu oryginalnego algorytmu numerycznego
(RTVD), zaś $\omega_i^{\rm SH}$ poprzez zwykłe przesunięcie wartości płynowych o
całkowitą liczbę komórek. Korzystając z faktu, że $\omega_i^{\rm SH} \gg
\max\left(\omega_i^{\rm cr}, \omega_{i,j,k}^{\rm res}\right)$, a tylko prędkości
po prawej stronie nierówności mają wpływ na warunek CFL, w znaczący sposób
zwiększamy krok czasowy. Co prawda, aby zachować stabilność algorytmu należy
zapewnić że przesunięcie w kierunku azymutalnym nie odseparuje dwóch sąsiednich
(w kierunku radialnym i wertykalnym) komórek, co przekłada się na warunek

\begin{equation}\label{eq:tshear}
   \Delta t_{\rm shear} = 0.5 ~ \min_{i,j,k} \left( \frac{\Delta\varphi}
   {|\omega_{i,j,k} - \omega_{i-1,j,k}|} \right)
\end{equation}

Niemniej jednak pomimo ograniczenia~\mref{eq:tshear}, dla typowych eksperymentów
numerycznych przeprowadzonych w tej pracy zastosowanie FARGO pozwoliło uzyskać
od 10 do 100 krotnego wydłużenia kroku czasowego.


\subsection{Potencjał grawitacyjny}
Potencjał grawitacyjny obecny w równianach ruchu~\mref{eq3} - \mref{eq4} można
rozbić na dwa składniki
\begin{equation}
   \Phi = \Phi_{\textrm{ext}} + \Phi_{\textrm{self}},
\end{equation}
gdzie $\Phi_{\textrm{ext}}$ jest stałym w czasie potencjałem pochodzącym od
gwiazdy macierzystej, zaś $\Phi_{\textrm{self}}$ potencjałem samograwitującego
płynu. Potencjał zewnętrzny $\Phi_{\textrm{ext}}$ został przyjęty jako potencjał
od masy punktowej
\begin{equation}
   \Phi_{\textrm{ext}} = -\frac{GM}{r} \mathbf{e}_r,
\end{equation}
gdzie $G$ to stała grawitacji, $M = 1\Msun$ masa obiektu centralnego, $r$
promień sferyczny, $\mathbf{e}_r$ radialny wersor kierunkowy.
Jako, że celem pracy jest wyizolowanie niestabilności strumieniowej z pośród
szeregu innych procesów, które mogą w dysku protoplanetarnym. Z tego względu
zaniedbano pionową składową przyspieszenia grawitacyjnego pochodzącą od
centralnego obiektu w dysku, która prowadziła by do naturalnej sedymentacji pyłu
w płaszczyźnie dysku i wzbudzenia się niestabilności
Kelvina-Helmholtza~\cite{JHK06}
\begin{equation}\label{eq:phiext}
   \Phi_{\textrm{ext}} = -\frac{GM}{R} \mathbf{e}_R.
\end{equation}

\par Potencjał $\Phi_{\textrm{self}}$ jest określony przez równanie Poissona
\begin{equation}\label{eq:poisson}
   \nabla^2 \Phi_{\textrm{self}} = 4\pi G \rho.
\end{equation}
Do rozwiązania równania \mref{eq:poisson} został użyty iteracyjny, solwer
multigridowy~\citep{doi:10.1137/S1064827598346235} połączony z solwerem
multipolowym~\citep{1977JCoPh..25...71J} w celu odpowiedniego obliczenia
potencjału na nieperiodycznych brzegach domeny obliczeniowej. Oba algorytmy
zostały zaimplementowane w PIERNIKu przez dra Artura Gawryszczaka (CAMK W-wa).

\section{Warunki początkowe}
Domena obliczeniowa we wszystkich eksperymentach rozciąga się pomiędzy $2\AU$, a
$7\AU$ w kierunku radialnym i ma $0.375\AU$ wysokości. W przypadkach
trójwymiarowych rozciągłość w kącie azymutalnym wynosi $\pi / 6$.
Zarówno w kierunku $z$ jak i $\phi$ zastosowano periodyczne warunki brzegowe,
zaś w kierunku radialnym użyto warunków odbiciowych, aby zapobiec ucieczce masy
z domeny obliczeniowej.

\par Początkowy rozkład gęstości materii w dysku określony jest poprzez formułę
wynikająca z przepisu Minimalnej Masy Mgławicy Słonecznej~\footnote{ang.
\emph{Minimal Mass Solar Nebula}}~\cite{H81}
\begin{equation}\label{eq:mmsn}
   \Sigma(R) = 1700 \left(\frac{R}{1\textrm{ AU}}\right)^{-3/2} 
   \textrm{ g cm}^{-2}.
\end{equation}
W przeciwieństwie do MMSN dla której profil temperatury jest wykładniczą funkcją
promienia, zakładamy że izotermiczny gaz posiada stałą temperaturę $T_0 = 170\K$
w całej objętości. Przyjmujemy iż zewnętrzny potencjał~\mref{eq:phiext} jest
określony przez masę punktową $M=1\,\textrm{M}_\odot$
Po mimo zaniedbywania pionowej składowej grawitacji, określamy charakterystyczną
pionową skalę wysokości $H$, aby oszacować gęstość przestrzenna na wybranym
promieniu, wykorzystując~\mref{eq:mmsn}, tak jakby gaz znajdował się w pionowej
równowadze hydrostatycznej
%
\begin{equation}\label{eq:rhoR}
   \rho(R,z) =  \rho(R,0) \exp\left(-\frac{z^2}{2H(R)^2}\right),
\end{equation}
gdzie $\rho(R,0)$ jest gęstością gazu w płaszczyźnie dysku, a $H^2 = 2 c_s^2 R^3/
GM$.
%
Z definicji gęstości powierzchniowej
\begin{equation} \label{eq:sigmaR}
   \Sigma(R) = \int_{-\infty}^\infty \rho(R,z) dz,
\end{equation}
%
Łącząc równanie~\mref{eq:sigmaR} oraz~\mref{eq:rhoR} otrzymujemy zależność
\begin{equation}
   \label{eq:rho}
    \rho(R,0) = \frac{\Sigma(R) }{\int_{-\infty}^\infty
   \exp\left(-\frac{z^2}{2H(R)^2}\right) dz}.
\end{equation}
Dla wybranej temperatury $T_0$ dysku, całka w mianowniku po prawej stronie
równanie~\mref{eq:rho} przyjmuje wartości z przedziału $[0.4,2]\AU$ dla $R \in
[2,7]\AU$. Dla ułatwienia obliczeń przyjmujemy, że wartość tej całki wynosi $1\AU$.
Warunek początkowy opiera się o radialną równowagę sił obliczoną niezależnie dla
składnika gazowego i pyłowego. Gaz utrzymywany jest w równowadze hydrostatycznej
pomiędzy grawitacją, siła odśrodkową i gradientem ciśnienia, pył natomiast
porusza się z prędkością keplerowską tak, aby równoważyć radialną składową
grawitacji.
\par Aby zminimalizować niefizyczne odbicia fal od radialnych brzegów domeny,
wprowadzono obszary tłumiące na wewnętrznych i zewnętrznych obszarach dysku o
szerokości $\sim0.5\AU$. W obszarach tych wszystkie wielkości płynowe są
poddawane ewolucji z dodatkowym wyrazem tłumiącym
\begin{equation}
  \frac{\textrm{d}X}{\textrm{d}t} = - \frac{X-X_0}{T_d}f(R).
\end{equation}
Ze względów stabilności numerycznej funkcja $f(R)$ ma skomplikowany przebieg
\begin{equation}
   \begin{split} 
      f(R) &= 1 - \tanh\left(\left(R - R_\textrm{in} + 1
      \right)^{f_\textrm{in}}\right)\\ &+ \max\left\{ \tanh\left(\left(R -
      R_\textrm{out} + 1\right)^{f_\textrm{out}}\right), 0\right\}, 
   \end{split}
\end{equation}
gdzie $X_0$ jest początkową wartością $X$, a $T_d$ jest skalą czasową tłumienia,
rzędu okresu orbitalnego na najniższej orbicie.
Wykładniki $f_\textrm{in}=f_\textrm{out}\equiv10$ określają szerokość przedziału
przejściowego pomiędzy obszarem ewoluującym bez tłumienia, a obszarem tłumionym
i zostały dobrane tak, aby tłumienie nie wpływało w znaczącym stopniu na
stabilność całego układu.
%
\subsection{Parametry symulacji}
%
W ramach pracy doktorskiej przeprowadzono szereg symulacji 2D modyfikując
początkowy $\epsilon$ oraz promień cząstek $a$, w celu weryfikacji użytych metod
i wskazania optymalnych parametrów dla pełnych symulacji trójwymiarowych. Aby
móc porównać wyniki z pracą innych autorów~\citep{JY07} parametr $\epsilon$
wybrano z przedziału $[0.2, 2.0]$. Taki dobór parametru $\epsilon$ pozwala także
wzbudzić morfologicznie różne manifestacje niestabilności strumieniowej, które
zostaną opisane w dalszej części pracy. Promień cząstek pyłu został dobrany,
poza zgodnością z JY07, tak aby ziarna pyłu były opisywane poprzez siłę tarcia
Epsteina i były na tyle małe żeby ich rozmiar można było wyjaśnić poprzez
zderzeniowy wzrost opisany w rozdziale 1. Pełne zestawienie użytych parametrów
zostało przedstawione w tabeli~\ref{tab1}.

\begin{table}
   \centering
   \begin{tabular}{cccccc}
      \hline
      Nazwa & $N_r \times N_\varphi \times N_z$ &
      $a$~[cm] & $\epsilon$ & $T_\textrm{end}$~[yr] \\
      \hline
      BD3d  &  $2560  \times 512 \times 192$  & 50  & 3.0 & 400  \\
      BD3dS &  $2560  \times 512 \times 192$  & 50  & 3.0 & 400  \\
      AA    &  $5120  \times 1   \times 300$  & 10  & 0.2 & 3000 \\
      AB    &  $5120  \times 1   \times 300$  & 10  & 1.0 & 3000 \\
      AC    &  $5120  \times 1   \times 300$  & 10  & 2.0 & 3000 \\
      BA    &  $5120  \times 1   \times 300$  & 50  & 0.2 & 3000 \\
      BB    &  $5120  \times 1   \times 300$  & 50  & 1.0 & 3000 \\
      BC    &  $5120  \times 1   \times 300$  & 50  & 2.0 & 3000 \\
      AAh   &  $10240 \times 1   \times 600$  & 10  & 0.2 & 1700 \\
      AAu   &  $20480 \times 1   \times 1200$ & 10  & 0.2 & 1800 \\
      ABh   &  $10240 \times 1   \times 600$  & 10  & 1.0 & 1400 \\
      BAh   &  $10240 \times 1   \times 600$  & 50  & 0.2 & 1730 \\
      BBh   &  $10240 \times 1   \times 600$  & 50  & 1.0 & 3000 \\
      \hline
   \end{tabular}
\caption{Parametry użyte w symulacjach. Kolumny opisują w kolejności: oznaczenie
   kodowe symulacji, ilość komórek obliczeniowych w kierunkach $r$, $\varphi$ i
   $z$, promień cząstek, początkowy stosunek gęstości pyłu do gęstości gazu,
całkowity czas trwania symulacji w latach.}
\label{tab1}
\end{table}

%%%%%%%%%%%%%%%%%%%%%%%%%%%%%%%%%%%%%%%%%%%%%%%%%%%%%%%%%%%%%%%%%%%%%%%%%%%%%%%%
% vim: tw=80 ts=3: 
