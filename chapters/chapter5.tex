%\begin{savequote}[75mm]
%\qauthor{Edward Mallory, The Difference Engine by William Gibson and Bruce
%   Sterling}
%\end{savequote}

\chapter{Podsumowanie}

Głównym osiągnięciem niniejszej pracy jest potwierdzenie hipotezy, że w globalnym
dysku gazowo-pyłowym, w którym oba składniki są traktowane jako płyn, zasadniczą
rolę w
powstawaniu gęstych obiektów pyłowych odgrywa niestabilność strumieniowa. 
Istotnym osiągnięciem jest również stwierdzenie, że niestabilność
strumieniowa w globalnym dysku gazowo-pyłowym, w obecności samograwitacji
materii, może prowadzić do wytworzenia się grawitacyjnie związanych obiektów.
Zebrany w niniejszej pracy materiał oparty na symulacjach numerycznych
pozwala stwierdzić, że:

\begin{itemize}
   \item zarówno dwu- jak i trójwymiarowe symulacje quasi-globalnego dysku
      gazowo-pyłowego, w którym oba wzajemnie ze sobą oddziałujące poprzez siłę
      tarcia składniki są traktowane w przybliżeniu płynowym, stanowią wiarygodne
      narzędzie do badania niestabilności strumieniowej;
   \item algorytmy numeryczne zaimplementowane w kodzie \textsc{Piernik}
      pozwalają na odtworzenie z wystarczającą dokładnością liniowej fazy
      wzrostu niestabilności strumieniowej;
   \item niestabilność strumieniowa prowadzi do wytworzenia się w gazowo-pyłowym
      dysku obszarów, w których gęstość pyłu jest ponad stukrotnie wyższa niż
      maksymalna początkowa gęstość pyłu;
   \item niestabilność strumieniowa z uwzględnieniem samograwitacji materii
      prowadzi do wytworzenia się populacji związanych grawitacyjnie, pyłowych
      obiektów o spektralnym rozkładzie masy danym funkcją potęgową o wykładniku
      $-1.25\pm0.12$;
   \item masa pyłu zgromadzona w pojedynczych, grawitacyjnie związanych obiektach
      odpowiada ciałom o rozmiarach rzędu kilkudziesięciu do kilkuset
      kilometrów;
   \item całkowita masa pyłu zgromadzona w grawitacyjnie związanych obiektach w
      przedstawionym modelu 
      jest wystarczająca do wytworzenia pojedynczego jądra gazowego
      olbrzyma, bądź dwóch lub trzech planet skalistych;
   \item dalsze prace nad modelem są potrzebne w celu uściślenia ilościowych
      szacunków całkowitej masy, liczby i spektrum masowego powstających
      planetezymali.
\end{itemize}
Zebrane w niniejsze pracy wyniki stanowią przyczynek do wyjaśnienia
kontrowersji tzw. \emph{metrowej bariery} w modelu akrecji na
jądra. Pomimo pewnych idealizacji i uproszczeń, obecnych w użytym modelu, praca
stanowi istotny krok na przód w kierunku zrozumienia procesów
prowadzących do powstawania planet.

%%%%%%%%%%%%%%%%%%%%%%%%%%%%%%%%%%%%%%%%%%%%%%%%%%%%%%%%%%%%%%%%%%%%%%%%%%%%%%%%
% vim: tw=80 ts=3: 
