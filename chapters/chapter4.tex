\begin{savequote}[75mm]
   If I model a phenomenon accurately, does that mean I understand it? Or might
   it be simple coincidence, or an artifact of the technique? Of course, as an
   ardent simulationist, I myself put much faith in Engine-modeling.
\qauthor{Edward Mallory, The Difference Engine by William Gibson and Bruce
   Sterling}
\end{savequote}

\chapter{Dyskusja i plany na przyszłość}
W ramach tej pracy przeprowadzono serię dwu- i trójwymiarowych symulacji
wzajemnie oddziałujących płynów: gazu i pyłu, w dysku keplerowskim. Przy czym
zaniedbano pionową składową grawitacji od masy punktowej umieszczonej w centrum
układu współrzędnych. Dla ziaren pyłu o rozmiarach od 10 do 50$\cm$
zaobserwowano gwałtowny wzrost zaburzeń w gęstości i prędkości pyłu na
charakterystyczych długościach fali zgodnych z przewidywaniami lokalnej,
liniowej analizy stabilności takiego układu. Wcześniejsze badania innych
autorów~\cite{YG05, JY07, TB09, BS10a, BS10b} opierały się w pełni na lokalnym
przybliżeniu kostki ścinanej, co prowadziło do szeregu uproszczeń m.in.
traktowania globalnego, radialnego ciśnienia gazu jako stałego i niezmiennego
współczynnika~\cite{N86}, stosowania bezwymiarowego czasu zatrzymania przy
obliczaniu wzajemnego oddziaływania obu płynów~\cite{YG05}. Bardzo ważnym
osiągnięciem poniższej pracy jest rozszerzenie modelu stosowanego przez innych
autorów poprzez uwzględnienie pełnej dynamiki okołogwiazdowego dysku w kierunku
radialnym i wykazanie, że w takim przypadku można wzbudzić niestabilność
strumieniową. Ponadto do opisu oddziaływania wiążącego oba płyny użyto
właściwego dla ich charakterystyk prawa tarcia aerodynamicznego~\mref{eq:tauf}
zamiast przybliżonego rachunku opartego na stałym, bezwymiarowym czasie
zatrzymania. Przyjęty model, choć stanowi wyraźny postęp względem innych prac, jest
ciągle mało realistycznym przybliżeniem pełnego dysku protoplanetarnego. Do jego
największych wad można zaliczyć brak stratyfikacji oraz nieuwzględnienie wpływu
globalnej turbulencji wywołanej niestabilnością magneto-rotacyjną~\cite{DKJ14}. 
%
\par Pierwszy etap niniejszej pracy opierał się na wykonaniu testów
zbieżnościowych dla różnych parametrów fizycznych dysku protoplanetarnego co
pozwolilo na określenie minimalnej liczby komórek obliczeniowych $(n\approx32)$
dla kodu \textsc{PIERNIK}, umożliwiającej na dokładne odwzorowanie liniowej fazy
wzrostu.  Otrzymane, we wszystkich przeprowadzonych eksperymentach numerycznych,
liczby falowe najszybciej rosnących modów niestabilności, a także ich pozycja na
mapie stabilności ($s(k_x, k_z)$), są zgodne z przewidywaniami liniowej analizy
niestabilności strumieniowej i potwierdzają właściwy wybór użytych metod
numerycznych. Kolejnym krokiem było wykonanie symulacji trójwymiarowych bez
samograwitacji i porównanie ich wyniku z wcześniejszymi symulacjami
dwuwymiarowymi. Uzyskane wyniki pokazują iż trójwymiarowe symulacje odtwarzają w
pełni przebieg ewolucji niestabilności strumieniowej, zgodnie z przewidywaniami
liniowej analizy stabilności przeprowadzonej dla układu zredukowanego do dwóch
wymiarów. Finalnym etapem pracy było przeprowadzenie w pełni trójwymiarowej
symulacji dwuskładnikowego dysku protoplanetarnego z uwzględnieniem wpływu
samograwitacji. Należy podkreślić, że pomimo założenia dużo większej niż
kanoniczna wartości stosunku gęstości gazu do gęstości pyłu, symulowany dysk
jest początkowo stabilny zgodnie z kryterium Toomre'a~\mref{eq:toomre}.
Parameter $Q$ jest stały dla całej domeny obliczeniowej i wynosi około $40$
(rów.~\ref{eq:Qemp}). Wzbudzenie się niestabilności strumieniowej prowadzi do
uformowania lokalnych zagęszczeń pyłu, których gęstość jest nawet stukrotnie
większa niż gęstość początkowa pyłu. Ilość zgromadzonej lokalnie masy pyłu
przekracza granicę stabilności grawitacyjnej i w rezultacie w układzie formuje
się znaczna liczby związanych grawitacyjnie obiektów, które należy interpretować
jako zalążki planetezymali. Pomimo tego, że zgromadzona w tych obiektach materia
pyłowa reprezentuje $50\cm$ ziarna pyłu, to ich całkowita masa jest
wystarczająca aby po ,,sprasowaniu'' na skutek grawitacji utworzyć ciała o
rozmiarach rzędu setek kilometrów. Dzięki temu połączone działanie
niestabilności strumieniowej i niestabilności grawitacyjnej jest w stanie obejść
\emph{metrowej bariery wzrostu} o którym była mowa w
rozdziale~\ref{sec:paradigm}.
%
\par Wyciągając wnioski odnośnie skuteczności opisywanego modelu w przełamaniu
\emph{metrowej bariery wzrostu} należy pamiętać o pewnych uproszczeniach i
przyjętych założeniach. Głównym niedostatkiem modelu jest przeszacowany
początkowy stosunek gęstości pyłu do gęstości gazu. Dla przypomnienia jest on
ponad stukrotnie wyższy niż wartość kanoniczna~\cite{FS03}. Jak pokazują
przedstawione w rozdziale~\ref{sec:sim_2d} wyniki symulacji dwuwymiarowych,
niższy stosunek gęstości pyłu do gęstości gazu wydłuża tempo wzrostu
niestabilności strumieniowej i jest dużo trudniejszy do śledzenia przy użyciu
zastosowanych w tej pracy metod numerycznych opisanych w
rozdziale~\ref{sec:metodyka}. 

Jednakże całkowita masa pyłu zawarta w domenie obliczeniowej wynosi około
$35\Mearth$ i co do rzędu wielkości odpowiada masie skalistych planet i jąder
planet gazowych Układu Słonecznego.

\par Teoria akrecji na jądra silnie zależy od procesu odpowiedzialnego za
formowanie się planetezymali~\cite{HBP13}. Warunek początkowy w przeprowadzonych 

Poczatkowa funkcja masy planetezymali nie jest okreslona. Przyjmuje sie ze ma
postać funkcji potegowej (bądź złożenia dwóch funkcji potęgowych~\cite{R03})

Zależność od parametru $\alpha$ w kryterium związania: dla $0.1$ tylko $8\%$.
Nie ma dobrego kryterium na zwiazanie obłoku, potrzeba duzo wiekszej rozdzielczosci

Typowe rozmiary otrzymanych w symulacji BD3dS zagęszczęń mieszczą się w
przedziale $10^{11} \div 10^{12}\cm$, zaś ich srednie gęstości wynoszą $10^{-9}
\div 10^{-8}\g/\cm^3$ i są one spójne z charakterystykami obiektów, które są
mogą tworzyć zwarte planetezymale w toku dalszej ewolucji~\cite{HS08}.

Dopiero uwzględnienie ewolucji obłoku (cytacja), dluzsza skala czasowa
pozwalajaca uwzglednic dynamike wiekowa i efekty migracyjne (cytacja).

Rozszerzenie moze isc w dwie strony: bardziej realistyczny model dysku t.j.
stratyfikacja~(Johansen ostatnie prace) i pole magnetyczne, rozwoj
algorytmniczny t.j. adaptywna siatka niezbedna do sledzenia zapadania sie
obloku, podskalowy model ewolucji pylu np.  metodami MC~(Aska)

Typowe rozmiary obiektow związanych grawitacyjnie wynoszą od kilkudziesięciu do
kilkuset komórek, co daje $7^3$ -- dużo za mało żeby móc mówic o rozdzieleniu
niestabilności grawitacyjnej

\par The instability saturates when dust overdensities synchronize their
velocity with the gas component. We must note that despite the steady outward
migration of the gas, dust blobs never really gain positive radial velocity and
never cease to migrate inwards (this was also noted by~\cite{JY07}). Though in
certain cases, their radial velocity is significantly slower than what is
expected from the standard estimation of the migration rate. The lack of
additional force that would keep the blobs bounded also leads to high dispersion
rate of the overdensities.

%%%%%%%%%%%%%%%%%%%%%%%%%%%%%%%%%%%%%%%%%%%%%%%%%%%%%%%%%%%%%%%%%%%%%%%%%%%%%%%%
% vim: tw=80 ts=3: 
