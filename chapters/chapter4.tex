\begin{savequote}[75mm]
   If I model a phenomenon accurately, does that mean I understand it? Or might it be simple coincidence, or an artifact
   of the technique? Of course, as an ardent simulationist, I myself put much faith in Engine-modeling.
\qauthor{Edward Mallory, The Difference Engine by William Gibson and Bruce Sterling}
\end{savequote}

\chapter{Dyskusja}

Zależność od parametru $\alpha$ w kryterium związania: dla $0.1$ tylko $8\%$.

We performed a set of 2D simulations and 3D simulation of two mutually
interacting fluids, i.e. gas and dust in protoplanetary disc, while neglecting
vertical component of gravity. For the spectrum of grain sizes ranging from
$10\div50$~cm we observe rapid growth of unstable modes in dust component, which
are consistent with modes obtained for streaming instability within the
framework of local linear analysis.

Our model enhances previous work of other authors by taking into account the
full dynamics of protoplanetary disc, e.g. radial migration, that lead to
significant variation in physical quantities, such as gas pressure gradient, and
were previously treated as constant. Moreover instead of using a general
dimensionless stopping time parameter to couple both fluid components, we
implement specific drag law (Eqn.~\mref{eq:tauf}) which is best suited to the
physical properties of simulated fluids.

\par In the present quasi-global approach we relieved the restriction of the
fixed  radial pressure gradient, adopted for numerical modeling of the streaming
instability within the shearing box approximation.  It remains unknown how much
this term affects the non-linear evolution of the system, since the additional
term acts as an infinite energy reservoir driving the difference in azimuthal
velocities of both fluids. In the global approach, even around fixed orbit, the
parameter $\eta$, measuring the radial pressure gradient, is function of time.
While locality of shearing box fully justifies using a constant dimensionless
stopping time to calculate mutual linear drag force, that approximation is no
longer valid in the global disc. 

\par The wave numbers of the fastest growing modes, extracted from numerical
models, and their positions at  the  map of growth rate  $s$ vs. $k_x$ and $k_z$
coincide with the predictions of linear analysis of streaming instability. We
reproduce reasonably well the linear growth rates, getting rapid convergence
near $\sim 32$ cells per wavelength. It is apparent, however,  that due to
diffusive nature of the Relaxing TVD scheme, PIERNIK needs high resolution to
capture linear growth phase accurately.

\par The instability saturates when dust overdensities synchronize their
velocity with the gas component. We must note that despite the steady outward
migration of the gas, dust blobs never really gain positive radial velocity and
never cease to migrate inwards (this was also noted by~\cite{JY07}). Though in
certain cases, their radial velocity is significantly slower than what is
expected from the standard estimation of the migration rate. The lack of
additional force that would keep the blobs bounded also leads to high dispersion
rate of the overdensities.

\par In our simulations we have neglected the role of stratification of the
disc, self-gravity of both fluids and also MRI induced turbulence. As each and
every of the aforementioned physical processes plays significant and integral
role in the protoplanetary disc evolution, we plan to gradually expand our setup
increasing its complexity in order to create the more complete global model.

\par We have found that nonlinear evolution of  the streaming instability forms
conditions for the formation  of gravitationally bound dust blobs in the
examined 3D semi-global discs configuration.  This indicates, in accordance with
the predictions by~\citet{J07}, the possibility  for  planetesimals formation in
accreting circumstellar discs,  and therefore we plan to incorporate selfgravity
in our future work.

\par We note that some of our runs may be underresolved, especially the
gas-dominated setups. However the nonlinear outcomes of all setups converge to
common solutions. Moreover, our initial simulation of full 3D setup, which took
0.36 MCPUh, shows that we are able to perform global simulation of streaming
instability in a convincing manner. 
However, due to the severe CFL constraint on time-step imposed by the high
azimuthal velocity in 3d simulations, we consider implementing {\sc FARGO}-like
algorithm~\cite{M00} for our future work.
This is an important step towards complete model of planetesimals formation due
to combined action of various fluid instabilities postulated by~\citet{J07}. 
