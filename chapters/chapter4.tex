\begin{savequote}[75mm]
   If I model a phenomenon accurately, does that mean I understand it? Or might
   it be simple coincidence, or an artifact of the technique? Of course, as an
   ardent simulationist, I myself put much faith in Engine-modeling.
\qauthor{Edward Mallory, The Difference Engine by William Gibson and Bruce
   Sterling}
\end{savequote}

\chapter{Dyskusja}
W ramach tej pracy przeprowadzono serię dwu- i trójwymiarowych symulacji
wzajemnie oddziałujących płynów: gazu i pyłu, w dysku keplerowskim. Przy czym
zaniedbano pionową składową grawitacji od masy punktowej umieszczonej w centrum
układu współrzędnych. Dla ziaren pyłu o rozmiarach od 10 do 50$\cm$
zaobserwowano gwałtowny wzrost zaburzeń w gęstości i prędkości pyłu na
charakterystyczych długościach fali zgodnych z przewidywaniami lokalnej,
liniowej analizy stabilności takiego układu. Wcześniejsze badania innych
autorów~\cite{YG05, JY07, TB09, BS10a, BS10b} opierały się w pełni na lokalnym
przybliżeniu kostki ścinanej, co prowadziło do szeregu uproszczeń m.in.
traktowania globalnego, radialnego ciśnienia gazu jako stałego i niezmiennego
współczynnika~\cite{N86}, stosowania bezwymiarowego czasu zatrzymania przy
obliczaniu wzajemnego oddziaływania obu płynów~\cite{YG05}. Bardzo ważnym
osiągnięciem poniższej pracy jest rozszerzenie modelu stosowanego przez innych
autorów poprzez uwzględnienie pełnej dynamiki okołogwiazdowego dysku w kierunku
radialnym i wykazanie, że w takim przypadku można wzbudzić niestabilność
strumieniową. Ponadto do opisu oddziaływania wiążącego oba płyny użyto
właściwego dla ich charakterystyk prawa tarcia aerodynamicznego~\mref{eq:tauf}
zamiast przybliżonego rachunku opartego na stałym, bezwymiarowym czasie
zatrzymania.

\par Przyjęty model, choć stanowi wyraźny postęp względem innych prac, jest
ciągle mało realistycznym przybliżeniem pełnego dysku protoplanetarnego. Do jego
największych wad można zaliczyć brak stratyfikacji oraz nieuwzględnienie wpływu
globalnej turbulencji wywołanej niestabilnością magneto-rotacyjną~\cite{DKJ14}. 


\par Otrzymane, we wszystkich przeprowadzonych eksperymentach numerycznych,
liczby falowe najszybciej rosnących modów niestabilności, a także ich pozycja na
mapie stabilności ($s(k_x, k_z)$), są zgodne z przewidywaniami liniowej analizy
niestabilności strumieniowej. Pierwszy etap pracy opierał się na wykonaniu
testów zbieżnościowych dla różnych parametrów fizycznych dysku protoplanetarnego
co pozwolilo na określenie minimalnej liczby komórek obliczeniowych
$(n\approx32)$ dla kodu \textsc{PIERNIK},umożliwiającej na dokładne odwzorowanie
liniowej fazy wzrostu. 

Model akrecji na jądra silnie zależy od procesu odpowiedzialnego za formowanie
się planetezymali.


Niestabiność strumieniowa zageszcza pył nawet 100 krotnie.


Zależność od parametru $\alpha$ w kryterium związania: dla $0.1$ tylko $8\%$.
Nie ma dobrego kryterium na zwiazanie obłoku, potrzeba duzo wiekszej rozdzielczosci


Należy podkreślić, że pomimo założenia dużo większej niż kanoniczna wartości
stosunku gęstości gazu do gęstości pyłu, symulowany dysk jest stabilny zgodnie z
kryterium Jeansa.  Parameter $Q$ jest stały dla całej domeny obliczeniowej i
wynosi około $40$ (rów.~\ref{eq:Qemp}).

\par Typowe rozmiary otrzymanych w symulacji BD3dS zagęszczęń mieszczą się w
przedziale $10^{11} \div 10^{12}\cm$, zaś ich srednie gęstości wynoszą $10^{-12}
\div 10^{-10}\g/\cm^3$ i są one spójne z charakterystykami obiektów, które są
mogą tworzyć zwarte planetezymale w toku dalszej ewolucji~\cite{HS08}.

Dopiero uwzględnienie ewolucji obłoku (cytacja), dluzsza skala czasowa
pozwalajaca uwzglednic dynamike wiekowa i efekty migracyjne (cytacja).


Rozszerzenie moze isc w dwie strony: bardziej realistyczny model dysku t.j.
stratyfikacja~(Johansen ostatnie prace) i pole magnetyczne, rozwoj
algorytmniczny t.j. adaptywna siatka niezbedna do sledzenia zapadania sie
obloku, podskalowy model ewolucji pylu np.  metodami MC~(Aska)



\par In the present quasi-global approach we relieved the restriction of the
fixed  radial pressure gradient, adopted for numerical modeling of the streaming
instability within the shearing box approximation.  It remains unknown how much
this term affects the non-linear evolution of the system, since the additional
term acts as an infinite energy reservoir driving the difference in azimuthal
velocities of both fluids. In the global approach, even around fixed orbit, the
parameter $\eta$, measuring the radial pressure gradient, is function of time.
While locality of shearing box fully justifies using a constant dimensionless
stopping time to calculate mutual linear drag force, that approximation is no
longer valid in the global disc. 


\par The instability saturates when dust overdensities synchronize their
velocity with the gas component. We must note that despite the steady outward
migration of the gas, dust blobs never really gain positive radial velocity and
never cease to migrate inwards (this was also noted by~\cite{JY07}). Though in
certain cases, their radial velocity is significantly slower than what is
expected from the standard estimation of the migration rate. The lack of
additional force that would keep the blobs bounded also leads to high dispersion
rate of the overdensities.

\par In our simulations we have neglected the role of stratification of the
disc, self-gravity of both fluids and also MRI induced turbulence. As each and
every of the aforementioned physical processes plays significant and integral
role in the protoplanetary disc evolution, we plan to gradually expand our setup
increasing its complexity in order to create the more complete global model.

\par We have found that nonlinear evolution of  the streaming instability forms
conditions for the formation  of gravitationally bound dust blobs in the
examined 3D semi-global discs configuration.  This indicates, in accordance with
the predictions by~\citet{J07}, the possibility  for  planetesimals formation in
accreting circumstellar discs,  and therefore we plan to incorporate selfgravity
in our future work.

\par We note that some of our runs may be underresolved, especially the
gas-dominated setups. However the nonlinear outcomes of all setups converge to
common solutions. Moreover, our initial simulation of full 3D setup, which took
0.36 MCPUh, shows that we are able to perform global simulation of streaming
instability in a convincing manner. 
However, due to the severe CFL constraint on time-step imposed by the high
azimuthal velocity in 3d simulations, we consider implementing {\sc FARGO}-like
algorithm~\cite{M00} for our future work.
This is an important step towards complete model of planetesimals formation due
to combined action of various fluid instabilities postulated by~\citet{J07}. 

%%%%%%%%%%%%%%%%%%%%%%%%%%%%%%%%%%%%%%%%%%%%%%%%%%%%%%%%%%%%%%%%%%%%%%%%%%%%%%%%
% vim: tw=80 ts=3: 
