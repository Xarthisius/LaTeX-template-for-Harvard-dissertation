\begin{savequote}[75mm]
   Va'esse deireádh eap eigean$\ldots$
\qauthor{Saga o wiedźminie, Andrzej Sapkowski}
\end{savequote}

\chapter{Dyskusja i plany na przyszłość}
Przedmiotem rozważań zaprezentowanych w niniejszej pracy jest niestabilność
strumieniowa, wynikająca z oddziaływania pyłu i gazu w dysku protoplanetarnym, z
udziałem i bez udziału samograwitacji. Przedstawione w poprzednich rozdziałach
wyniki opierają się na serii symulacji numerycznych fragmentu dysku
protoplanetarnego w dwu i trzech wymiarach oraz liniowej analizie stabilności,
wykonanej dla przypadku dwuwymiarowego.
%
%Przy czym
%zaniedbano pionową składową grawitacji od masy punktowej umieszczonej w centrum
%układu współrzędnych.
%
W eksperymentach numerycznych, przeprowadzonych dla ziaren pyłu o rozmiarach od
10 do 50$\cm$, zaobserwowano gwałtowny wzrost zaburzeń w gęstości i prędkości
pyłu o charakterystycznych długościach fali, zgodnych z przewidywaniami lokalnej,
liniowej analizy stabilności takiego układu. Wcześniejsze symulacje numeryczne
innych autorów~\cite{YG05, JY07, TB09, BS10a, BS10b} opierały się na lokalnym
przybliżeniu kostki ścinanej. Prowadziło to do szeregu uproszczeń, m.in.
traktowania globalnego, radialnego ciśnienia gazu jako stałego i niezmiennego
współczynnika~\cite{N86}, stosowania bezwymiarowego czasu zatrzymania przy
obliczaniu wzajemnego oddziaływania obu płynów~\cite{YG05}. Głównym
osiągnięciem niniejszej pracy jest rozszerzenie modelu stosowanego przez innych
autorów poprzez uwzględnienie pełnej dynamiki okołogwiazdowego dysku w kierunku
radialnym i wykazanie, że w takim przypadku dostępne algorytmy i zasoby
obliczeniowe umożliwiają wiarygodne modelowanie niestabilności
strumieniowej. Istotnym rozszerzeniem wcześniejszych modeli fizycznych było 
użycie właściwego prawa tarcia aerodynamicznego~\mref{eq:tauf}
zamiast przybliżonego rachunku opartego na stałym, bezwymiarowym czasie
zatrzymania. 
%
Przedstawiony model globalnego dysku jest rozwinięciem dotychczas
publikowanych modeli lokalnych. Należy jednakże podkreślić, że model zaprezentowany 
w niniejszej pracy opiera się na uproszczających założeniach, do których należy
zaliczyć brak stratyfikacji, pominięcie pola magnetycznego skutkujące
brakiem niestabilności magnetorotacyjnej, uwzględnienie jednej tylko frakcji
aglomeratów pyłowych o wybranym rozmiarze oraz przeszacowanie początkowej masy
pyłu.
%
%Do jego największych wad można zaliczyć brak stratyfikacji oraz nieuwzględnienie
%wpływu globalnej turbulencji wywołanej niestabilnością
%magneto-rotacyjną~\cite{DKJ14}. 

%
\par Pierwszy etap niniejszej pracy opierał się na wykonaniu testów
zbieżnościowych dla różnych parametrów fizycznych dysku protoplanetarnego, co
pozwoliło na określenie minimalnej liczby komórek obliczeniowych $(n\approx32)$
dla kodu \textsc{PIERNIK}, umożliwiającej na dokładne odwzorowanie liniowej fazy
wzrostu.  Otrzymane we wszystkich przeprowadzonych eksperymentach numerycznych
liczby falowe najszybciej rosnących modów niestabilności, a także ich położenia na
mapie stabilności ($s(k_x, k_z)$), są zgodne z przewidywaniami liniowej analizy
niestabilności strumieniowej i potwierdzają właściwy wybór użytych metod
numerycznych. Kolejnym krokiem było wykonanie symulacji trójwymiarowych bez
samograwitacji i porównanie ich wyniku z wcześniejszymi symulacjami
dwuwymiarowymi. Uzyskane wyniki pokazują, iż trójwymiarowe symulacje odtwarzają w
pełni przebieg ewolucji niestabilności strumieniowej, zgodnie z przewidywaniami
liniowej analizy stabilności, przeprowadzonej dla układu zredukowanego do dwóch
wymiarów. Finalnym etapem pracy było przeprowadzenie w pełni trójwymiarowej
symulacji dwuskładnikowego dysku protoplanetarnego z uwzględnieniem wpływu
samograwitacji. Należy podkreślić, że pomimo założenia dużo większej niż
kanoniczna wartości stosunku gęstości gazu do gęstości pyłu, symulowany dysk
jest początkowo stabilny zgodnie z kryterium Toomre'a~\mref{eq:toomre}.
Parametr $Q$ jest stały dla całej domeny obliczeniowej i wynosi około $40$
(rów.~\ref{eq:Qemp}). Wzbudzenie niestabilności strumieniowej prowadzi do
uformowania lokalnych zagęszczeń pyłu, których gęstość jest nawet stukrotnie
większa niż gęstość początkowa pyłu. Ilość zgromadzonej lokalnie masy pyłu
przekracza granicę stabilności grawitacyjnej i~w~rezultacie w~układzie formuje
się znaczna liczba związanych grawitacyjnie obiektów, które należy interpretować
jako zalążki planetezymali. Zgromadzona w tych obiektach materia
pyłowa reprezentuje $50\cm$ aglomeraty pyłowe, a ich całkowita masa jest
wystarczająca, aby po ,,sprasowaniu'' na skutek grawitacji powstały zwarte ciała o
rozmiarach rzędu setek kilometrów. Dzięki temu połączone działanie
niestabilności strumieniowej i niestabilności grawitacyjnej skutkuje pokonaniem
\emph{metrowej bariery wzrostu} aglomeratów pyłowych, o którym była mowa w
rozdziale~\ref{sec:paradigm}.
%
\par 
Oceniając przydatność przedstawionego modelu należy pamiętać o zastosowanych
przybliżeniach. Głównym niedostatkiem modelu jest przeszacowany początkowy
stosunek gęstości pyłu do gęstości gazu. Dla przypomnienia, jest on ponad
stukrotnie wyższy niż wartość kanoniczna, typowa dla ośrodka
międzygwiazdowego~\cite{FS03}. Jak pokazują przedstawione w rozdziale~\ref{sec:sim_2d}
wyniki symulacji dwuwymiarowych, niższy stosunek gęstości pyłu do gęstości gazu
wydłuża tempo wzrostu niestabilności strumieniowej, dlatego modelowanie
takiego przypadku jest  dużo trudniejsze i bardziej kosztowne
obliczeniowo.  Jednakże całkowita masa pyłu zawarta w domenie obliczeniowej o
rozpiętości $1/12$ pełnego kąta azymutalnego wynosi około $35\Mearth$ i~co do rzędu
wielkości odpowiada masie skalistych planet i jąder planet gazowych Układu
Słonecznego. Biorąc pod uwagę, że użyty w symulacjach radialny profil gęstości
MMSN jest tylko \emph{dolnym ograniczeniem} rzeczywistego rozkładu gęstości
materii w dysku protoplanetarnym, to przedstawiony w niniejszej pracy model
ciągle pozostaje zgodny z antycypowanymi warunkami początkowymi dla Układu
Słonecznego~\cite{D07}.
\par Przewidywania teorii ,,akrecji na jądra'' istotnie zależą od procesu
odpowiedzialnego za formowanie się planetezymali~\cite{HBP13}. Początkowa
funkcja masy planetezymali nie jest dobrze określona. Przyjmuje się, że ma
postać funkcji potęgowej, bądź złożenia dwóch funkcji potęgowych~\cite{R03}.
Wyniki przedstawione w rozdziale~\ref{sec:sim_3d} pozwalają określić spektralny
rozkład masy zgromadzonej w grawitacyjnie związanych obiektach, stanowiących
zgodnie z przyjętym modelem zalążki planetezymali.  Zakładając, że rozkład masy
grawitacyjnie związanych obiektów pyłowych jest dany funkcją:
%
\begin{equation}
   f(m) \propto m^{-a},
\end{equation}
%
w rozdziale~\ref{sec:sim_3d} dopasowano do otrzymanego rozkładu masy funkcję o
wykładniku $a = 1.25\pm0.12$. Wykładnik ten mieści się w zakresie przewidywanym
przez innych
autorów~\cite{R03}, którzy szacują jego wartość między $1$ a $3$. Ponadto prawy
skraj funkcji masy (Rysunek~\ref{fig:massfun} można opisać funkcją o wykładniku
$a$ w przedziale $2 < a < 3$, który jest zbieżny z wynikami symulacji N--ciałowych
formowania się planet~\cite{MFFK98}.  Należy jednak zauważyć, iż przedstawiona
na rysunku~\ref{fig:massfun} funkcja masy utworzonych w symulacji grawitacyjnie
związanych obiektów może być obarczona sporymi błędami systematycznymi.
W szczególności lewy skraj jest ograniczony poprzez skończoną rozdzielczość
siatki obliczeniowej: typowe rozmiary obiektów związanych grawitacyjnie wynoszą
od kilkudziesięciu do kilkuset komórek, co przekłada się na rozmiar liniowy
obiektów związanych grawitacyjnie rzędu kilku komórek obliczeniowych. Jest to
wielkość wysoce niewystarczająca do poprawnego odwzorowania niestabilności
grawitacyjnej w takim obiekcie.  Prawy skraj natomiast zawiera mało
reprezentatywną statystycznie liczbę obiektów.
%
\par Typowe rozmiary otrzymanych w symulacji BD3dS grawitacyjnie związanych
zagęszczeń mieszczą się w przedziale od $10^{11}$ do $10^{12}\cm$ i są one spójne z
charakterystykami obiektów, które mogą tworzyć zwarte planetezymale w toku
dalszej ewolucji~\cite{HS08}. Średnie, końcowe gęstości pyłu nie są miarodajne
ze względu na przypuszczalny wpływ ograniczonej rozdzielczości siatki
obliczeniowej. Należy jednak mieć na uwadze, że dodatkową niepewność dotyczącą
liczby powstających obiektów wprowadza podskalowa turbulencja, sparametryzowana
współczynnikiem $\alpha$, występującym w kryterium związania
grawitacyjnego\mref{eq:ekin}.
%Jak pokazano w rozdziale~\ref{sec:sim_3d} dla
%$\alpha = 0.1$ masa pyłu zgromadzona w związanych grawitacyjnie obiektach
%stanowi tylko $5\%$ całej masy pyłu obecnej w dysku.  Jednakże oczekiwane
%wartości $\alpha$ w dyskach protoplanetarnych oscylują wokół wartości
%$0.01$~\cite{FD11}, dla których efektywnie związana w obiektach masa stanowi
%ponad $15\%$ całkowitej masy pyłu w dysku.
W rozdziale~\ref{sec:sim_3d} oszacowano, że wybór parametru $\alpha$ w
niewielkim stopniu zmienia kryterium grawitacyjnego związania obiektów pyłowych.
Dalszym krokiem pozwalającym na rozwinięcie przedstawionego w~niniejszej pracy
modelu dysku protoplanetarnego powinno być uwzględnienie rzeczywistego źródła
turbulencji w dysku, np. poprzez rozszerzenie modelu o wpływ pola magnetycznego
i~uwzględnienie skutków niestabilności magnetorotacyjnej. Podobnie, wprowadzenie
stratyfikacji w dysku pozwoliłoby na uwzględnienie wpływu niestabilności
Kelvina--Helmholtza.  Niezbędne jest także zwiększenie rozdzielczości siatki
obliczeniowej w obszarach, które podlegają niestabilności grawitacyjnej. Byłoby
to możliwe dzięki zastosowaniu zaimplementowanego niedawno w kodzie
\textsc{PIERNIK} mechanizmu \emph{adaptywnej siatki obliczeniowej}\footnote{ang.
\emph{Adaptive Mesh Refinement}}. Pozwoliłoby to dokładniej śledzić proces
kolapsu grawitacyjnego i zdecydowanie poprawić oszacowanie uzyskiwanej
początkowej funkcji masy obiektów grawitacyjnie związanych.  Mimo tego, że
zastosowanie techniki siatek adaptywnych może efektywnie zwiększyć rozdzielczość
liniową siatki obliczeniowej o parę rzędów wielkości, to śledzenie w dysku
rozciągającym się na wiele jednostek astronomicznych obiektów, których rozmiary
nie przekraczają setek kilometrów, nadal nastręcza spore trudności.  Naturalnym
krokiem w takiej sytuacji jest rozszerzenie algorytmu numerycznego o tzw.
,,cząstki zbierające''\footnote{ang. \emph{,,sink particles''}}. ,,Cząstki
zbierające'' są punktami materialnymi, które oddziałują z otaczającym je
płynem tylko poprzez siły grawitacji i pochłaniają płyn znajdujący się wewnątrz
zadanego promienia akrecji, zwiększając swoją masę~\cite{FBCK10}. Wprowadzenie
,,cząstek zbierających'' do przedstawionych w~niniejszej pracy symulacji
pozwoliłoby m.in. na określenie tempa akrecji masy na związane grawitacyjnie
obiekty, co przy użytych dotychczas metodach numerycznych nie było możliwe.
Niezmiernie ważne jest również wydłużenie czasu trwania symulacji tak, aby
możliwe było uwzględnienie efektów migracji formujących się, grawitacyjnie
związanych obiektów~\cite{ML14}.

%\par Rozszerzenie może iść w dwie strony: bardziej realistyczny model dysku t.j.
%stratyfikacja~(Johansen ostatnie prace) i pole magnetyczne, rozwój
%algorytmiczny t.j. adaptywna siatka niezbędna do śledzenia zapadania się
%obłoku, podskalowy model ewolucji pyłu np.  metodami MC~(Aska)


%%%%%%%%%%%%%%%%%%%%%%%%%%%%%%%%%%%%%%%%%%%%%%%%%%%%%%%%%%%%%%%%%%%%%%%%%%%%%%%%
% vim: tw=80 ts=3: 
